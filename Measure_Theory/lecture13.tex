%! TeX root = ./main.tex
\begin{remark}
	This theorem implies for any $L_p$ function $f$, we can take
	simple functions $f_1,f_2,\dots\to f$ and $|f_n|\uparrow |f|$,
	so $f_n\xrightarrow{L_p} f$.
\end{remark}

\begin{definition}[Weak convergence]
	Let $1<p<\infty$, and $f_1,f_2\dots\in L_p$. If
	\[
	\lim_{n\to \infty}\int_X f_n g\dd\mu = \int_X fg\dd \mu,\quad
	\forall g\in L_q.
	\]
	Then we say $f_n$ \vocab{weak convergent} to $f$, denoted by
	$f_n \xrightarrow{(w)L_p} f$.

	When $p = 1$ and $(X, \mathscr{F}, \mu)$ is a $\sigma$-finite
	measure space, and the condition also holds,
	we say  $\{f_n\}$ weak convergent to $f$ in $L_1$.
\end{definition}

\begin{corollary}
    Let $1\le p < \infty$, then
	\[
		f_n \xrightarrow{L_p} f\implies f_n \xrightarrow{(w)L_p} f.
	\]
\end{corollary}
\begin{proof}[Proof]
    By Holder's inequality,
	\[
	\left| \int_X (f_n - f)g\dd \mu \right|\le \lVert f_n - f \rVert _p
	\lVert g \rVert _q \to 0.
	\]
\end{proof}

If $\sup_{t\in T} \lVert f_t \rVert _p =: M < \infty$,
then we say $\{f_t, t\in T\}$ is \vocab{bounded in $L_p$}.

\begin{theorem}
    Let $1<p<\infty$, $\{f_n\}\subset L_p$, there exists $M$ s.t.
	$ \lVert f_n \rVert _p \le M$, $\forall n$.
	If $f_n \to f, a.e.$ or in measure, then $f\in L_p$
	and $f_n \to f$ weakly.
\end{theorem}
\begin{proof}[Proof]
    First $ \lVert f \rVert _p \le M$:
	\[
	\int_X |f|^p\dd \mu\le \liminf_{n\to \infty}\int_X |f_n|^p\dd\mu \le M^p.
	\]

	Next we prove the weak convergence:
	For all $g\in L_q$, recall the bounded convergence theorem in probability,
	we can view  $M$ as a bound of $f_n$, and $ \lVert g \rVert _q$	as $P$.

	Let $B = \{|f_n - f|\le \hat{\varepsilon}\}$, consider
	\[
	a := \int_B (f_n - f)g \dd\mu, \quad
	b:= \int_{B^c}(f_n - f)g\dd \mu.
	\]
	Note that
	\[
	|a| \le \hat{\varepsilon}\int_X |g|\dd \mu.
	\]
	But $\int_X |g|\dd \mu$ might be infinity, so
	let $A_k := \{\frac{1}{k}\le |g|^q \le k\}$, we have
	\[
	\int_{A_k}|g|\dd \mu \le k^{\frac{1}{q}}\mu(A_k) < \infty.
	\]
	($\frac{1}{k}\mu(A_k) < \int_{A_k} |g|^q\dd \mu < \infty$ since $g\in L_q$).

	Now we can proceed:
	\[
	a:= \int_{A_kB} (f_n - f)g\dd\mu,\quad
	b:= \int{A_k^c\cup B^c} (f_n - f)g\dd \mu.
	\]
	Now $|a| \le \hat{\varepsilon} k^{\frac{1}{q}}\mu(A_k) < \varepsilon$.
	\[
	\left| \int_X (f_n - f)g \ii_{A_k^c\cup B^c}\dd \mu \right|
	\le \lVert f_n-f \rVert _p \lVert g\ii_{A_k^c\cup B^c} \rVert _q
	\le 2M \left( \int_{A_k^c}|g|^q\dd \mu +
	\int_{A_k\backslash B}|g|^q\dd\mu \right).
	\]
	By LDC(Dominated convergence), $A_k^c \to \{g = 0, \infty\}$,
	so $\int_{A_k^c} |g|^q\dd \mu < \varepsilon$.

	Since $\mu(A_k) < \infty$, $f_n \to f, a.e.\implies f_n\xrightarrow{\mu} f$.
	By the continuity of integrals, $\mu(A_k \backslash B) \le \mu(B^c)
	< \delta\implies \int_{A_k \backslash B} |g|^q \dd\mu < \varepsilon$.

	Now we can conclude: $\forall \varepsilon>0$,
	first choose $k$ large, then $\hat{\varepsilon}$ small, we get
	\[
	\int_X (f_n - f)g\dd \mu \le \varepsilon + 4M \varepsilon \implies
	f_n \xrightarrow{(w)L_p} f.
	\]
\end{proof}
\begin{remark}
    The proof is a little complicated, we divide the entire integral to
	three part, and estimate them respectively.
\end{remark}

When $p = 1$, $f_n$ bounded in $L_p$ \textit{cannot} imply weak convergence.
\begin{example}
    Let $X = \mathbb{N}$, $\mu(\{k\}) = 1, \forall k$,
	clearly it's $\sigma$-finite.

	Let $f_n(k) = \ii_{k=n}$, then $ \lVert f_n \rVert = \sum_k \mu(k)|f_n(k)| = 1$,
	and $f_n \to 0, a.e.$.

	But let $g = 1\in L_\infty$, $\int_X (f_n - f)g\dd \mu = 1\not \to 0$.
\end{example}
\begin{proposition}
	Let $f_1,f_2,\dots\in L_1$, then:
	\[
		\lVert f_n \rVert \to \lVert f \rVert \& f_n\to f, a.e.
		\implies f_n\xrightarrow{L_1} f \implies f_n\xrightarrow{(w)L_1}f
		\implies \int_A f_n\dd \mu \to \int_A f\dd \mu, \forall A.
	\]
\end{proposition}
\begin{proof}[Proof]
    For the last part let $g = \ii_A$, the rest is trivial.
\end{proof}

\subsection{Integrals in probability space}
\label{sub:Integrals in probability space}

We can also consider $L_p$ space in probability space  $(\Omega, \mathscr{F}, P)$.
\begin{theorem}
    Let $0<s<t<\infty$. Then $L_t \subset L_s$.
	If $s\ge 1$, we have $ \lVert f \rVert _s \le \lVert f \rVert _t$,
	with equality $f$ constant.
\end{theorem}
\begin{proof}[Proof]
    When $f\in L_t$, let $p = \frac{t}{s}, q = \frac{t}{t-s}$.
	\[
	\int_\Omega |f|^s \cdot 1\dd P \le \lVert |f|^s \rVert _p \lVert 1 \rVert _q
	= (E|f|^{sp})^{\frac{1}{p}} = (E|f|^t)^{\frac{1}{p}}.
	\]
	So $f\in L_s\implies L_t \subset L_s$.
	When $s\ge 1$,
	\[
	\lVert f \rVert _s^s\le (\lVert f \rVert _t)^{\frac{t}{p}}
	= \lVert f \rVert _t^s \implies \lVert f \rVert _s\le \lVert f \rVert _t.
	\]
\end{proof}

From this we know $L_\infty \subset L_p$,
and $ \lVert f \rVert _p \uparrow \lVert f \rVert _\infty$.

\begin{remark}
    This theorem does not hold for general space.
	Let $X = \mathbb{N}$, $\mu(\{n\}) = 1$, $f(n) = \frac{1}{n}$,
	then $f\in L_2\backslash L_1$.
\end{remark}

The expectation $Ef^k$ is called \vocab{$k$-order moment} of random variable $f$.

\begin{definition}[Uniformly integrable]
	Let $ \{f_t, t\in T\}$ be r.v.'s,
	if $\forall \varepsilon >0, \exists \lambda >0$, such that
	\[
		E|f_t|\ii_{\{|f_t| > \lambda\}} < \varepsilon, \quad \forall t\in T,
	\]
	then we say $\{f_t, t\in T\}$ \vocab{uniformly integrable}.
\end{definition}

If $\forall \varepsilon>0, \exists \delta>0$ s.t $\forall A\in \mathscr{F}$,
\[
P(A)<\delta \implies E|f_t|\ii_A < \varepsilon, \forall t\in T,
\]
we say $\{f_t\}$ is uniformly absolutely continuous, which is abbreviated as
absolutely continuous.

\begin{theorem}
    Uniformly integrable $\iff$ absolute continuity and bounded in $L_1$.
\end{theorem}
\begin{proof}[Proof]
    Firstly when $\{f_t\}$ uniformly integrable,
	$\forall A\in \mathscr{F}, \lambda>0$,
	\begin{align*}
	E|f_t|\ii_A &= E|f_t|\ii_{A\cap \{|f_t|\le \lambda\}}
	+ E|f_t|\ii_{A\cap \{|f_t|>\lambda\}}\\
	&\le \lambda P(A) + E|f_t|\ii_{\{|f_t|>\lambda\}}
	\end{align*}
	Let $A = X$ we know $E|f_t| \le \lambda + \frac{\varepsilon}{2},
	\forall t\in T$.
	Now let $\delta = \frac{\varepsilon}{2\lambda}$ we get AC property.

	On the other hand,
	\[
	\lambda P(|f_t|>\lambda) \le E|f_t|\ii_{\{|f_t|>\lambda\}}\le E|f_t|\le M,
	\forall t\in T.
	\]
	So when $\lambda > \frac{M}{\delta}$, $P(|f_t|>\lambda) < \delta$,
	hence $E|f_t|\ii_{\{|f_t|>\lambda\}}\le \varepsilon$, $\forall t\in T$.
\end{proof}

\begin{theorem}
    Let $0<p<\infty$, and $f_n\to f$ in probability. TFAE:
	\begin{enumerate}[\indent (1)]
		\item $\{|f_n|^p\}$ uniformly integrable;
		\item $f_n \xrightarrow{L_p} f$;
		\item $f\in L_p$ and $ \lVert f_n \rVert _p \to \lVert f \rVert _p$.
	\end{enumerate}
\end{theorem}
\begin{proof}[Proof]
    $(1)\implies (2)$: Take subsequence $f_{n'} \to f, a.s.$,
	\[
	E|f|^p \le \liminf_{n\to \infty} E|f_n|^p < \infty,
	\]
	since $\{|f_n|^p\}$ is bounded in $L_1$. This means $f\in L_p$.

	Let $A_n = \{|f_n - f|> \varepsilon\}$, now we compute
	\[
	E|f_n - f|^p \le \varepsilon^p + E|f_n - f|^p \ii_{A_n}
	\le \varepsilon^p + C_p E|f_n|^p \ii_{A_n} + C_p E|f|^p\ii_{A_n}
	\]
	Since $P(A_n) \to 0$ and $\{|f_n|^p\}$ absolutely continuous
	(also note $E|f|^p \ii_{A_n} \to 0$), RHS converges to 0.
	Therefore $f_n \xrightarrow{L_p} f$.

	As for $(3)\implies (1)$, we'll prove a lemma:
	\begin{lemma}
		If $f_n \xrightarrow{P} f$, then $\forall 0<p<\infty$,
		\[
		|f_n|^p \ii_{\{|f_n|\le \lambda\}} \xrightarrow{P}
		|f|^p \ii_{\{|f|\le \lambda\}}, \quad \forall \lambda \in C(F_{|f|}).
		\]
	\end{lemma}

	By lemma and bounded convergence theorem, their expectation also converges.
	Note that $ \lVert f_n \rVert _p \to \lVert f \rVert _p$, so
	\[
	E|f_n|^p\ii_{\{|f_n|>\lambda\}} \to E|f|^p\ii_{\{|f|>\lambda\}},
	\]
	thus $\forall \varepsilon>0, \exists \lambda_0\in C(F_{|f|})$, s.t.
	$E|f|^p\ii_{\{|f|>\lambda_0\}}$ TODO!

	\begin{proof}[Proof of the lemma]
	    Since $|f_n| \to |f|$ in probability, WLOG $f_n ,f \ge 0$.
		Define
		\[
		A_n := \{f_n \le \lambda\}\Delta\{f\le \lambda\} \cap
		\{|f_n^p-f^p|<\varepsilon\}
		\]
		\[
		B_n := \{f_n, f\le \lambda, |f_n^p - f^p| < \varepsilon\}.
		\]
		Since $x^p$ is uniformly continuous in $[0, \lambda]$,
		$B_n \subset \{|f_n - f| > \kappa_{\varepsilon, \lambda}\}$,
		$P(B_n) \to 0$.

		Also $P(A_n) \to 0$ as
		\[
		A_n \subset \{\lambda-\delta<f\le \lambda+\delta\}
		\cup \{|f_n - f|\ge \delta\},
		\]
		and $F_{|f|}$ continuous at $\lambda$.
	\end{proof}
\end{proof}
