%! TeX root = ./main.tex
\section{Measure spaces}
\label{sec:Measure spaces}

\subsection{The definition of measure and its properties}
\label{sub:The definition of measure and its properties}

The concept of ``measure'' is frequently used in our everyday life:
length, area, weight and even propbability.
They all share a similarly: the measure of a whole is equal to the sum of
the measure of each part.

In the language of mathematics, let $\mathscr{E}$ be a collection of sets,
and there's a function $\mu: \mathscr{E}\to [0,\infty]$ which
stands for the measure.

\vocab{countable additivity}: Let $A_1,A_2,\dots\in \mathscr{E}$ be pairwise disjoint
sets, and $\sum_{i=1}^\infty A_i \in \mathscr{E}$, then
\[
\mu \left( \sum_{i=1}^\infty A_i \right) = \sum_{i=1}^\infty \mu(A_i).
\]

\begin{definition}[Measure]
	Suppose $\emptyset\in \mathscr{E}$, if a non-negative function
	\[
		\mu: \mathscr{E}\to [0,\infty]
	\]
	satisfies countable additivity, and $\mu(\emptyset)=0$, then
	we say  $\mu$ is a \vocab {measure} on $\mathscr{E}$.
\end{definition}

If $\mu(A)<\infty$ for all $A\in \mathscr{E}$, we say $\mu$ is finite.
(In practice we'll just simplify this to $\mu(X)<\infty$)

If $\exists A_1,A_2,\dots\in \mathscr{E}$ are pairwise disjoint sets, s.t.
\[
X = \sum_{n=1}^{\infty} A_n, \quad \mu(A_n)<\infty, \forall n.
\]
Then we say $\mu$ is  $\sigma$-finite.

There's a weaker version of countable additivity, that is \vocab{finite additivity}:
If $A_1,\dots,A_n\in \mathscr{E}$, pairwise disjoint, and $\sum A_i\in \mathscr{E}$,
\[
\mu\left( \sum_{i=1}^{n} A_i \right)=\sum_{1=i}^{n} \mu(A_i),
\]
then we say $\mu$ is finite additive.

Subtractivity: $\mu(B-A)=\mu(B)-\mu(A)$, where  $A,B,B-A\in \mathscr{E}$,
and $\mu(A)<\infty$.

\begin{proposition}
	Measure satisfies finite additivity and subtractivity.
\end{proposition}

\begin{example}[Counting measure]
    Let $\mu(A)=\#A$,  $\forall A\in \mathscr{T}_X$.
	Then $\mu$ is a measure.
\end{example}

\begin{example}[Point measure]
    Let $(X,\mathscr{F})$ be a measurable space,
	define $\delta_x(A) = \ii_A(x)$. Then we can define a measure
	\[
	\mu(A) = \sum_{i=1}^{n} p_i\delta_{x_i}(A)
	\]
\end{example}

\begin{example}[Length]
	Let $\mathscr{E}=\mathscr{Q}_{\mathbb{R}}=\{(a,b] : a,b\in \mathbb{R}\}$, $a\le b$,
	then $\mu((a,b]) = b - a$ gives a measure.
\end{example}

Another classical example is the so-called ``coin space'':

Let $X=\{x=(x_1,x_2,\dots): x_i\in {0,1}, \forall n\}$.
\[
C_{i_1,\dots,i_n}:=\{x:x_1=i_1,\dots, x_n=i_n\},
\]
Let
\[
\mathscr{Q} = \{\emptyset,X\}\cup
\left\{C_{i_1,\dots,i_n}:n\in \mathbb{N}, i_1,\dots,i_n\in \{0,1\}\right\}
\]
be a semi-ring.
Then $\mu(C_{i_1,\dots,i_n}) = \frac{1}{2^n}$ gives a measure.

We need to check the countable additivity, but actually
this can be realized as a compact space and the $C$'s are open sets,
so in fact we only need to check finite additivity.
(Or we can prove this explicitly)

Another more complex example: finite markov chain.

\begin{proposition}
	Let $X=\mathbb{R}$, $\mathscr{E}=\mathscr{R}_{\mathbb{R}}$.
	$F: \mathbb{R}\to \mathbb{R}$ is non-decreasing, right continuous, then
	$\mu((a,b]) = F(b)-F(a)$ gives a measure on $\mathscr{E}$.
\end{proposition}
\begin{proof}[Proof]
    First $\mu(\emptyset)=0$, suppose
	 \[
		 \sum_{i=1}^{\infty} (a_i,b_i] = (a,b]. 
	\]
	Since every partial sum has measure at most $F(b_{n+1})-F(a_1) < F(b)-F(a)$,

	$\implies \sum_{i=1}^{n} \mu((a_i,b_i]) \le \mu((a,b])$.

	For the reversed inequality, first we prove that for intervals
	\[
		\bigcup_{i=1}^n (c_i,d_i] \supseteq (a,b] \implies
		\sum_{i=1}^{n} \mu((c_i,d_i]) \ge \mu((a,b]).
	\]
	This can be easily proved by induction, WLOG $b_{n+1} = \max_i b_i$.

	Our idea is to extend each $(a_i, b_i]$ a little bit to apply above inequality.

	For all  $\varepsilon>0$, take  $\delta_i>0$ s.t.
	 \[
	\tilde{b}_i := b_i + \delta_i, \quad F(\tilde{b}_i) - F(b_i) \le \frac{\varepsilon}{2}.
	\]
	Hence for all $\delta>0$, $\bigcup_{i=1}^\infty (a_i, \tilde{b}_i)\supseteq [a+\delta,b]$,
	by compactness exists a finite open cover.

	\[
	F(b) - F(a+\delta)\le \sum_{i=1}^{n} \left(F(\tilde b_i)-F(a_i)\right)\le
	\varepsilon+ \sum_{i=1}^{\infty} (F(b)-F(a)).
	\]
	Let $\varepsilon,\delta\to 0$ to conclude.
\end{proof}

\begin{definition}[Measure space]
	A triple $(X,\mathscr{F},\mu)$ is called a \vocab{measure space},
	if $(X,\mathscr{F})$ is a measurable space and $\mu$ is a measure on $ \mathscr{F}$.
\end{definition}

If $N\in \mathscr{F}$ s.t. $\mu(N)=0$, we say  $N$ is a \vocab{null set}.

A probability space is a measure space $(X,\mathscr{F},P)$ with $P(X)=1$. 

\begin{example}[Discrete measure]
	If $X$ is countable,  $p:X\to [0,\infty]$, $\mu(A) := \sum_{x\in A} p(x)$
	is a measure.
\end{example}

There are other important properties which we think a sensible measure would have:
\begin{itemize}
	\item Monotonicity: If $A,B\in \mathscr{E}$, $A \subset B$, then $\mu(A)\le \mu(B)$. 

	\item Countable subadditivity: $A_1,A_2,\dots\in \mathscr{E}$,
	\[
	\mu \left( \bigcup_{i=1}^\infty A_i \right)\le \sum_{i=1}^{\infty} \mu(A_i).
	\]

	\item Lower continuity: $A_1,A_2,\dots\in \mathscr{E}$ and $A_n \uparrow A\in \mathscr{E}$.
	\[
	\mu(A) = \lim_{n\to \infty}\mu(A_n).
	\]

	\item Similarly there's upper continuity (which requires $\mu(A_1)<\infty$).
\end{itemize}

\begin{theorem}
    The measure on a semi-ring has all the above properties.
\end{theorem}
\begin{proof}[Proof]
    In fact,
	\begin{itemize}
		\item Finite additivity $ \implies $ monotonicity, subtractivity;
		\item Countable additivity $\implies$ subadditivity, upper and lower continuity.
	\end{itemize}

	Here we only prove the subadditivity, since others are trivial.

	Let $A_1,A_2,\dots\in \mathscr{Q}$, and $\bigcup_{i=1}^\infty A_i\in \mathscr{Q}$.
	\[
	B_n := A_n \backslash \bigcup_{i=1}^{n-1} A_i \in r(\mathscr{Q})
	\implies B_n = \sum_{k=1}^{k_n} C_{n,k},\quad C_{n,k}\in \mathscr{Q}.
	\]
	\[
	A_n \backslash B_n \in r(\mathscr{Q}) \implies
	A_n \backslash B_n = \sum_{l=1}^{l_n} D_{n,l},\quad D_{n,l}\in \mathscr{Q}.
	\]
	
	Thus by countable additivity,
	\begin{align*}
		\mu\left(\bigcup_{i=1}^\infty A_i\right) &= \sum_{n=1}^{\infty} \mu(B_n)
		= \sum_{n=1}^{\infty}\left( \sum_{k=1}^{k_n} \mu(C_{n,k}) \right)\\
		&\le \sum_{n=1}^{\infty}\left( \sum_{k=1}^{k_n} \mu(C_{n,k})+ \sum_{l=1}^{l_n}\mu(D_{n,l}) \right)
		= \sum_{n=1}^{\infty} \mu(A_n). 
	\end{align*}
	
	Using similar technique we can deduce the upper and lower continuity.
\end{proof}
