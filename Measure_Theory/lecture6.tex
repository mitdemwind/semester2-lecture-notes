%! TeX root = ./main.tex
\begin{example}
    In probability, let $\mathscr{E}_1, \mathscr{E}_2$ be collections of sets.
	We say they're independent if
	\[
	P(AB) = P(A)P(B), \quad \forall A\in \mathscr{E}_1, B\in \mathscr{E}_2.
	\]

	By the previous theorem we can derive
	$\lambda(\mathscr{E}_1),\lambda(\mathscr{E}_2)$ are independent.

	If  $A_1,A_2,\dots$ satisfy
	\[
		P(A_{i_1}\cdots A_{i_k}) = P(A_{i_1}) \cdots P(A_{i_k}),
	\]
	we say they are independent.

	Let $\{1,2,\dots\} = I+J$, then the $\sigma$-algebra generated by
	\[
	\mathscr{E}_1 = \{A_\alpha\mid \alpha\in I\},\quad
	\mathscr{E}_2 = \{A_\alpha\mid \alpha\in J\}
	\]
	are independent.
\end{example}

\begin{theorem}[Measure extension theorem]
    Let $\mu$ be a measure on a semi-ring $\mathscr{Q}$, $\tau$ is the
	outer meaasure generated by $\mu$. We have
	\[
	\sigma(\mathscr{Q})\in \mathscr{F}_\tau,\quad \tau|_{\mathscr{Q}} = \mu.
	\]
\end{theorem}
\begin{remark}
    Any measure on a semi-ring $\mathscr{Q}$ can extend to the $\sigma(\mathscr{Q})$,
	and if  $\mu$ is $\sigma$-finite, the extension is unique.
\end{remark}

\begin{proof}[Proof]
    For any $A\in \mathscr{Q}$, let $B_1=A$, $B_n = \emptyset, n\ge 2$.
	Then $\tau(A)\le \sum\mu(B_n) = \mu(A)$.

	On the other hand, if  $A_1,A_2,\dots\in \mathscr{Q}$ s.t.
	$\bigcup_{n=1}^\infty A_n \supseteq A$, then
	\[
	\mu(A) = \mu\left(\bigcup_{n=1}^\infty \mu(AA_n)\right)
	\le \sum_{n=1}^{\infty} \mu(AA_n) \le \sum_{n=1}^{\infty} \mu(A_n).
	\]
	Thus $\tau(A) = \mu(A)$, where we used the fact that $\mu$ is countable subadditive.

	Next we prove $A\in \mathscr{F}_\tau$.
	We need to show that
	\[
	\tau(D) \ge \tau(D\cap A) + \tau(D\cap A^c).
	\]
	WLOG $\tau(D)<\infty$.
	Take $B_1,B_2,\dots \in \mathscr{Q}$ s.t.
	\[
	\bigcup_{n=1}^\infty B_n \supseteq D,\quad
	\sum_{n=1}^{\infty}\mu(B_n) < \tau(D)+\varepsilon.
	\]
	Denote $\hat{D} := B_n\in \mathscr{Q}$ for a fixed $n$.
	Suppose $\hat{D}\cap A^c = \hat{D}\backslash A = \sum_{i=1}^{n} C_i$.
	\[
	\mu(\hat{D}) = \mu(\hat{D}\cap A) + \sum_{i=1}^{n} \mu(C_i)
	\ge \tau(\hat{D}\cap A) + \tau(\hat{D}\cap A^c).
	\]

	Apply this inequality to each $B_n$,
	\[
	\tau(D)+\varepsilon > \sum_{n=1}^{\infty} (\tau(B_n\cap A) + \tau(B_n\cap A^c))
	\ge \tau(D\cap A) + \tau(D\cap A^c).
	\]
	this implies $\tau(D)\ge \tau(D\cap A) + \tau(D\cap A^c)
	\implies A\in \mathscr{F}_\tau$.

	At last by Caratheodory's theorem,
	$\tau$ is a measure on $\mathscr{F}_\tau\supseteq \sigma(\mathscr{Q})$.
\end{proof}

\begin{theorem}[Equi-measure hull]
    Let $\tau$ be the outer measure generated by $\mu$,
	\begin{itemize}
		\item $\forall A\in \mathscr{F}_\tau$, $\exists B\in \sigma(\mathscr{Q})$
			s.t. $B\supseteq A$ and $\tau(A) = \tau(B)$;
		\item If $\mu$ is $\sigma$-finite, then $\tau(B\backslash A) = 0$.
	\end{itemize}
\end{theorem}
\begin{remark}
    This theroem states that $\mathscr{F}_\tau$ is just $\sigma(\mathscr{Q})$
	appended with null sets.
\end{remark}

\begin{proof}[Proof]
	If $\tau(A) = \infty$, $B = X$ suffices.

    By definition, there exists $B_n = \bigcup_{k=1}^{k_n} B_{n,k}\supseteq A$
	s.t. $\tau(B_n)<\tau(A)+\frac{1}{n}$.
	Let $B = \bigcap_{n=1}^\infty B_n$, we must have $\tau(B) = \tau(A)$.

	Now for the second part, let $X = \sum_{n=1}^{\infty} A_n$,
	$A_n\in \mathscr{Q}$, $\mu(A_n)<\infty$.

	Since $A = \sum_{n=1}^{\infty}AA_n$,
	we have
	\[
	AA_n\in \mathscr{F}_\tau,\quad \tau(AA_n)\le \tau(A_n) = \mu(A_n) <\infty.
	\]
	Let $B_n\in \sigma(\mathscr{Q})$ s.t. $B_n \supseteq AA_n$
	and $\tau(B_n) = \tau(AA_n)<\infty$.
	Let $B := \bigcup_{n=1}^\infty B_n$ we have
	\[
	\tau(B-A) = \tau\left(\bigcup_{n=1}^\infty (B_n - AA_n)\right)\le
	\sum_{n=1}^{\infty}\tau(B_n -AA_n) = 0.
	\]
\end{proof}

Let $\mathscr{R},\mathscr{A},\mathscr{F}$ be the ring, algebra, $\sigma$-algebra
generated by $\mathscr{Q}$, respectively.
The outer measure $\tau$ restricts to a measure on each of these collections,
denoted by $\mu_1,\mu_2,\mu_3$.
Each $\mu_i$ can generate an outer measure $\tau_i$, but actually they're all
the same as our original $\tau$, since $\tau_i$ are ``build up'' from $\tau$,
intuitively $\tau_i$ cannot be any better than $\tau$. (The proof
says exactly the same thing, so I'll omit it)

\begin{proposition}
    Let $\mu$ be a measure on an algebra $\mathscr{A}$. $\tau$ is the outer
	measure generated by $\mu$, for all $A\in \sigma(\mathscr{A})$,
	if $\tau(A)<\infty$, then $\forall \varepsilon>0$, $\exists B\in \mathscr{A}$
	s.t. $\tau(A\Delta B)<\varepsilon$.
\end{proposition}
\begin{remark}
    In practice we often replace $\tau$ with a $\sigma$-finite
	measure $\mu$ on $\sigma(\mathscr{A})$.
	(Here $\sigma$-finite is on $\mathscr{A}$)
\end{remark}

\begin{proof}[Proof]
    Choose $B_1,B_2,\dots\in \mathscr{A}$ s.t.
	\[
	\hat{B} := \bigcup_{n=1}^\infty B_n \supseteq A, \quad
	\sum_{n=1}^{\infty}\mu(B_n) < \tau(A) + \frac{\varepsilon}{2}.
	\]
	Let $N$ be a sufficiently large number,
	$B := \bigcup_{n=1}^N B_n\in \mathscr{A}$,
	\[
	\tau(A\backslash B) \le \tau\left(\bigcup_{n=N+1}^\infty B_n\right)
	\le \sum_{n=N+1}^{\infty} \tau(B_n)\le \frac{\varepsilon}{2}.
	\]
	As $\tau(B \backslash A)\le \tau(\hat{B} \backslash A)< \frac{\varepsilon}{2}$,
	$\tau(A\Delta B) < \varepsilon$.
\end{proof}

\begin{example}
    Consider the Bernoulli test,
	recall $C_{i_1,\dots,i_n}$ we defined earlier.
	A measure(probability) $\mu$ is defined on
	the semi-ring $\{C_{i_1,\dots,i_n}\}\cup\{\emptyset,X\}$,
	then it can extend uniquely to the $\sigma$-algebra.
\end{example}
