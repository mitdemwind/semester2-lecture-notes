%! TeX root = ./main.tex
\begin{definition}
	A function $p(x_1, A_2)$ is called a \vocab{transform function}
	from $X_1$ to $X_2$ if $p(x_1, \cdot)$ is a measure on $\mathscr{F}_2$,
	and $p(\cdot, A_2)$ is measurable in $\mathscr{F}_1$.
\end{definition}
If $X_2 = \sum_{n} A_n$ and $p(x, A_n) < \infty$ for all $n$ and $x$,
then we say $p(\cdot, \cdot)$ is $\sigma$-finite.
Note that this partition is independent of $x$.
If each $p(x, \cdot)$ is a probabilty, we say $p$ is a
\vocab{probabilty transform function}.

Let $X = X_1 \times X_2, \hat{X} = X_2\times X_1,
\mathscr{F} = \mathscr{F}_1 \times \mathscr{F}_2$.

\begin{theorem}
    Let $p(x_1, A_2)$ be a $\sigma$-finite transform function from $X_1$ to $X_2$.
	\begin{itemize}
		\item For all $\sigma$-finite measure  $\mu_1$ on $X_1$, $\exists !$ measure
			$\mu$ s.t.
			\[
			\mu(A_1 \times  A_2) = \int_{A_1} p(x_1, A_2)\mu_1(\dd x_1),
			\]
		\item If $f: X\to \mathbb{R}$'s integral exists, then
			\[
			\int_X f\dd \mu = \int_{X_1}\mu_1(\dd x_1)
			\int_{X_2}f(x_1, x_2)p(x_1, \dd x_2).
			\]
	\end{itemize}
\end{theorem}
\begin{proof}[Proof]
    See proof of Fubini's theorem in analysis.
\end{proof}

Hence given a measure on $X_1$ and a transform function, we can get a
measure on the product space.

If we start from the conditional probabilty, let $g(x) = x_1, f(x) = x_2$,
we have
\[
E(h_2(x_2)| x_1) = \varphi(x_1), \quad \phi(x_1) =\int_{X_2}h_2(x_2)\nu(x_1, \dd x_2).
\]
Multiplying a function of $x_1$, (i.e. $h_1(x_1)$) taking the integral we get
\[
E(h_1(x_1)h_2(x_2)) = \int_{X_1} \mu_1(x_1)
\int_{X_2} h_1(x_1)h_2(x_2)\nu(x_1, \dd x_2).
\]
Thus by typical method we can generalize $h_1(x_1)h_2(x_2)$ to any
function $f(x_1, x_2)$.
Hence the transform function $p$ is nothing but the regular conditional probabilty.

\begin{corollary}[Fubini's theorem]
    If $p(x_1, \cdot) \equiv \mu_2$, denote $\mu$ as $\mu_1 \times \mu_2$,
	if the integral of $f$ exists,
	\[
	\int_X f\dd \mu_1 \times \mu_2 = \int_{X_1}\mu_1(\dd x_1)\int_{X_2}f(x_1, x_2)
	\mu_2(\dd x_2) = \int_{X_2}\mu_2(\dd x_2)\int_{X_1}f(x_1, x_2)\mu_1(\dd x_1).
	\]
\end{corollary}
\begin{remark}
    The integral of $f$ exists means that the integral of $f$ exists in
	the product space, i.e. the LHS must exist.
	It's not true we only have the RHS exists.
\end{remark}

\begin{example}
    Let $X_1 = X_2 = \mathbb{R}$, we use the Lebesgue measure $\lambda$.
	Let $f(x, y) = \ii_{\{0< y\le 2\}} - \ii_{\{-1 < y \le 0\}}$.

	It's easy to see the integral of $f$ doesn't exist,
	but $\iint f(x,y) \dd y\dd x = \infty$, while $\iint f(x,y)\dd x\dd y$ does
	not exist.
\end{example}

By induction we can reach product space of finitely many spaces:
\begin{theorem}
	Let $p_k$ be the transform function from $\prod_{i=1}^{k-1}X_i$ to $X_k$,
	for any $\sigma$-finite measure  $\mu_1$ on $X_1$, $\exists !$ measure $\mu$,
	such that ...TODO
\end{theorem}

\subsection{Countable dimensional product space}
\label{sub:Countable dimensional product space}
Again let $\pi_n$ be the projection onto $X_n$,
and $\pi_{(n)}$ be the projection onto $X_{(n)} := \prod_{i=1}^n X_i$.

Let $\mathscr{F}_{(n)} := \prod_{i=1}^n \mathscr{F}_i = \sigma(\mathscr{Q}_{(n)})$,
and define
\[
	\mathscr{Q}_{[n]} = \left\{\prod_{k=1}^n A_k \times \prod_{k = n+1}^\infty X_k\mid
	A_k \in X_k\right\} = \pi_{(n)}^{-1} \mathscr{Q}_{(n)}.
\]
\begin{proposition}
	$\mathscr{Q} = \bigcup_{n=1}^\infty\mathscr{Q}_{[n]}$ is a semi-ring,
	and $X\in \mathscr{Q}$.
	Similarly, $\mathscr{A} = \bigcup_{n=1}^\infty \mathscr{F}_{[n]}$ is an algebra.
\end{proposition}

\begin{theorem}[Tulcea]
    Let $p_k$ be probabilty transform functions $\prod_{i=1}^{k-1}X_i \to X_k$,
	then for all probabilty $P_1$ on $X_1$, there exists unique
	probabilty $P$ on $\prod_{k=1}^\infty X_k$ s.t.
	\[
	P\left(\prod_{k=1}^n A_k \times \prod_{k=n+1}^\infty X_k\right)
	= \int_{A_1}P_1(\dd x_1)\int_{A_2}p_2(x_1, \dd x_2) \cdots
	\int_{A_n} p_n(x_1, \dots, x_{n-1}, \dd x_n).
	\]
\end{theorem}
\begin{proof}[Proof]
    By results in previous section,
	we can define $P_n$ on $\mathscr{F}_{[n]}$.

	Since $P_{n+1}\big|_{\mathscr{F}_{[n]}} = P_n$, we can get
	a function $P$ on the
	algebra $\mathscr{A} = \bigcup_{n=1}^\infty \mathscr{F}_{[n]}$.
	(By transfinite induction)

	At last we'll prove $P$ is a measure on $\mathscr{A}$, thus it can be uniquely
	extended to $\mathscr{F} = \sigma(\mathscr{A})$.

	\begin{claim}
		$P_n = P_{n+1}\big|_{\mathscr{F}_{[n]}}$.
	\end{claim}
	\begin{proof}[Proof]
	    Some abstract nonsense. Just note that $A_{(n+1)} = A_{(n)} \times X_{n+1}$
		for $A \in \mathscr{F}_{(n)}$,
		and just compute the $(n+1)$-th integral to get the equality.
	\end{proof}

	\begin{claim}
	    $P$ is countablely additive on $\mathscr{A}$.
	\end{claim}
	\begin{proof}[Proof]
	    It's easy to see that $P$ has finite additivity,
		so it suffices to prove $P$ is continuous at empty set.

		Let  $A_1, A_2, \dots \in \mathscr{A}$, $A_n \downarrow \emptyset$,
		if  $P(A_n) \not\to 0$,
		let $\varepsilon := \downarrow \lim_{n\to \infty} P(A_n) > 0$.

		There exist $1\le m_1 < m_2 < \cdots$ s.t. $A_n \in \mathscr{F}_{[m_n]}$.
		WLOG $m_n = n$ (otherwise add more sets in the sequence, i.e.
		$B_k = A_n$ when $m_n \le k < m_{n+1}$).

		Therefore we have $A_{(n)} = \pi_{(n)}^{-1} A_{(n)}$,
		\[
		A_n = \pi_{(n+1)}^{-1}(A_{(n)} \times X_{n+1}) \supseteq A_{n+1}
		\implies A_{(n)} \times X_{n+1} \supseteq A_{(n+1)}.
		\]
		Equivalently,
		\[
		\ii_{A_{(n+1)}}(x_1,\dots, x_{n+1}) \le \ii_{A_{(n)}}(x_1,\dots, x_n).
		\]
		Therefore, we have $0\le \phi_{1,n+1}(x_1)\le \phi_{1, n}(x_1)\le 1$,
		where
		\[
		\phi_{1, n}(x_1) := \int_{X_2} p_2(x_1, \dd x_2)\cdots
		\int_{X_n}\ii_{A_{(n)}}(x_1,\dots,x_n) p_n(x_1,\dots,x_{n-1},\dd x_n).
		\]
		Note that $P(A_{[n]}) = P_n(A_{[n]}) = \int_{X_1}\phi_{1, n}P_1(\dd x_1)$.

		Let $\phi_1 := \downarrow \lim_{n\to \infty}\phi_{1, n}$, by dominated
		convergence theorem,
		\[
		\int_{X_1}\phi_1\dd P_1 = \downarrow \lim_{n\to \infty}\int_{X_1}\phi_{1, n}
		\dd P_1 = \varepsilon > 0.
		\]
		Hence $\exists \tilde x_1\in X_1$ s.t. $\phi_1(\tilde x_1) > 0$.
		We must have $\tilde x_1\in A_{(1)}$, otherwise
		\[
		\ii_{A_{(n)}}(\tilde x_1, x_2,\dots, x_n) \le \ii_{A_{(1)}}(\tilde x_1) = 0,
		\]
		which gives $\phi_{1, n}(\tilde x_1) = 0$, $\forall n$, contradiction!

		By the same process we can take $\phi_2(x_2) = \downarrow \lim_{n\to \infty}
		\phi_{2, n}(x_2)$, where $\phi_{2, n}(x_2)$ is defined as
		\[
		\int_{X_3}p_3(\tilde x_1, x_2, \dd x_3)\cdots \int_{X_n}\ii_{A_{(n)}}
		(\tilde x_1, x_2, \dots, x_n) p_n(\tilde x_1, x_2,\dots, x_{n-1}, \dd x_n).
		\]
		We'll get $\tilde x_2$ s.t. $(\tilde x_1, \tilde x_2)\in A_{(2)}$,
		and $\phi_2(\tilde x_2)>0$.

		By induction we get $(\tilde x_1, \tilde x_2, \dots)\in
		\bigcap_{n=1}^\infty A_{[n]}$,
		which contradicts with $A_n\downarrow \emptyset$!
	\end{proof}
	Hence the conclusion holds.
\end{proof}
