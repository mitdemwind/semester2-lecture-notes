%! TeX root = ./main.tex
\subsection{Properties of integrals}
\label{sub:Properties of integrals}
\begin{theorem}[Linearity of integrals]
    Let $f,g$ be functions whose integral exists.
	\begin{itemize}
		\item $\forall a\in \mathbb{R}$, the integral of $af$ exists,
			and $\int_X (af)\dd \mu = a\int_X f\dd \mu$;
		\item If $\int_X f\dd \mu + \int_X g\dd \mu$ exists,
			then $f + g\ a.e.$ exists, its integral exists and
			\[
			\int_X(f+g)\dd \mu = \int_X f\dd \mu + \int_X g\dd \mu.
			\]
	\end{itemize}
\end{theorem}
\begin{proof}[Proof]
    The first one is trivial by definition.

	As for the second,
	\begin{enumerate}
		\item First we prove $f+g \ a.e.$ exists.
			If $|f|<\infty, a.e.$, we're done.

			If $\mu(f = \infty) > 0$, then $\int_X f\dd \mu = \infty$.
			This means $\int_X g\dd \mu \ne -\infty$, so $\mu(g = - \infty) = 0$.
			Thus $f + g\ a.e.$ exists.
			Similarly we can deal with the case $\mu(f = -\infty) > 0$.
		\item Next we prove the equality.
			$f + g = (f^+ + g^+) - (f^- + g^-)$.
			Let $\varphi = f^+ + g^+, \psi = f^- + g^-$.
			Our goal is
			\[
			\int_X (\varphi - \psi)\dd \mu = \int_X \varphi\dd \mu - \int_X \psi\dd \mu.
			\]
			Since $f+g \ a.e.$ exists, so $\varphi - \psi$ exists almost everywhere.
			If $\int_X \varphi \dd\mu = \int_X \psi\dd \mu = \infty$,
			then the integral of $f,g$ must be $+\infty$ and $-\infty$,
			which contradicts with our condition.
			So both sides of above equation exist.

			Since $\max\{\varphi, \psi\} = \psi + (\varphi - \psi)^+
			= \varphi + (\varphi - \psi)^-$, by the linearity of non-negative integrals,
			\[
			\int_X \psi\dd \mu + \int_X (\varphi - \psi)^+\dd \mu
			= \int_X \varphi\dd \mu + \int_X (\varphi - \psi)^-\dd \mu.
			\]
			which rearranges to the desired equality.
	\end{enumerate}
	Note: we need to verify that we didn't add infinity to the equation in the last step.
\end{proof}

\begin{proposition}
	Let $f,g$ be integrable functions,
	If $\int_A f\dd \mu \ge \int_A g\dd \mu, \forall A\in \mathscr{F}$,
	then $f\ge g, a.e.$.
\end{proposition}
\begin{proof}[Proof]
    Let $B = \{ f < g\}$, then $(g - f)\ii_B \ge 0$,
	\[
	\int_B (g - f)\dd \mu = \int_B (g - f)\ii_B \dd \mu \ge 0.
	\]
	By the linearity of integrals we get $(g - f)\ii_B = 0, a.e.$,
	i.e. $\mu(B) = 0$.
\end{proof}

\begin{proposition}
	If $\mu$ is $\sigma$-finite, the integral of $f,g$ exists,
	the conclusion of previous proposition also holds.
\end{proposition}
\begin{proof}[Proof]
    Let $X = \sum_{n} X_n$, $\mu(X_n) < \infty$. By looking at $X_n$,
	we may assume $\mu(X) < \infty$.

	Since $\{ f < g \} = \{-\infty \ne f < g\} + \{f = -\infty < g\}$.

	Let $B_{M, n} = \{|f|\le M, f + \frac{1}{n} < g\}$.
	By condition,
	\[
	\int_{B_{M,n}} f\dd \mu \ge \int_{B_{M,n}} g\dd \mu \ge
	\int_{B_{M,n}} f\dd \mu + \frac{1}{n}\mu(B_{M,n}).
	\]
	Since $\int_{B_{M,n}} f\dd \mu \le M\mu(X)$ is finite, we get $\mu(B_{M,n}) = 0$.
	This implies $\{-\infty \ne f < g\} = \bigcup B_{M,n}$ is null.

	Let $C_M = \{g > -M\}$, similarly,
	\[
	-\infty\cdot\mu(C_M) = \int_{C_M}f \dd \mu \ge\int_{C_M} g\dd \mu = -M \mu(C_M).
	\]
	Hence $\mu(C_M) = 0$, $\{-\infty = f < g\} = \bigcup C_M$ is null.
\end{proof}

\begin{remark}
	When $\ge$ is replaced by $=$, the conclusion holds as well.
    This proposition tells us that the integrals of $f$ totally determines $f$.
	(In calculus, taking the derivative of integrals gives original functions)
\end{remark}

\begin{theorem}[Absolute continuity of integrals]
    Let $f$ be an integrable function, $\forall \varepsilon>0$, $\exists \delta>0$,
	such that $\forall A\in \mathscr{F}$,
	\[
	\mu(A) < \delta \implies \int_A |f|\dd \mu < \varepsilon.
	\]
\end{theorem}
\begin{proof}[Proof]
Take non-negative simple functions $g_n \uparrow |f|$.
Since $\int |f|\dd \mu < \infty$, $\exists N$ s.t.
\[
\int_X (|f| - g_N)\dd \mu = \int_X |f|\dd \mu - \int_X g_N \dd\mu < \frac{\varepsilon}{2}.
\]

Let $M = \max_{x\in X} g_N(x)$, $\delta = \frac{\varepsilon}{2M}$, so
\[
\int_A |f|\dd \mu < \frac{\varepsilon}{2} + \int_A g_N \dd \mu = \frac{\varepsilon}{2}
+ M\mu(A) < \varepsilon.
\]
\end{proof}

\begin{example}
    Fundamental theorem of Calculus, Lebesgue version:
	Let $g$ be a measurable function, then $g$ is absolutely continuous
	iff $\exists f: [a,b]\to \mathbb{R}$ Lebesgue integrable, s.t.
	\[
	g(x) - g(a) = \int_a^x f(z)\dd z.
	\]
	The absolute continuity can be implied by the absolute continuity of
	integrals.
\end{example}

\subsection{Convergence theorems}
\label{sub:Convergence theorems}

Levi, Fatou, Lebesgue.

In this section we mainly discuss the commutativity of
integrals and limits, i.e. if $f_n \to f$, we care when does the following holds:
\[
\lim_{n\to \infty} \int_X f_n \dd \mu = \int_X f\dd \mu.
\]

\begin{theorem}[Monotone convergence theorem, Levi's theorem]
    Let $f_n \uparrow f, a.e.$ be non-negative functions, then
	\[
	\int_X f_n \dd \mu \uparrow \int_X f\dd \mu.
	\]
\end{theorem}
\begin{proof}[Proof]
    By removing countable null sets, we may assume $0 \le f_n(x)\uparrow f$.

	Take non-negative simple functions $f_{n,k}\uparrow f_n$.
	Let $g_k = \max_{1\le n\le k}f_{n, k}$ be simple functions.
	\[
	g_k = \max_{1\le n\le k}f_{n, k}\le \max_{1\le n\le k+1} f_{n, k+1} = g_{k+1}.
	\]
	So $g_k \uparrow$, say $g_k \to g$ for some function $g$.
	Clearly $g\le f$ as $g_k \le f_k$, $\forall k$.

	Note as $k\to \infty$, $g_k \ge f_{n,k}\implies g\ge f_n, \forall n$.
	so $g = f$.

	By definition of integrals,
	\[
	\int_X f\dd \mu = \lim_{k\to \infty} \int_X g_n\dd \mu,
	\]
	and
	\[
	\int_X g_n \dd \mu \le \int_X f_n\dd \mu \le \int_X f\dd \mu.
	\]
	So the conclusion follows.
\end{proof}
\begin{corollary}
    Let $f_n$ be functions whose integrals exist, if
	\[
	f_n \uparrow f, a.e. \quad \int_X f_1^- \dd \mu < \infty,\quad
	\text{or}\quad
	f_n \downarrow f, a.e. \quad \int_X f_1^+ \dd \mu < \infty,
	\]
	then the integral of $f$ exists, and $\int_X f_n \dd \mu \to \int_X f\dd \mu$.
\end{corollary}
\begin{remark}
    Counter example when $\int_X f_1^+ \dd \mu = \infty$: let $X = \mathbb{R}$,
	\[
		f_n = \ii_{[n, \infty)} \downarrow f = 0,\quad
		\int_X f_n \dd \mu = \infty, \quad \int_X f \dd \mu = 0.
	\]
\end{remark}

\begin{corollary}
    If the integral of $f$ exists, then for any measure partition $\{A_n\}$,
	\[
	\int_X f\dd \mu = \sum_{n=1}^{\infty} \int_{A_n}f\dd \mu.
	\]
\end{corollary}

If $f\ge 0$, then $\nu: A\mapsto \int_A f\dd \mu$ is a measure on $\mathscr{F}$.
If we don't require $f\ge 0$, $\nu$ will become a signed measure
which we'll cover later.
