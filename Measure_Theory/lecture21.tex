%! TeX root = ./main.tex
\begin{theorem}[Kolmogorov]
    Let $P_k$ be a probability on $(X_k, \mathscr{F}_k)$,
	then there exists a unique measure $P$ on $(\prod X_k, \prod \mathscr{F}_k)$,
	such that
	\[
		P\left(\prod_{k=1}^n A_k \times \prod_{k=n+1}^\infty X_k\right)
		= \prod_{k=1}^nP_k(A_k).
	\]
\end{theorem}
\begin{proof}[Proof]
    This is immediate by Tulcea's theorem.
\end{proof}

Let's make a summary of Tulcea's theorem.
To get a measure on $\mathscr{F}$, we need:
\begin{itemize}
	\item Measures $P_n$ on $ \mathscr{F}_{[n]}$,
		which is induced by measures on $ \mathscr{F}_{(n)}$.
	\item Compatibility, i.e. $P_{n+1}|_{\mathscr{F}_{[n]}} = P_n$.
		Hence we'll get a function $P$ on the algebra $\bigcup \mathscr{F}_{[n]}$.
	\item At last to prove $P$ is a measure, we need the continuity at $\emptyset$.
\end{itemize}
Tulcea's theorem tells us that the measure induced by the probability transform
functions statisfies above conditions.

\subsection{Arbitary infinite dimensional product space}
\label{sub:Arbitary infinite dimensional product space}
Let $\{X_t, t\in T\}$ be a collection of sets, where $T$ is uncountable.
Let $X = \prod_{t\in T}X_t$ be the product space.

Let $U \subset S \subset T$, where $|S|<\infty$, define the projection
\[
\pi_S: X\to X_S := \prod_{t\in S}X_t,\quad
\pi_{S\to U}: X_S \to X_U, \quad \pi_{S\to U}\circ \pi_S = \pi_U.
\]
Similarly, we can define the cylinder set:
\[
\mathscr{Q}_S = \left\{\pi_S^{-1}\left(\prod_{t\in S}A_t\right): A_t\in \mathscr{F}_t,
\forall t\in S\right\}; \quad \mathscr{F}_S = \sigma(\mathscr{Q}_S).
\]
\begin{proposition}
	We have $\mathscr{Q}_S$, $\mathscr{Q}:= \bigcup_{|S|<\infty} \mathscr{Q}_S$ are
	semi-rings containing $X$.
\end{proposition}
\begin{proposition}
	$\mathscr{A}:= \bigcup_{|S|< \infty} \mathscr{F}_S$ is an algebra
	containing $\mathscr{Q}$.
\end{proposition}
\begin{proposition}
	Let $\mathscr{F} := \sigma(\mathscr{Q}) = \sigma(\mathscr{A})$, we have
	\[
	\mathscr{F} = \sigma(\{\pi_t, t\in T\}) = \{\pi_S^{-1}A: A\in \mathscr{F}_S,
	|S| \le \omega\}.
	\]
\end{proposition}
\begin{remark}
    To prove the equality, first note $LHS =
	\sigma(\bigcup_{t\in T}\pi_t^{-1}\mathscr{F}_t)$,
	and $RHS$ is a $\sigma$-algebra.
\end{remark}

In random process, $(\Omega, \mathscr{S})$ is the sample space,
the index set $T$ is regarded as time,
for each time $t\in T$, there's a random variable $f_t: \Omega \to X_t$.
Thus $f := \{f_t, t\in T\}$ is a map $\Omega\to \prod_{t\in T}X_t$.

\begin{theorem}
    Let $ \mathscr{F} = \prod_{t\in T} \mathscr{F}_t$,
	\[
	f:(\Omega, \mathscr{S})\to (X, \mathscr{F}) \iff
	f_t:(\Omega, \mathscr{S}) \to (X_t, \mathscr{F}_t), \forall t.
	\]
\end{theorem}

If $(X_t, \mathscr{F}_t)\equiv (S, \mathscr{S}_0)$, then we say $f$ is a random
process; $S$ is said to be the range space,
and $f(\omega) = \{f_t(\omega): t\in T\}\in S^T$ is an orbit.

For any probability $Q$ on $(\Omega, \mathscr{S})$, $Q\circ f^{-1}$ is
the distribution of $f$, by previous proposition,
we only need all the countably dimensional joint distribution of $f$.

From Tulcea's theorem, we only need to study
\textit{finite dimensional joint distribution} $P_{t_1, \dots, t_n}$ where
$t_1, \dots, t_n \in T$.

Similarly we require the probability to have some compatibility:
\begin{itemize}
	\item Let $t(1), \dots, t(n)$ be a permutation of $ t_1, \dots, t_n$.
		We require
		\[
			P_{t_1, \dots, t_n}\left(\prod_{i=1}^n A_{t_i}\right)
			= P_{t(1),\dots, t(n)}\left(\prod_{i=1}^n A_{t(i)}\right).
		\]
	\item Let $t_{n+1} \in T$,
		\[
		P_{t_1, \dots, t_{n+1}}\left(\prod_{i=1}^n A_{t_i} \times X_{t_{n+1}}\right)
		= P_{t_1, \dots, t_n}\left(\prod_{i=1}^n A_{t_i}\right).
		\]
\end{itemize}

\begin{theorem}[Kolmogorov]
    If $\mathbf{P}$ is compatible, then $\exists! P $ on
	$(\mathbb{R}^T, \mathscr{B}^T)$ s.t.
	\[
	P(\pi_S^{-1} A) = P_S(A),\quad \forall |S| < \infty, A\in \mathscr{B}^S.
	\]
\end{theorem}
\begin{proof}[Sketch of the proof]
    Let $\mathscr{F}_0 = \{\pi_{T_0}^{-1}(A): A\in \mathscr{F}_{T_0},
	|T_0|\le \omega\}$.

	Step 1, fix a countable $T_0 \subset T$,
	by Tulcea's theorem, we can define $P(\pi_{T_0}^{-1}A) = P_{T_0}(A)$.

	Step 2, $P$ is well-defined in different permutations of $T_0$.

	Step 3, if $T_1, T_2$ countable, and $\pi_{T_1}^{-1}(A_1) = \pi_{T_2}^{-1}(A_2)$,
	we have $P_{T_1}(A_1) = P_{T_2}(A_2)$.
	This can be done by looking at $T_0 = T_1\cup T_2$.

	Step 4, check $P$ statisfies countable additivity.
\end{proof}

\begin{example}[Brownian motion]
    Let $\mathbf{B} = \{B_t, t\in T\}$, $T = \mathbb{R}_+$.
	Let $(\Omega, \mathscr{S}, \hat{P})$ be the sample space,
	$(\mathbb{R}^T, \mathscr{B}^T)$ be the orbit space,
	where $\varphi: T\to \mathbb{R}$ is an orbit.
	\[
	\mathbf{B}(\omega) := \varphi: t\mapsto \varphi(t) = B_t(\omega).
	\]
	Initially, let $B_0 = 0$, define the transformation density
	\[
	p_t(x, y) = \frac{1}{\sqrt{2\pi t}}\exp\left(- \frac{(y - x)^2}{2t}\right).
	\]
	Starting from finite dimensional orbit distribution,
	we can get countable dimensional orbit distribution.

	TODO
\end{example}
