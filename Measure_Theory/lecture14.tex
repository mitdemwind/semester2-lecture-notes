%! TeX root = ./main.tex
\section{Signed measure}
\label{sec:Signed measure}
\subsection{Definitions}
\label{sub:Definitions}

Let $(X, \mathscr{F}, \mu)$ be a measure space, consider
\[
\varphi(A) := \int_A f\dd\mu, \quad \forall A\in \mathscr{F}.
\]
If the integral of $f$ exists, then $\varphi$ has countable additivity.
Also note $\varphi(\emptyset) = 0$, so $\varphi$ looks like a measure,
except it can take negative values.

In fact, denote  $X^+ = \{f\ge 0\}, X^- = \{f < 0\}$, then
$\varphi(A) = \varphi(AX^+) + \varphi(AX^-)$.

\begin{definition}[Signed measure]
	If a set function $\varphi : \mathscr{F} \to \overline{\mathbb{R}}$ which
	satisfies countable additivity and $\varphi(\emptyset) = 0$,
	then we call $\varphi$ a \vocab{signed measure}.

	If $|\varphi(A)| < \infty, \forall A\in \mathscr{F}$, then $\varphi$ is
	\vocab{finite}; Similarly we define \vocab{$\sigma$-finite}.
\end{definition}

Since $\int_A f\dd \mu$ can't reach both $\pm \infty$(otherwise the integral
doesn't exist), so
\begin{proposition}
	Let $\varphi$ be a signed measure, then:
	\[
	\varphi(A) < \infty, \quad \forall A\in \mathscr{F},
	\quad or \quad \varphi(A) > -\infty, \quad \forall A\in \mathscr{F}.
	\]
\end{proposition}
\begin{proof}[Proof]
    Assume that $\varphi(A) = \infty, \varphi(B) = -\infty$,
	then:
	\[
	\varphi(A\cup B) = \varphi(A) + \varphi(A\backslash B) = +\infty,
	\]
	and similarly $\varphi(A\cup B) = -\infty$, contradiction!
\end{proof}
\begin{remark}
    From now on we may assmue $\varphi(A) > -\infty$.
\end{remark}

\begin{proposition}
	If $A\supseteq B$, and $|\varphi(A)| < \infty$, then $|\varphi(B)| < \infty$.
\end{proposition}
\begin{proof}[Proof]
    Trivial, same as above proposition.
\end{proof}

\begin{proposition}
	Let $A_1, A_2,\dots$ be pairwise disjoint sets,
	and $|\varphi(\sum_{n=1}^\infty A_n)| < \infty$, then
	\[
	\sum_{n=1}^{\infty} |\varphi(A_n)| < \infty.
	\]
\end{proposition}
\begin{proof}[Proof]
    Let $I = \{n: \varphi(A_n) > 0\}$, $J = \{n: \varphi(A_n) < 0\}$,
	\[
	B = \sum_{n\in I}A_n,\quad C = \sum_{n\in J} A_n,
	\]
	since $B,C \subset \sum_{n=1}^\infty A_n$,
	thus $\varphi(B), \varphi(C) \in \mathbb{R}$.

	Note that $\sum_{n\in I} |\varphi(A_n)| = |\varphi(B)|$,
	$\sum_{n\in J}\varphi(A_n) = |\varphi(C)|$, and we're done.
\end{proof}

\subsection{Hahn decomposition and Jordan decomposition}
\label{sub:Hahn decomposition and Jordan decomposition}
Let's look at the indefinite integral again, notice that
\[
\varphi(A) = \int_{A\cap \{f>0\}} f\dd\mu + \int_{A\cap \{f<0\}}f\dd\mu
= \int_A f^+ \dd\mu - \int_A f^-\dd \mu.
\]
It turns out that this property holds for any signed measure.

\begin{definition}[Hahn decomposition]
	If a patition $\{X^+, X^-\}$ of $X$ satisfies:
	\[
	\varphi(A) \ge 0, \forall A \subset X^+, \quad
	\varphi(A) \le 0, \forall A \subset X^-,
	\]
	then $\{X^+, X^-\}$ is called a \vocab{Hahn decomposition} of $\varphi$.
\end{definition}

\begin{definition}[Jordan decomposition]
	Let $\varphi^{\pm} = \int_A f^{\pm} \dd\mu$ be measures, if
	\[
	\varphi = \varphi^+ - \varphi^-,
	\]
	then it's called a \vocab{Jordan decomposition} of $\varphi$.
\end{definition}

We're going to find $X^+$, or equivalently, find $\varphi^+$.
Let $\varphi^*(A) := \sup\{\varphi(B): B \subseteq A\}$.

It's clear that $\varphi^*$ is non-negative, monotone, and $\varphi^*(\emptyset) = 0$.

Consider $\mathscr{F}^- = \{A: \varphi^*(A) = 0\}$. Intuitively, this
is all the subsets of $X^-$, unioned with ``null sets'' in $X^+$.

\begin{theorem}[Hahn decomposition]
	\label{Hahn decomposition}
    Let $X^-$ be a set with maximum $|\varphi|$ in $\mathscr{F}^-$,
	(since $\varphi > -\infty$, $X^-$ must exist)
	and $X^+ = X \backslash X^-$ doesn't contain any set $A$ with $\varphi(A) < 0$.

	Furthermore, the Hahn decomposition is unique:
	\[
	\varphi(A) = 0, \quad \forall A\in X_1^+\Delta X_2^+
	= X_1^- \Delta X_2^-.
	\]
\end{theorem}
The critical part of this theorem is:
\begin{lemma}
	\label{lem:Hahn1}
	If $\varphi(A) < 0$, then we can find $A_0 \subset A$ s.t. $\varphi^*(A_0) = 0$,
	$\varphi(A_0) < 0$.
\end{lemma}

To prove this lemma, we need another lemma:
\begin{lemma}
	If $\varphi(A) < \infty$, then $\forall \varepsilon>0$,
	$\exists A_\varepsilon \subset A$ s.t.
	\[
		\varphi(A_\varepsilon) \ge 0,\quad
		\varphi^*(A \backslash A_\varepsilon) \le \varepsilon.
	\]
\end{lemma}
\begin{proof}[Proof]
    Assume by contradiction that $\exists \varepsilon_0\ge 0$ s.t.
	$\forall A_0 \subset A$, $\varphi(A_0)<0$ or
	$\varphi^*(A \backslash A_0) > \varepsilon_0$,
	this means,
	\[
		\varphi(A_0)\ge 0\implies \varphi^*(A \backslash A_0) > \varepsilon_0.
	\]
	This will clearly yield a contradiction:

	Take any $\varphi(A_0) \ge 0$(say $A_0=\emptyset$),
	then exists $A_1 \subset A\backslash A_0$ s.t.
	$\varphi(A_1) > \varepsilon_0$, and $\varphi(A_0\cup A_1) \ge 0$,
	continuing this process we can get infinitely many pairwise disjoint
	sets $A_1, A_2,\dots$, with $\varphi(A_n) > \varepsilon_0$,
	so $\varphi(\sum_{i=1}^\infty A_n) = \infty\implies \varphi(A) = \infty$,
	contradiction!
\end{proof}

\begin{proof}[Proof of \autoref{lem:Hahn1}]
    Applying above lemma repeatedly and take a limit:

	Take $C_1 \subset A$ s.t. $\varphi(C_1) \ge 0$ and
	$\varphi^*(A \backslash C_1)\le 1$.
	Let $A_1 = A \backslash C_1$, $\varphi(A_1) < 0$.

	Again take
	\[
		C_{k+1} \subset A_k, A_{k+1} = A_k \backslash C_{k+1}
		\implies \varphi^*(A_{k+1}) \le \frac{1}{k+1}, \varphi(A_{k+1}) < 0.
	\]
	Since $A_k \downarrow$, let $A_0 = \lim_{k\to \infty} A_k$,
	note $\varphi^*(A_k) \downarrow 0$, we must have $\varphi^*(A_0) = 0$.

	Also $\varphi(\sum C_k) = \sum \varphi(C_k) \ge 0$, so $\varphi(A_0) < 0$.
\end{proof}

\begin{proof}[Proof of \autoref{Hahn decomposition}]
    First we prove that $\mathscr{F}^-$ is a $\sigma$-ring:
	$\emptyset\in \mathscr{F}^-$, if $A_1, A_2\in \mathscr{F}^-$,
	\[
	0 \le \varphi^*(A_1 \backslash A_2)\le \varphi(A_1) = 0.
	\]
	Thus $A_1 \backslash A_2 \in \mathscr{F}^-$.

	If $A_1, A_2,\dots \in \mathscr{F}^-$ pairwise disjoint,
	\[
	\varphi(B) = \sum_{n=1}^{\infty} \varphi(B\cap A_n) \le 0, \quad
	\forall B \subset \sum_{n=1}^{\infty} A_n.
	\]
	Hence $\sum_{n=1}^\infty A_n \in \mathscr{F}^-$.

	Next we'll prove Hahn decomposition exists:

	Let $\alpha := \inf \{\varphi(A) : A\in \mathscr{F}^-\}$, $\alpha \le 0$.

	Let $\{A_n\}\in \mathscr{F}^-$ s.t. $\varphi(A_n) \to \alpha$,
	then $X^- := \bigcup_{n=1}^\infty A_n \in \mathscr{F}^-$.
	\[
	\varphi(X^-) = \varphi(A_n) + \varphi(X^- \backslash A_n)
	\le \varphi(A_n) + \varphi^*(X^- \backslash A_n) = \varphi(A_n) \to \alpha.
	\]
	Therefore $-\infty < \varphi(X^-) = \alpha$.

	Hence $\forall A, \varphi(AX^-) \le \varphi^*(X^-) = 0$.
	By \autoref{lem:Hahn1} we get $\forall A, \varphi(AX^+) \ge 0$,
	otherwise $\exists A_0 \subset A$ s.t. $\varphi^*(A_0) = 0, \varphi(A_0) < 0$.
	Then $\varphi(X^- \cup A_0) = \alpha + \varphi(A_0) < \alpha$, contradiction!

	At last we'll prove the uniqueness:

	If $X_1^{\pm}, X_2^{\pm}$ are both Hahn decompositions,
	then $A \in X_1^+\cap X_2^- + X_1^- \cap X_2^+$, it's clear $\varphi(A) = 0$.
\end{proof}

\begin{theorem}[Jordan decomposition]
    The Jordan decomposition exists and is unique:
	\[
	\varphi = \varphi^+ - \varphi^-, \quad
	\varphi^+ = \varphi^*, \varphi^- = (-\varphi)^*.
	\]
\end{theorem}
\begin{proof}[Proof]
    Let $\varphi^{\pm}$ be measures with $\varphi^{\pm} = \pm\varphi(A\cap X^{\pm})$.
	It's clear that this is a Jordan decomposition.

	Now given any Jordan decomposition $\varphi^{\pm}$.

	Since
	\[
		\forall B \subset A, \varphi(B)\le \varphi^+(B) \le \varphi^+(A),
	\]
	so $\varphi^* \le \varphi^+$.
	But $A \cap X^+ \subset A$, so $\varphi^* \ge \varphi^+$,
	which proves the result.

	Similarly $\varphi^- = (-\varphi)^*$, so it is unique.
\end{proof}
\begin{remark}
    The support of $\varphi^{\pm}$ are disjoint, but if $\phi \ne 0$,
	then the support of $\varphi^{\pm} + \phi$ intersects.
	$\varphi^{\pm}$ are called the \vocab{upper variation}
	and \vocab{lower variation},
	respectively, and $|\varphi| = \varphi^+ + \varphi^-$ is called the
	\vocab{total variation}.
\end{remark}
