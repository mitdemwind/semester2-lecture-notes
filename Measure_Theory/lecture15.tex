%! TeX root = ./main.tex
\begin{lemma}
	$|\varphi|(A) = 0\iff \varphi(B) = 0, \forall B \subset A$.
\end{lemma}
\begin{proof}[Proof]
    Just write $|\varphi| = \varphi^+ + \varphi^-$,
	we know $\varphi(B) = 0$.

	Conversely, $\varphi(X^\pm \cap A) = 0\implies |\varphi|(A) = 0$.
\end{proof}

\subsection{Radon-Nikodym theorem}
\label{sub:Radon-Nikodym theorem}
We assume the functions and sets below are all measurable.
Let $(X, \mathscr{F})$ be a measurable space, $\varphi$ a signed measure.

\begin{definition}[R-N derivative]
	If there exists a a.e. unique function $f$ s.t.
	\[
	\varphi(A) = \int_A f\dd \mu, \quad \forall A\in \mathscr{F},
	\]
	we say $f$ is the \vocab{Radon-Nikodym derivative} of $\varphi$
	with respect to $\mu$, abbreviated by R-N derivative or derivative,
	denoted by $\frac{\dd\varphi}{\dd \mu}$.
\end{definition}
\begin{remark}
    When $\mu$ is $\sigma$-finite, then $f$ must be unique a.e..
\end{remark}

\begin{definition}[Absolute continuity]
	If $\forall A\in \mathscr{F}$,
	\[
	\mu(A) = 0 \implies \varphi(A) = 0,
	\]
	then we say $\varphi$ is \vocab{absolutely continuous} with
	respect to $\mu$, denoted by $\varphi\ll \mu$.
\end{definition}
Observe that
\[
\mu(A) = 0\implies \mu(A\cap X^\pm) = 0\implies \varphi^\pm(A) = 0,
\]
so $\varphi\ll\mu \iff \varphi^\pm\ll \mu \iff |\varphi|\ll\mu$.

It's obvious that $\frac{\dd\varphi}{\dd\mu}$ exists only if $\varphi\ll\mu$,
but it turns out that this is also the sufficient condition
when $\mu$ is a $\sigma$-finite measure.

We can't prove this directly, so we'll prove some easy cases first.
\begin{lemma}
	Let $\varphi, \mu$ be finite measures. Then
	\[
	\exists f\in \mathscr{L}:= \left\{g\in L_1: g\ge 0,
	\int_A g\dd\mu\le \varphi(A), \forall A\right\},
	\]
	such that $\int_X f\dd\mu = \sup \int_X g\dd \mu$.
\end{lemma}
\begin{proof}[Proof]
    This is somehow similar to find simple functions
	approaching non-negative measurable functions.

	First let $\beta = \sup\int_X g\dd \mu$, and choose $g_k$ s.t.
	$\int_X g_k\dd \mu \to \beta$.

	Let $f_n := \max_{k\le n}g_k$, and $f_n \uparrow f$.
	By Levi's theorem, $\int_A f\dd\mu = \lim_{n\to \infty}f_n\dd \mu$,
	so if $f_n\in \mathscr{L}$, $f\in \mathscr{L}$ as well.
	Let $A_k = A\cap \{f_n = g_k, f_n\ne g_j, j< k\}$ be a partition of $A$,
	\[
	\int_A f_n\dd\mu = \sum_{k=1}^{n} \int_{A_k} g_k\dd\mu
	\le \sum_{k=1}^{n} \varphi(A_k) = \varphi(A).
	\]
	Thus $f_n\in \mathscr{L}$, we have $\int_X f \dd\mu = \beta\ge \int_Xg\dd\mu$,
	for all $g\in \mathscr{L}$.
\end{proof}
\begin{proposition}
	Suppose $\varphi, \mu$ are both finite, then $\varphi\ll \mu\implies
	\frac{\dd\varphi}{\dd \mu}$ exists.
\end{proposition}
\begin{proof}[Proof]
	Decompose $\varphi$ to $\varphi^+ - \varphi^-$, we may assume $\varphi\ge 0$.

    Starting from previous lemma, we'll prove that $\int_A f\dd\mu = \varphi(A)$.
	Let $\nu(A) = \varphi(A) - \int_A f\dd \mu$ be a measure.

	Let $\nu_n$ be increasing signed measures.
	\[
		\nu_n(A):= \nu(A) - \frac{1}{n}\mu(A), \quad \forall A\in \mathscr{F}.
	\]
	Let $X_n^\pm$ be the Hahn decomposition of $\nu_n$, and
	\[
	X^+ = \bigcup_{n=1}^\infty X_n^+,\quad
	X^- = \bigcap_{n=1}^\infty X_n^-.
	\]
	First since $X^- \subset X_n^-$,
	\[
		\nu(X^-) = \nu_n(X^-) + \frac{1}{n}\mu(X^-)\le \frac{1}{n}\mu(X^-)\to 0.
	\]
	We have $f + \frac{1}{n}\ii_{X_n^+}\in \mathscr{L}$ since
	\begin{align*}
		\int_A\left(f+\frac{1}{n}\ii_{X_n^+}\right)\dd\mu
		&= \varphi(A) - \nu(A) + \frac{1}{n}\mu(X_n^+ \cap A)\\
		&\le \varphi(A) - \nu(X_n^+\cap A) + \frac{1}{n}\mu(X_n^+\cap A)\\
		&=\varphi(A) - \nu_n(X_n^+\cap A) \le \varphi(A).
	\end{align*}
	So we have $\int_X f\dd\mu \ge \int_X (f+\frac{1}{n}\ii_{X_n^+})\dd\mu$,
	$\mu(X_n^+) = 0 \implies \mu(X^+) = 0$.

	Since $\varphi\ll \mu$,  $\varphi(X^+) = 0\implies \nu(X^+) = 0$.
\end{proof}

\begin{proposition}
	Let $\varphi$ be a $\sigma$-fintie signed measure,  $\mu$ be a finite measure,
	if $\varphi\ll \mu$, then $\frac{\dd\varphi}{\dd\mu}$ exists and
	its integral exists.
\end{proposition}
\begin{proof}[Proof]
    Let $X = \sum_{n=1}^{\infty} A_n$, $|\varphi(A_n)| <\infty$,
	then the R-N derivative $f_n$ exists on $A_n$,

	Let $f = \sum_{n=1}^{\infty} f_n\ii_{A_n}$, then $f$ finite a.e.,
	\[
	\varphi(A\cap A_n) = \int_{A\cap A_n}f_n \dd\mu = \int_{A\cap A_n}f\dd \mu.
	\]
	WLOG $\varphi^-$ finite, then
	\[
	\varphi(\{f<0\}\cap A_n) = \int_{A_n}f^-\dd\mu = \int_{A_n}f_n^-\dd\mu
	\ge -\varphi^-(A_n)
	\]
	So the integral of $f$ exists.

	Since $\varphi$ is countably additive and the integral of $f$ exists,
	we can add the above equality to get the desired.
\end{proof}

\begin{proposition}
	Let $\varphi$ be a signed measure, the above conclusion also holds.
\end{proposition}
\begin{proof}[Proof]
    Let
	\[
	\mathscr{G}:=\left\{\sum_{n=1}^{\infty} A_n :
	|\varphi(A_n)|< \infty, n=1,2,\dots\right\}.
	\]
	Since $\emptyset\in \mathscr{G}$, and it's closed under set difference:
	\[
	\sum_{n=1}^{\infty} A_n \backslash \sum_{n=1}^{\infty} B_n
	= \sum_{n=1}^{\infty} (A_n \backslash B)
	\]
	by $A_n \backslash B \subset A_n$,
	we have $|\varphi(A_n \backslash B)| < \infty$.

	$\mathscr{G}$ is a $\sigma$-ring.

	There exists  $B$ s.t. $\mu(B) = \gamma:= \sup\{\mu(A): A\in \mathscr{G}\}$.
	Just take $\mu(B_n) \to \gamma, B = \bigcup_{n=1}^\infty B_n$.

	So $\varphi$ is $\sigma$-finite on  $(B, B\cap \mathscr{F})$,
	the R-N derivative exists.

	For all $C \subset B^c$, we must have $\varphi(C) = 0$ or $ \infty$.
	TODO!!
\end{proof}

At last we come to the full statement:
\begin{theorem}
    Let $\varphi$ be a signed measure, $\mu$ a $\sigma$-finite measure,
	if  $\varphi\ll\mu$, then $\frac{\dd \varphi}{\dd \mu}$ exists.
\end{theorem}
\begin{example}
    Let $X = \mathbb{R}$, $\mu(A) = \#A$, $\mu$ is not $\sigma$-finite.
	Let  $\varphi(A) = 0$ when $A$ countable, $1$ otherwise.

	In this case the R-N derivative doesn't exist, otherwise
	\[
	0 = \varphi(\{x\}) = \int_{\{x\}} f\dd \mu = f(x)\mu(x) = f(x),
	\]
	contradiction!
\end{example}
