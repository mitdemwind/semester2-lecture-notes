%! TeX root = ./main.tex
\subsection{Generation of $\sigma$-algebras}
\label{sub:Generation of sigma-algebras}

Let $\mathscr{E}$ be a nonempty collection of sets.
\begin{definition}[Generate rings]
	We say $\mathscr{G}$ is the ring (algebra, etc.) generated by $\mathscr{E}$, if
	\begin{itemize}
		\item $ \mathscr{G}\supseteq \mathscr{E}$;
		\item For any ring $\mathscr{G}'$, $ \mathscr{G}'\supseteq \mathscr{E}$
			$\implies \mathscr{G}'\supseteq \mathscr{G}$
	\end{itemize}
\end{definition}
\begin{proposition}
	The ring (or whatever) generated by $\mathscr{E}$ always exists.
\end{proposition}
\begin{proof}[Proof]
    Let $\mathbf{A}$ be the set consisting of the rings
	containing $\mathscr{E}$, then $\bigcap_{\mathscr{G}\in \mathbf{A}} \mathscr{G}$
	is the desired ring.
\end{proof}

Denote $r(\mathscr{E}), m(\mathscr{E}), p(\mathscr{E}), l(\mathscr{E}), \sigma(\mathscr{E})$
the ring/monotone class/$\pi$-system/$\lambda$-system/$\sigma$-algebra
generated by  $\mathscr{E}$.
\begin{theorem}
    Let $\mathscr{A}$ be an algebra, then $\sigma(\mathscr{A}) = m(\mathscr{A})$.
\end{theorem}
\begin{proof}[Proof]
    Clearly $\sigma(\mathscr{A})\supseteq m(\mathscr{A})$.

	On the other hand, we only need to prove $m(\mathscr{A})$ is
	a $\sigma$-algebra.

	Since $\mathscr{A}$ is an algebra, so $X\in \mathscr{A} \subset m(\mathscr{A})$.

	\textbf{For the completion:}

	Let $\mathscr{G}:=\{A:A^c\in m(\mathscr{A})\}$,
	we want to prove $\mathscr{G}\supseteq m(\mathscr{A})$.

	Clearly $\mathscr{A} \subset \mathscr{G}$; If $A_1,A_2,\dots\in \mathscr{G}$,
	$A_n\uparrow A$, then
	\[
	A_n^c\in m(\mathscr{A}) \implies A^c = \downarrow\lim_{n} A_n^c\in m(\mathscr{A}).
	\]
	Similarly if $A_n\downarrow A$, we can also deduce  $A^c\in m(\mathscr{A})$.

	So $\mathscr{G}$ is a monotone class containing $\mathscr{A}$, hence
	it must contain $m(\mathscr{A})$ $ \implies \forall A\in m(\mathscr{A})$,
	$A^c\in m(\mathscr{A})$.

	\textbf{For the intersection:}
	\begin{itemize}
		\item $\forall A\in \mathscr{A}, B\in m(\mathscr{A}), AB\in m(\mathscr{A})$ :
			If $B\in \mathscr{A}$, this clearly holds;

			Moreover, such $B$'s constitude a monotone class:
			\begin{claim}
				Let $\mathscr{M}$ be a monotone class, then $\forall C\in \mathscr{M}$,
				$\mathscr{G}_C = \{D: CD\in \mathscr{M}\}$ is a monotone class.
			\end{claim}
			If  $D_1,D_2,\dots \to D$ satisfy $C\cap D_i\in m(\mathscr{A})$,
			then $D\cap C = \lim_n D_i\cap C \in \mathscr{M}$.

			Therefore such $B$'s constitude a monotone class
			$\mathscr{G}_A$ containing $\mathscr{A}$
			$\implies \mathscr{G}_A\supseteq m(\mathscr{A})$.
		\item All the $A$'s which satisfies the first condition constitude a
			monotone class:

			Let  $\mathscr{G}_B = \{A: AB\in m(\mathscr{A})\}$,
			then $\mathscr{G} = \bigcup_{B\in m(\mathscr{A})} \mathscr{G}_B$
			is a monotone class containing $\mathscr{A}$.

			Hence $\mathscr{G}\supseteq m(\mathscr{A})\implies$
			$\forall A\in m(\mathscr{A})$, $\forall B\in m(\mathscr{A})$,
			we have $AB\in m(\mathscr{A})$.
	\end{itemize}
\end{proof}

\begin{theorem}[$\lambda$-$\pi$ theorem]
    Let $\mathscr{P}$ be a $\pi$-system, then  $\sigma(\mathscr{P})=l(\mathscr{P})$.
\end{theorem}
\begin{proof}[Proof]
    Obviously $\sigma(\mathscr{P})\supseteq l(\mathscr{P})$.

	We only need to check that $l(\mathscr{P})$ is a $\pi$-system,
	i.e. closed under intersection.

	\begin{claim}
		If $\mathscr{L}$ is a $\lambda$-system, then  $\forall C\in \mathscr{L}$,
		$\mathscr{G}_C$ is a $\lambda$-system, where
		\[
		\mathscr{G}_C := \{D: CD\in \mathscr{L}\}.
		\]
	\end{claim}
	\begin{proof}[Proof of the claim]
		First of all, $X\in \mathscr{G}_C$ as $CX = C\in \mathscr{G}_C$.

		Second, if $D_1,D_2\in \mathscr{G}_C$,
		\[
			CD_1,CD_2\in \mathscr{L}\implies C(D_1-D_2)=CD_1-CD_2\in \mathscr{L}
			\implies D_1-D_2\in \mathscr{G}_C.
		\]

		Lastly, if $D_n\in \mathscr{G}_C$, $D_n \to D$,
		\[
		CD_n\in \mathscr{L}\implies CD = \lim_{n}CD_n \in \mathscr{L}
		\implies D\in \mathscr{G}_C
		\]
	\end{proof}

	The rest is similar to the previous theorem:
	\begin{itemize}
		\item $\forall A\in \mathscr{P}, B\in l(\mathscr{P}), AB\in l(\mathscr{P})$ :
			If $B\in \mathscr{P}$ this clearly holds;

			By the claim, $\mathscr{G}_A = \{B:AB\in l(\mathscr{P})\}$ is
			a $\lambda$-system, so  $\mathscr{G}_A\supseteq l(\mathscr{P})$.
		\item For $B\in l(\mathscr{P})$, let
			\[
			\mathscr{G}_B = \{A: AB\in l(\mathscr{P})\}.
			\]
			By our claim, $\mathscr{G}_B$'s are $\lambda$-systems.
			So  $\mathscr{G} = \bigcap_{B\in l(\mathscr{P})} \mathscr{G}_B$
			is a $\lambda$-system.

			Moreover $\mathscr{G}\supseteq \mathscr{P}$ (This is proved above),
			so $\mathscr{G}\supseteq l(\mathscr{P})$.

			This means  $\forall A,B\in l(\mathscr{P}), AB\in l(\mathscr{P})$.
	\end{itemize}
\end{proof}

\begin{remark}
    These two proofs are very similar. Note how we make use of the conditions.
\end{remark}

Let $X$ be a topological space,  $\mathscr{O}$ is the collection of all the
open sets.

Let $\mathscr{B}_X:=\sigma(\mathscr{O})$ be the \vocab{Borel $\sigma$-algebra}
on the space $X$,  $B\in \mathscr{B}_X$ are called \vocab{Borel sets},
and  $(X, \mathscr{B}_X)$ is called the \vocab{topological measurable space}.

\begin{theorem}
    Let $ \mathscr{Q}$ be a semi-ring, then
	\[
	r(\mathscr{Q}) = \mathscr{G} := \bigcup_{n=1}^\infty \left\{\sum_{k=1}^n A_k:
	A_1,\dots,A_n\in \mathscr{Q} \text{ and pairwise disjoint}\right\}.
	\]
\end{theorem}
\begin{remark}
    If $\mathscr{R}$ is a ring, then $\mathscr{A}=a(\mathscr{R})=\mathscr{R}\cup
	\{A^c: A\in \mathscr{R}\}$ can also be written out explicitly,
	while $\sigma(\mathscr{A})$ usually cannot be expressed explicitly.
\end{remark}
\begin{proof}[Proof]
    Since $r(\mathscr{Q})$ is closed under finite unions,
	so $r(\mathscr{Q})\supseteq \mathscr{G}$.

	Reversely, $\mathscr{G}$ is nonempty.
	In fact we only need to prove it's closed under subtraction,
	as \[
	A\cap B = A \backslash (A\backslash B)\in \mathscr{G},
	A\cup B = (A\backslash B)\cup (B\backslash A)\cup (A\cap B)\in \mathscr{G}.
	\]
	Suppose $A = \sum_{i=1}^n A_i, B = \sum_{j=1}^m B_j$.

	Then $A_i\backslash B_1$ can be split to several disjoint sets $C_k$ in $\mathscr{Q}$.
	Continue this process, each $C_k$ can be split again into smaller set.
	When all of the $B_j$'s are removed, we end up with many tiny sets
	which are in  $\mathscr{Q}$ and pairwise disjoint.
	(This process can be formalized using induction)

	Therefore $A\backslash B\in \mathscr{G}$, the conclusion follows.
\end{proof}

\subsection{Measurable maps and measurable functions}
\label{sub:Measurable maps and measurable functions}
For a map $f:X\to Y$, we say the \vocab {preimage} of $B \subset Y$
is $f^{-1}(B) := \{x: f(x)\in B\}$.

Some properties of preimage:
	\[
	f^{-1}(\emptyset) = \emptyset,\quad f^{-1}(Y) = X;
	\]
	\[
	B_1 \subset B_2\implies f^{-1}(B_1)\subset f^{-1}(B_2),
	\quad (f^{-1}(B))^c = f^{-1}(B^c);
	\]
	\[
	f^{-1}\left(\bigcup_{t\in T}A_t\right) = \bigcup_{t\in T}f^{-1}(A_t),\quad
	f^{-1}\left(\bigcap_{t\in T}A_t\right) = \bigcap_{t\in T}f^{-1}(A_t).
	\]
\begin{proposition}
	Let $ \mathscr{T}$ be a $\sigma$-algebra on $Y$, then $ f^{-1}(\mathscr{T})$
	is also a $\sigma$-algebra on $X$.

	Furthermore, for $ \mathscr{E}$ on  $Y$,
	\[
	\sigma(f^{-1}(\mathscr{E})) = f^{-1}(\sigma(\mathscr{E})).
	\]
\end{proposition}
\begin{proof}[Proof]
    $f^{-1}(\mathscr{E}) \subset f^{-1}(\sigma(\mathscr{E}))\implies$
	$f^{-1}(\sigma(E))\supseteq \sigma(f^{-1}(\mathscr{E}))$.

	Again, let
	\[
	\mathscr{G}:=\{B \subset Y: f^{-1} (B) \in \sigma(f^{-1}(\mathscr{E}))\}.
	\]
	We need to prove $\mathscr{G}$ is a $\sigma$-algebra.
	This can be checked easily by previous properties, so I leave them out.
	Hence $\mathscr{G} \supseteq \mathscr{E}\implies
	\mathscr{G} \supseteq \sigma(\mathscr{E})$
	$\implies f^{-1}(\sigma(\mathscr{E})) \subset \sigma(f^{-1}(\mathscr{E}))$.
\end{proof}

\begin{definition}[Measurable maps]
	Let $(X, \mathscr{F})$ and $(Y, \mathscr{S})$, and $f:X\to Y$ a map.
	We say  $f$ is \vocab{measurable} if $f^{-1}(\mathscr{S}) \subset \mathscr{F}$,
	i.e. the preimage of measurable sets are also measurable,
	denoted by
	\[
	f:(X,\mathscr{F})\to(Y,\mathscr{S}) \quad\text{or}\quad
	(X,\mathscr{F})\xrightarrow{f}(Y, \mathscr{S}) \quad\text{or}\quad f\in \mathscr{F}.
	\]
Clearly the composition of measurable maps is measurable as well.
\end{definition}

A map $f$ is measurable is equivalent to  $\sigma(f)\subset \mathscr{F}$, where
\[
\sigma(f):= f^{-1}(\mathscr{S})
\]
is the smallest $\sigma$-algebra which makes $f$ measurable,
called the generate $\sigma$-algebra of $f$.
\begin{theorem}
    Let $ \mathscr{E}$ be a nonempty collection on $Y$, then
	\[
		f:(X,\mathscr{F})\to (Y, \sigma(\mathscr{E})) \iff
		f^{-1}(\mathscr{E})\subset \mathscr{F}.
	\]
\end{theorem}
\begin{proof}[Proof]
    Trivial.
\end{proof}
