%! Tex root = ./main.tex
\section{Introduction}
\label{sec:Introduction}
\begin{center}
	\sffamily\large\bfseries Teacher: Zhang Fuxi

	Email: zhangfxi@math.pku.edu.cn

	Homepage: \url{http://www.math.pku.edu.cn/teachers/zhangfxi}
\end{center}

\subsection{Starting from probablistics}
\label{sub:Starting from probablistics}

\begin{definition}[$\sigma$-algebra]
	Let $\mathscr{F}$ be a family of subsets of a set $\Omega$, if
	\begin{itemize}
		\item $\Omega\in \mathscr{F}$;
		\item If $A\in \mathscr{F}$, $A^c\in \mathscr{F}$;
		\item If  $ A_1,A_2,\dots\in \mathscr{F}$,
			then $\bigcup_{i=1}^\infty A_i\in \mathscr{F}$.(Countable union)
	\end{itemize}
	then we call $\mathscr{F}$ a $\sigma$-algebra.
\end{definition}

Some intros about probablistics (left out because I haven't studied
probablistics yet;)

\subsection{What is measure theory?}
\label{sub:What is measure theory?}
It's an abstract theory, different from probablistics and real analysis.
In this course we study a general set $X$,
focus on mathematical thinking and skills, from the simple
to construct the complex.

Measure theory studies the intrinsic structure of mathematical objects,
and the map between different measure spaces.

\section{Measure spaces and measurable maps}
\label{sec:Measure spaces and measurable maps}
\subsection{Sets and set operations}
\label{sub:Sets and set operations}
\begin{definition}
	A non-empty set $X$ is our space(universal set),
	its elements (points) are denoted by
	lower case letters $x,y,\dots$.

	Some notations:
	\[
	x\in A, x\notin A, x\in A^c, A \subset B, A\cup B, AB = A\cap B,
	\]
	\[
	B\backslash A (B - A\text{ when } A \subset B), A\Delta B.
	\]
	A family of sets $\{A_t, t\in T\}$.
	\[
	\bigcup_{t\in T}A_t := \{x: \exists t\in T, s.t. x\in A_t\},\quad
	\bigcap_{t\in T}A_t := \{x: x\in A_t, \forall t\in T\}.
	\]
	Sometimes we write the union of disjoint sets as sums to
	emphasize the disjoint property.

	Monotone sequence of sets:
	\[
	A_n\uparrow: A_n \subset A_{n+1}, \forall n;\quad
	A_n \downarrow:	A_n \supseteq A_{n+1}, \forall n.
	\]
\end{definition}

Next we define the limits of sets:
\begin{definition}
	For monotone sequences:
	\[
	\lim_{n\to \infty}A_n := \bigcup_{n=1}^\infty A_n \text{ or }
	\bigcap_{n=1}^\infty A_n.
	\]
	For general sequence of sets:
	\[
	\limsup A_n := \bigcap_{n=1}^\infty \bigcup_{k=n}^\infty A_n;\quad
	\liminf A_n := \bigcup_{n=1}^\infty \bigcap_{k=n}^\infty A_n.
	\]
	A clearer intepretation of limsup and liminf:

	limsup is the set of elements which occurs infinitely many times in $A_n$,
	and liminf is the elements which doesn't occur in only finitely many $A_n$'s.
\end{definition}

\subsection{Families of sets}
\label{sub:Families of sets}
\begin{definition}
	A family of sets is denoted by script letters $\mathscr{A},\mathscr{B},\dots$.

	\begin{itemize}
		\item A family is a $\pi$-family if  $\mathscr{P}\ne \emptyset$ and
			it's closed under intersections, e.g.
			$\{(-\infty, a]: a\in \mathbb{R}\}$.
		\item Semi-rings: $\mathscr{Q}$ is a $\pi$-family, and for all
			$A \subset B$, then there exists finitely many pairwise disjoint sets
			$C_1,\dots,C_n\in \mathscr{Q}$ s.t.
			\[
			B\backslash A = \bigcup_{k=1}^n C_k = \sum_{k=1}^n C_k.
			\]
			e.g. $\mathscr{Q}= \{(a,b]:a,b\in \mathbb{R}\}$.
			\begin{remark}
				The condition $A \subset B$ can be removed.
			\end{remark}
		\item Rings: $\mathscr{R}$ is nonempty, and it's closed under
			union and substraction.

			e.g. $\mathscr{R} = \{\bigcup_{k=1}^n (a_k,b_k]:a_k,b_k\in \mathbb{R}\}$.
		\item Algebras (fields): $\mathscr{A}$ is a $\pi$-family that contains
			$X$, and is closed under completion.
	\end{itemize}
\end{definition}

\begin{proposition}
	Semi-rings are $\pi$-families, rings are semi-rings, algebras are rings.
\end{proposition}
\begin{proof}[Proof]
    By definition we only need to check rings are $\pi$-families:
	$A\cap B = A\backslash (A\backslash B)$.

	For algebras, $A\cup B = (A^c\cap B^c)^c, A\backslash B = A\cap B^c$,
	so they are rings.
\end{proof}
\begin{remark}
    Rings are semi-rings with unions, Algebras are rings with universal set $X$.
\end{remark}

\begin{definition}
	Some other families that start from taking limits:
	\begin{itemize}
		\item Monotone families: If $A_1,\dots\in \mathscr{U}$ and $A_n$
			monotone, then  $\lim_{n\to \infty}A_n =\mathscr{U}$.
		\item $\lambda$-families:
			\[
			X\in \mathscr{L};\quad
			A_1,A_2,\dots\in \mathscr{L}, A_n \uparrow A\implies A\in \mathscr{L};
			\]
			\[
			A,B\in \mathscr{L}, A \supseteq B \implies A\backslash B \in\mathscr{L}.
			\]
			notes: $A_n\in \mathscr{L}\iff B_n = A_n^c\in \mathscr{L}$.

		\item $\sigma$-algebra:
			\[
			X\in \mathscr{F};\quad A\in \mathscr{F}\implies A^c\in \mathscr{F};
			\]
			\[
			A_1,A_2,\dots\in \mathscr{F}\implies
			\bigcup_{n=1}^\infty A_n\in \mathscr{F}.
			\]
	\end{itemize}
\end{definition}
\begin{proposition}
	$\sigma$-algebra = algebra \& monotone family;

	$\sigma$-algebra =  $\lambda$-family \&  $\pi$-family.
\end{proposition}
\begin{definition}
	$\sigma$-rings:  $\mathscr{R}$ nonempty, $A,B\in \mathscr{R}\implies
	A\backslash B\in \mathscr{R}$ ;
	\[
	A_1,A_2,\dots\in \mathscr{R}\implies \bigcup_{n=1}^\infty A_n\in \mathscr{R}.
	\]
	Note: We only need to verify the case when $A_n$'s are disjoint.
\end{definition}

\begin{definition}[Measurable space]
	Let $\mathscr{F}$ be a $\sigma$-algebra on a set $X$,
	we say $(X, \mathscr{F})$ is a measurable space.
\end{definition}

\begin{proposition}
	Let $(X,\mathscr{F})$ be a measure space, $A$ is a subset of $X$.
	Then $(A, A\cap \mathscr{F})$ is also a measurable space.
\end{proposition}

The smallest $\sigma$-algebra is  $\{\emptyset, X\}$,
the largest $\sigma$-algebra is the power set $\mathscr{T}=\mathcal{P}(X)$.

In some cases, $\mathscr{T}$ is too large, for example,
when $X=\mathbb{R}$, we can't assign a ``measure'' to every subset
that fits our common sense.
