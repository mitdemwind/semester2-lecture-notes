%! TeX root = ./main.tex
\begin{theorem}
    Let $\mu$ be a set function on a ring with finite additivity,
	then $1\iff 2\iff 3\implies 4\implies 5$.
	\begin{itemize}
		\item $\mu$ is countablely additive;
		\item $\mu$ is countablely subadditive;
		\item $\mu$ is lower continuous;
		\item $\mu$ is upper continuous;
		\item $\mu$ is continuous at $\emptyset$.
	\end{itemize}
\end{theorem}

\subsection{Outer measure}
\label{sub:Outer measure}

Once we construct a measure on a semi-ring, we want to extend it
to a $\sigma$-algebra. Since we can't directly do this, we shall relax some of
our restrictions, say reduce countable additivity to subadditivity.

\begin{definition}[Outer measure]
	Let $\tau: \mathscr{T}\to [0,\infty]$ satisfying:
	\begin{itemize}
		\item $\tau(\emptyset) = 0$;
		\item If  $A \subset B \subset X$, then $\tau(A)\le \tau(B)$;
		\item (Countable subadditivity) $\forall A_1,A_2,\dots\in \mathscr{T}$, we have
			\[
			\tau\left( \bigcup_{n=1}^\infty A_n \right)\le \sum_{n=1}^{\infty}\tau(A_n).
			\]
	\end{itemize}
	We call $\tau$ an \vocab{outer measure} on $X$.
\end{definition}

It's easier to extend a measure on semi-ring to an outer measure:
\begin{theorem}
    Let $\mu$ be a non-negative set function on a collection  $\mathscr{E}$,
	where $\emptyset\in \mathscr{E}$ and $\mu(\emptyset)=0$.
	Let
	\[
	\tau(A) := \inf \left\{ \sum_{n=1}^{\infty} \mu(B_n): B_n\in \mathscr{E}, \bigcup_{n=1}^\infty B_n \supseteq A \right\}, \quad \forall A\in \mathscr{T}.
	\]
	By convention, $\inf \emptyset = \infty$.
	(\textit{$\mu$ need not be a measure!})

	Then $\tau$ is called the outer measure generated by $\mu$.
\end{theorem}

\begin{proof}[Proof]
    Clearly $\tau(\emptyset) = 0$, and  $\tau(A)\le \tau(B)$ for $A \subset B$.
	\[
	\bigcup_{n=1}^\infty B_n \supseteq B \implies \bigcup_{n=1}^\infty B_n \supseteq A.
	\]

	For all $A_1,A_2,\dots\in \mathscr{T}$, WLOG $\tau(A_n)<\infty$.
	Take $B_{n,k}$ s.t. $\bigcup_{k=1}^\infty B_{n,k}\supseteq A_n$, such that
	\[
	\sum_{k=1}^{\infty}\mu(B_{n,k}) < \tau(A_n) + \frac{\varepsilon}{2^n}.
	\]
	Therefore
	\[
	\bigcup_{n=1}^\infty\bigcup_{k=1}^\infty B_{n,k}\supseteq A_n,
	\]
	\[
	\tau\left(\bigcup_{n=1}^\infty A_n\right)
	\le \sum_{n=1}^{\infty}\sum_{k=1}^{\infty}\mu(B_{n,k}) + \varepsilon
	\le \sum_{n=1}^{\infty} \tau(A_n) + \varepsilon.
	\]
\end{proof}

\begin{example}
    Let $\mathscr{E}=\{X,\emptyset\}$, $\mu(X)=1$, $\mu(\emptyset)=0$.
	Then  $\tau(A)=1$, $\forall A\ne \emptyset$.
\end{example}
\begin{example}
    Let $X=\{a,b,c\}$, $\mathscr{E}=\{\emptyset,\{a\},\{a,b\},\{c\}\}$.
	$\mu(A) = \#A$ for $A\in \mathscr{E}$.

	Here something strange happens:
	$\tau(\{b\}) = 2$ instead of $1$, and  $\tau(\{b,c\}) = 3$ instead of $2$.
\end{example}

In the above example,
we found the set $\{b\}$ somehow behaves badly:
if we divide $\{a,b\}$ to $\{a\}+\{b\}$, the outer measure is not the sum
of two smaller measure.

Hence we want to get rid of this kind of inconsistency to get
a proper measure:
\begin{definition}[Measurable sets]
	Let $\tau$ be an outer measure, if a set  $A$ satisfies
	\textit{Caratheodory condition}:
	\[
	\tau(D) = \tau(D\cap A) + \tau(D\cap A^c), \quad \forall D\in \mathscr{T},
	\]
	we say $A$ is \vocab{measurable}.
\end{definition}
\begin{remark}
    Inorder to prove $A$ measurable, we only need to check
	\[
	\tau(D)\ge \tau(D\cap A)+\tau(D\cap A^c), \quad \forall D\in \mathscr{T}.
	\]
\end{remark}

Let $\mathscr{F}_\tau$ be the collection of all the $\tau$ measurable sets,

\begin{definition}[Complete measure space]
	Let $(X,\mathscr{F},\mu)$ be a measure space, if for all null set $A$,
	and $\forall B \subset A, B\in \mathscr{F} \implies \mu(B)=0$,
	we say $(X,\mathscr{F},\mu)$ is \vocab{complete}.
\end{definition}

\begin{theorem}[Caratheodory's theorem]
    Let $\tau$ be an outer measure, then $\mathscr{F}:=\mathscr{F}_\tau$ is
	a $\sigma$-algebra, and $(X,\mathscr{F},\tau)$ is a complete measure space.
\end{theorem}
\begin{proof}[Proof]
    First we prove $\mathscr{F}$ is an algebra:

	Note $\emptyset\in \mathscr{F}$, and  $\mathscr{F}$ is closed under completements.

	For measurable sets $A_1,A_2$,
	\begin{align*}
	\tau(D) &= \tau(D\cap A_1)+\tau(D\cap A_1^c)\\
	&= \tau(D\cap A_1\cap A_2)+\tau(D\cap(A_1)\cap A_2^c)+\tau(D\cap A_1^c)\\
	&= \tau(D\cap (A_1\cap A_2)) + \tau(D\cap (A_1\cap A_2)^c).
	\end{align*}
	So $A_1\cap A_2$ is measurable.

	Secondly, we prove $\mathscr{F}$ is a $\sigma$-algebra.

	Let $A_1,A_2,\dots\in \mathscr{F}$,
	\[
	B_n := A_n \backslash \bigcup_{i=1}^{n-1}A_i \in \mathscr{F},
	\]
	Then $B_i$ pairwise disjoint and $\bigcup_{i=1}^\infty A_i = \bigcup_{i=1}^\infty B_i$.
	Let $B_f = \bigcup_{i=1}^\infty B_i$.

	It's sufficient to prove
	\[
	\tau(D) \ge \tau(D\cap B_f) + \tau(D\cap B_f^c).
	\]
	Let $D_n = \sum_{i=1}^n B_i \cap D$, $D_f=D\cap B_f$,
	$D_\infty = D \backslash D_f$.

	Since $B_i$ are measurable,
	\[
	\tau(D) = \tau(D_n) + \tau(D \backslash D_n)
	\ge \tau(D_n) + \tau(D_\infty) = \sum_{i=1}^{n} \tau(D\cap B_i) + \tau(D_\infty).
	\]
	Now we take $n\to \infty$,
	\[
	\tau(D)\ge \sum_{i=1}^{\infty} \tau(D\cap B_i) + \tau(D_\infty)
	\ge \tau\left(D\cap \sum_{i=1}^{\infty} B_i\right) + \tau(D_\infty).
	\]
	Where the last step follows from countable subadditivity.

	This implies $B_f$ measurable $\implies \mathscr{F}$ is a $\sigma$-algebra.

	Next we prove $\tau|\mathscr{F}$ is a measure:
	Just let $D = \sum_{i=1}^{\infty} B_i$ in the previous equation.

	Last we prove $(X,\mathscr{F},\tau)$ is complete:

	If $\tau(A) = 0$, $\tau(D)\ge \tau(D\cap A^c) = \tau(D\cap A) + \tau(D\cap A^c)$.
	Thus $A\in \mathscr{F}$.
\end{proof}

\subsection{Measure extension}
\label{sub:Measure extension}
\begin{definition}[Measure extension]
	Let $\mu$,  $\nu$ be measures on $\mathscr{E}$ and $\overline{\mathscr{E}}$,
	and $\mathscr{E}\subset\overline{\mathscr{E}}$. If
	\[
	\nu(A) = \mu(A) , \quad \forall A\in \mathscr{E},
	\]
	we say $\nu$ is a extension of  $\mu$ on $\overline{\mathscr{E}}$.
\end{definition}

If we start from a measure $\mu$ on  $\mathscr{E}$,
ideally, $\mu$ can generate an outer measure  $\tau$, and we can take
$\mathscr{F}_\tau$ to construct a mesaure space.

However, things could go wrong:
\begin{example}
    Let $X=\{a,b,c\}, \mathscr{E}=\{\emptyset, \{a,b\}, \{b,c\}, X\}$ with
	\[
	\mu(\emptyset)=0, \mu(\{a,b\})=1, \mu(\{b,c\})=1, \mu(X)=2.
	\]
	Then $\mu$ is a measure on  $\mathscr{E}$, and the outer measure
	\[
	\tau(\emptyset) = 0.
	\]

	Observe that $\mathscr{F}_\tau=\{\emptyset, X\}$, so in this case
	$\tau|_{\mathscr{F}}$ is the trivial measure.
\end{example}
\begin{example}
	Let $X=\mathbb{R}$, $\mathscr{E}=\{(a,b]: a<b, a,b\in \mathbb{R}\}$.
	Let $\mu(\emptyset)=0$, and  $\mu(A)=\infty$ for $A\ne \emptyset$.

	Then $\mu$ can be extend to the Borel $\sigma$-algebra on $\mathbb{R}$
	with $\mu_\alpha = \sum_{q\in \mathbb{Q}} \alpha\delta_q$, $\forall \alpha\ge 0$.
	So the extension is not unique.
\end{example}

Therefore in order to get a ``proper'' extension,
we must put some requirements on both
the starting collection and the set function $\mu$.

\begin{proposition}
	Let $ \mathscr{P}$ be a $\pi$ system. If two measures $\mu,\nu$
	on $\sigma(\mathscr{P})$ satisfying
	\[
	\mu|_{\mathscr{P}} = \nu|_{\mathscr{P}}, \quad \mu|_{\mathscr{P}}
	\text{ is $\sigma$-finite,}
	\]
	Then  $\mu = \nu$.
\end{proposition}
\begin{proof}[Proof]
    Let $A_1,A_2,\cdots\in \mathscr{P}$ s.t. $X = \sum_{n=1}^\infty A_n$ and
	$\mu(A_n)<\infty$.

	Fix $n$, let $B = A_n$, we want to prove that
	\[
		\mu(B\cap A)=\nu(B\cap A),\quad \forall A\in \sigma(\mathscr{P}).
	\]

	Let $B\in \mathscr{P}$ with $\mu(B)<\infty$,
	\[
	\mathscr{L} := \{A\in \sigma(\mathscr{P}): \mu(A\cap B) = \nu(A\cap B)\}.
	\]
	We'll prove $\mathscr{L}$ is a $\lambda$ system,
	so that $\mathscr{L}\supseteq \sigma(\mathscr{P})$.

	Suppose $A_1,A_2\in \mathscr{L}$ and $A_1\supseteq A_2$, by $\mu(B)<\infty$,
	\[
	\mu((A_1-A_2)B) = \mu(A_1B) - \mu(A_2B)
	= \nu(A_1B-A_2B) = \nu((A_1-A_2)B).
	\]
	So $A_1-A_2\in \mathscr{L}$.

	Let $A_1,A_2,\dots\in \mathscr{L}$ and $A_n\uparrow A$, then
	 \[
	\mu(AB) = \lim_{n\to \infty}\mu(A_nB) = \lim_{n\to \infty}\nu(A_n B) = \nu(AB).
	\]
	Which implies $A\in \mathscr{L}$.

	Hence $\sigma(\mathscr{P}) \subset \mathscr{L}$, i.e.
	\[
	\mu(A\cap A_n) = \nu(A\cap A_n), \quad\forall A\in \sigma(\mathscr{P}).
	\]
	
	Therefore
	\[
	\mu(A) = \sum_{n=1}^{\infty}\mu(A\cap A_n) = \sum_{n=1}^{\infty} \nu(A\cap A_n)
	= \nu(A), \quad \forall A\in \sigma(\mathscr{P}).
	\]
\end{proof}
