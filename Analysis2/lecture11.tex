%! TeX root = ./main.tex
There's another version of this thoerem which looks like
Newton-Lebniz formula more:
\begin{theorem}
	Let $F$ be a differentiable function on $[a,b]$,
	if $F'$ is Lebesgue integrable, then
	\[
	F(b) - F(a) = \int_{a}^{b} F'(x)\dd x.
	\]
\end{theorem}

We need to prove a lemma first.
\begin{theorem}
	Let $F$ be real function on $[a,b]$, if $F$ is differentiable on $E$,
	and $|F'|\le M$ in $E$, then
	\[
	m^*(F(E)) \le Mm^*(E).
	\]
\end{theorem}
\begin{proof}[Proof]
    For all $\varepsilon > 0$, $x\in E$, $\exists \delta > 0$,
	\[
	\left| \frac{F(x+h) - F(x)}{h} - M \right| < \varepsilon,\quad
	\forall |h| < \delta.
	\]
	So $[x-h, x+h]$ is a Vitali covering of $E$.
	By Vitali's theorem (\ref{thm:vitali}), exists disjoint
	intervals $I_i = [x_i-h_i, x_i+h_i]$ s.t.
	\[
	m^*\left(E \backslash \bigcup_{i=1}^\infty I_i\right) = 0,\quad
	\sum_{i=1}^{\infty} 2h_i \le m^*(E) + \varepsilon.
	\]
	But for $y\in I_i$, $|F(y) - F(x_i)| \le (M+\varepsilon)h_i$,
	thus $m^*(F(I_i)) \le 2(M+\varepsilon)h_i = (M+\varepsilon)|I_i|$.
	\begin{align*}
		m^*(F(E)) &\le m^*(F(E\cap \bigcup_{i=1}^\infty I_i))
		+ m^*(F(E \backslash \bigcup_{i=1}^\infty I_i))\\
		& \le \sum_{i=1}^{\infty} m^*(F(I_i))
		+ m^*(F(E \backslash \bigcup_{i=1}^\infty I_i))\\
		& \le (M+\varepsilon)(m^*(E) + \varepsilon)
		+ m^*(F(E \backslash \bigcup_{i=1}^\infty I_i))
	\end{align*}

	So it suffices to prove the case when $E$ is null.
	Define
	\[
	E_n = \left\{x\in E: |F(y) - F(x)|\le (M+\varepsilon)|y - x|,
	\forall |y - x|<\frac{1}{n}\right\}.
	\]
	Observe that $E_n \nearrow E$ and $F(E_n)\nearrow F(E)$.
	There exists disjoint intervals $J_{n,k}$ s.t.
	\[
	E_n \subset \bigcup_{k=1}^\infty J_{n,k},\quad
	\sum_{k=1}^{\infty} |J_{n,k}| \le \min\left\{\frac{1}{n}, \varepsilon\right\}.
	\]
	Thus
	\[
	m^*(F(E_n)) \le \sum_{k=1}^{\infty} m^*(F(E_n \cap J_{n,k}))
	\le \sum_{k=1}^{\infty} (M+\varepsilon) |J_{n,k}|\le \varepsilon(M+\varepsilon).
	\]

	Taking $\varepsilon \to 0$ we get $F(E_n)$ is null.
	So $F(E) = \lim_{n\to \infty} F(E_n)$ is null, which completes the proof.
\end{proof}

Returning to the proof of the theorem,
in fact we only need to prove
\[
|F(b) - F(a)|\le \int_{a}^{b} |F'(x)|\dd x,
\]
since this implies $F$ is absolutely continuous.
For all $\varepsilon > 0$, let
\[
	E_n = \{x\in [a,b]: n\varepsilon \le |F'(x)| < (n+1)\varepsilon\}.
\]
By our lemma, $m^*(F(E_n))\le (n+1)\varepsilon m(E_n)\le \varepsilon m(E_n)
+ \int_{E_n} |F'(x)|\dd x$.

Hence
\begin{align*}
	|F(b) - F(a)| &\le m(F([a,b]))
	\le \sum_{n=0}^{\infty} m^*(F(E_n))\\
	&\le \varepsilon(b-a) + \int_{a}^{b} |F'(x)|\dd x.
\end{align*}

\begin{theorem}
    A rectifiable curve $\gamma(t) = (x(t), y(t))$ with $x,y$ absolutely
	continuous has length
	\[
	L(\gamma) = \int_{a}^{b} |\gamma'(t)|\dd t.
	\]
\end{theorem}
\begin{proof}[Proof]
    Since $|\gamma(t_i) - \gamma(t_{i-1})| = |\int_{t_{i-1}}^{t_i} \gamma'(t)\dd t|
	\le \int_{t_{i-1}}^{t_i} |\gamma'(t)|\dd t$,
	thus $L(\gamma)\le \int_{a}^{b} |\gamma'(t)|\dd t$.

	$\forall \varepsilon>0$, we can take a step function (with vector values)
	$g$ s.t. $\gamma' = g + h$, and $\int_{a}^{b} |h|\dd x < \varepsilon$.

	Define
	\[
	G(x) = G(a) + \int_{a}^{x} g(t)\dd t,\quad
	H(x) = H(a) + \int_{a}^{x} h(t)\dd t.
	\]
	We have $\gamma(t) = G(t) + h(t)$,
	and $T_\gamma([a,b])\ge T_G([a,b])-T_H([a,b])$.
	\begin{align*}
		L(\gamma) =
		T_\gamma([a,b]) &\ge \int_{a}^{b} |g|\dd t - \int_{a}^{b} |h|\dd t\\
		&\ge \int_{a}^{b} |\gamma'(t)|\dd t - 2\int_{a}^{b} |h|\dd t\\
		&\ge \int_{a}^{b} |\gamma'(t)|\dd t - 2 \varepsilon.
	\end{align*}
	which gives the opposite inequality.
\end{proof}

\begin{proposition}[substitution formula]
	Let $\phi:[a,b]\to [c,d]$ be strictly increasing AC function.
	For a function $f$ on $[c,d]$, we have
	\[
	\int_{c}^{d} f(y)\dd y = \int_{a}^{b} f(\phi(x))\phi'(x) \dd x.
	\]
\end{proposition}
\begin{proof}[Proof]
    It's equivalent to $m(\phi(E)) = \int_E \phi' \dd x$.
\end{proof}

\section{Multi-dimensional Calculus}
\label{sec:Multi-dimensional Calculus}

In this section we'll generalize the differentiation and integration theory
to higher dimensions. Recall that Lebesgue integral is already
defined on higher dimensions, so here we mainly study the differentiation
of multi-dimensional functions.
