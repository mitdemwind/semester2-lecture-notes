%! TeX root =./main.tex
\begin{example}
    Consider the spherical coordinates $x=r\sin \theta \sin \varphi$,
	$y = r\sin\theta \cos\varphi$, $z = r\cos\theta$.

	Let $F: (r,\theta,\varphi)\mapsto (x,y,z)$.
	\[
	J_F = \begin{pmatrix}
		\sin\theta\sin\varphi &r\cos\theta\sin\varphi &r\sin\theta\cos\varphi\\
		\sin\theta\cos\varphi &r\cos\theta\cos\varphi &-r\sin\theta\sin\varphi\\
		\cos\theta &-\sin\theta &0
	\end{pmatrix}
	\]
	So $\det(J_F) = r^2\sin\theta$. Thus
	\[
	\int_{\mathbb{R}^{3}} f(x,y,z)\dd x\dd y\dd z
	= \int_{(0,2\pi)^2}\int_{0}^{+\infty}
	f(r, \theta, \phi)r^2\sin\theta \dd r\dd \theta \dd \varphi.
	\]
\end{example}

\begin{theorem}[Clairaut-Schwarz]
    Given an open set $\Omega \subset \mathbb{R}^{n}$ and $f: \Omega\to \mathbb{R}$.
	Assume $\pfr{f}{x_i}(x), \pfr{f}{x_j}(x), \pfr{}{x_i}(\pfr{f}{x_j})(x)$ exists
	and are continuous, then
	$\pfr{}{x_j}(\pfr{f}{x_i})(x)$ exists and
	\[
	\pfr{}{x_i}\left(\pfr{f}{x_j}\right)(x)
	= \pfr{}{x_j}\left(\pfr{f}{x_j}\right)(x).
	\]
\end{theorem}
\begin{proof}[Proof]
    WLOG $n = 2$.
	We'll just expand and compute:
	\begin{align*}
		\pfr{}{y}\pfr{f}{x}(x_0, y_0)
		&= \lim_{s\to 0}\frac{1}{s}
		\left(\pfr{f}{x}(x_0,y_0+s) - \pfr{f}{x}(x_0,y_0)\right)\\
		&= \lim_{s\to 0}\frac{1}{s} \lim_{t\to 0} \frac{1}{t}
		(f(x_0+t, y_0+s) - f(x_0, y_0+s) - f(x_0+t, y_0) + f(x_0,y_0)).
	\end{align*}

	Since
	\[
		(f(x_0+t, y_0+s) - f(x_0, y_0+s) - f(x_0+t, y_0) + f(x_0,y_0))
		= \int_{0}^{s} \int_{0}^{t} \pfr{}{x}\pfr{f}{y}
		(x_0+\tilde t,y_0+\tilde s)\dd \tilde t\dd \tilde s.
	\]
	So by Fubini's theorem,
\end{proof}

Notation:
Let $\alpha = (\alpha_1,\dots,\alpha_n)$ be a multiple index,
where $\alpha_i\ge 0$ are integers.
define
\[
\partial^\alpha f = \left(\pfr{f}{x_1}\right)^{\alpha_1}\cdots
\left(\pfr{f}{x_n}\right)^{\alpha_n}f.
\]
or we can write
\[
\partial^\alpha = \frac{\partial^{|\alpha|}}{\partial x_1^{\alpha_1}\cdots
\partial x_n^{\alpha_n}},\quad |\alpha| = \sum_{i=1}^{n} \alpha_i.
\]

\begin{theorem}[Multi-dimensional Taylor expansion]
    Let $\Omega \subset \mathbb{R}^{n}$ be a convex open set.
	Let $f\in C^{k+1}(\Omega)$, for all $x,y\in \Omega$,
	then $\exists \theta\in (0,1]$ s.t.
	\[
	f(y) = \sum_{|\alpha|\le k} \frac{\partial^\alpha f(x)}{\alpha!}(y-x)^{\alpha}
	+ \sum_{|\alpha|=k+1} \frac{\partial^\alpha f(x+\theta(y-x))}{\alpha!}
	(y-x)^\alpha.
	\]
	where $(y - x)^\alpha = \prod_{i=1}^n(x_i-y_i)^{\alpha_i}$,
	$\alpha! = \prod_{i=1}^n \alpha_i !$.
\end{theorem}
\begin{proof}[Proof]
    Let $g(t) = f(ty + (1-t)x)$, $g\in C^{k+1}((-1,1))$.
	By Taylor expansion, there exists $\theta\in [0,1]$,
	\[
	g(1) = \sum_{l=0}^{k} \frac{g^{(l)}(0)}{l!} + \frac{g^{(k+1)}(\theta)}{(k+1)!}.
	\]
	so it's just a differential formula of composite function,
	which can be easily proved by induction,
	and I don't bother to write it down.
\end{proof}

\subsection{Implicit function theorem}
\label{sub:Implicit function theorem}

As usual let $C^k(\Omega)$ denote the $k$ times continuously differentiable
functions on $\Omega$.
\begin{definition}[Differential homeomorphisms]
	Let $U, V \subset \mathbb{R}^n$, if there exists a bijection $f: U\to V$,
	such that $f, f^{-1}$ are smooth, then we say $U$ and $V$ are
	\vocab{smoothly homeomorphic}.

	Denoted by $C^{\infty} (U, V)$ the smooth homeomorphisms from $U$ to $V$.
\end{definition}

\begin{example}
    Let $f: \mathbb{R}\to \mathbb{R}$ by $x\mapsto x^3$,
	then $f$ is a smooth bijection, but $f^{-1}$ is not differentiable at $0$.
\end{example}

Recall that in $\mathbb{R}$ we have the following results:
\begin{itemize}
	\item If $f$ is strictly increasing and continuous, then $f ^{-1}$ is
		continuous.
	\item If $f$ is strictly increasing and $C^1$, $f' \ne 0$,
		then $f^{-1}\in C^1$.
\end{itemize}

\begin{theorem}
    Let $\Phi$ be an differential homeomorphism $U\to V$, $f\in C^k(V)$.
	Then $f \circ \Phi =: \Phi^* f\in C^k(\Omega)$, this is called
	the \vocab{pullback} of $f$ by $\Phi$.
\end{theorem}
\begin{proof}[Proof]
    We proceed by induction on $k$.
	When $k = 0$, this is just the continuity of composite functions.

	Assume $k = n$ holds, then for $k = n+1$,
	\[
	\pfr{\Phi^*f}{x_j} = \pfr{f(\Phi(x))}{x_j}
	= \sum_{i=1}^{n} \pfr{f}{y_i}(\Phi(x)) \cdot \pfr{\Phi_i(x)}{x_j}.
	\]
	Since $f \in C^{n+1}\implies \pfr{f}{y_i}\in C^n$,
	and $\pfr{\Phi_i}{x_j}$ is smooth, so $\pfr{\Phi^*f}{x_j}\in C^n$.
\end{proof}
