%! TeX root = ./main.tex
Recall the Gauss integral:
\[
2\int_0^\infty e^{-x^2}\dd x = \sqrt{\pi}.
\]
Here we give a different proof:
\begin{align*}
    \int e^{-x^2}\dd x \int e^{-y^2}\dd y
	&= \iint_{\mathbb{R}^2} e^{-(x^2+y^2)}\dd x\dd y\\
	&= \int_{0}^{+\infty} e^{-r^2} \dd r^2 \cdot \pi\\
	&= \pi.
\end{align*}

\section{Lebesgue differentiation}
\label{sec:Lebesgue differentiation}

The most important theorem in calculus is no doubt the
Fundamental theorem of Calculus (which is also called Newton-Lebniz formula).
Since we generalized the integrals,
there must be a generalized version of this theorem:
\begin{theorem}[Lebesgue differentiation theorem, part 1]
	If $f$ is integrable on $\mathbb{R}^d$,
	for any ball $B \subset \mathbb{R}^d$, we have
	\[
	\lim_{x\in B, |B|\to 0}\frac{1}{m(B)} \int_B f(y)\dd y = f(x), a.e.
	\]
\end{theorem}
This theorem clearly holds for continuous points of $f$.

Our basic idea is to take a continuous  $g$, such that
$\lVert g-f \rVert_{\mathcal{L}^1} <\varepsilon$.

and to prove
\[
\left\{x:
\limsup_{x\in B, |B|\to 0} \frac{1}{m(B)}\int_B|f(y)-f(x)|\dd y \ge \varepsilon_0
\right\}
\]
is a null set.

Now we estimate
\begin{align*}
    \frac{1}{m(B)}\int_B |f(y)-f(x)|\dd y
	&\le \frac{1}{m(B)}\int _B \left( |f(y)-g(y)|+|g(y)-g(x)|+|g(x)-f(x)| \right)\dd y
	\\
	&= |f(x)-g(x)| + \varepsilon + \frac{1}{m(B)}\int_B |f(y)-g(y)|\dd y
\end{align*}

We find that the last term is pretty hard to deal with,
so we'll introduce some new tools:
\begin{definition}[Hardy-Littlewood maximal function]
	Let $f$ be an integrable function on  $\mathbb{R}^d$. Define
	\[
	Mf(x) = \sup_{x\in B}\frac{1}{m(B)}\int_B |f(y)|\dd y.
	\]
	to be the \vocab{maximal function} of $f$.
\end{definition}

\begin{theorem}[Hardy-Littlewood]
    The maximal function $Mf$ satisfies:
	 \begin{itemize}
	 	\item $Mf$ is measurable;
		\item For  $x$ almost everywhere,  $|f(x)|\le Mf(x)<+\infty$.

		\item There exists a constant $C$ s.t.
			 \[
			\left|\{x: Mf> \alpha\}\right| \le
			\frac{C}{\alpha}\lVert f \rVert _{\mathcal{L}^1}.
			\]
	 \end{itemize}
\end{theorem}
\begin{proof}[Proof]
    First we prove $\{Mf>\alpha\}$ is measurable.
	If $Mf(x_0)>\alpha$, then exists an open ball $B\ni x_0$,
	\[
	\int_B |f(y)|\dd y > \alpha m(B).
	\]
	This implies that $B \subset \{Mf>\alpha\}$
	$ \implies \{Mf>\alpha\}$ is an open set.

	For the second part, we'll prove for $\forall \varepsilon_0>0, N>0$,
	\[
	m(\{x: Mf(x)+ \varepsilon_0<|f(x)|\le N\}) = 0.
	\]
	Otherwise denote the above set as $E$,
	for  $\forall 0<\lambda<1$,  $\exists B$
	s.t. $|E\cap B|>\lambda |B|$.

	Thus for $x\in E$,
	 \begin{align*}
		 Mf(x) &\ge \frac{1}{m(B)}\int_B |f(y)|\dd y\\
			   &\ge \frac{1}{m(B)}\int _{E\cap B}|f(y)|\dd y\\
			   &\ge \frac{1}{m(B)}\int_{E\cap B} \varepsilon_0+Mf(y)\dd y\\
			   &= \frac{m(E\cap B)}{m(B)}\varepsilon_0
			   + \frac{1}{|B|}\int _{E\cap B} Mf(y)\dd y.
	\end{align*}
	Taking the integral with respect to $x$:
	 \[
	\left( 1-\frac{|E\cap B|}{|B|} \right)\int_{E\cap B} Mf
	\ge \frac{|E\cap B|^2}{|B|}\varepsilon_0.
	\]
	This implies $(1-\lambda)N \ge \lambda\varepsilon_0$, which is impossible
	as $\lambda\to 1$.

	Now for the last part, since $\{Mf>\alpha\}$ is open,
	$\forall x\in \{Mf>\alpha\}$, $\exists B$ s.t.
	 \[
	\int_B |f(y)|\dd y >\alpha m(B).
	\]

	Hence for disjoint balls $B_{i_k}$,
	\[
	\lVert f \rVert _{\mathcal{L}^1}\ge
	\sum_{l=1}^k \int_{B_{i_l}}|f(y)|\dd y > \alpha\sum_{l=1}^k |B_{i_l}|.
	\]
	If we could select $B_{i_l}$'s such that their measure achieves
	say 1\% of $E$, then we're done.

	\begin{lemma}
		Let $B_1,\dots,B_n$ be open balls in $\mathbb{R}^d$.
		There exists $i_1,\dots,i_k$ such that $B_{i_j}$'s are
		pairwise disjoint, and
		\[
		\bigcup_{i=1}^n B_i \subset \bigcup_{j=1}^k 3B_{i_j}.
		\]
		Here $3B$ means to multiply the radius of the ball by 3.
	\end{lemma}
\end{proof}
