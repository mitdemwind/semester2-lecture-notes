%!TEX root=./main.tex
\section{Introduction}
\label{sec:Introduction}
\begin{center}
	\bfseries Teacher: Yang Shiwu
\end{center}

Grading: Homework-Midterm-Endterm: 30-30-40

Contents of this course: Real analysis

\subsection{Recap}
\label{sub:Recap}
\begin{definition}[Measurable space]
	Let $X$ be a set and $\mathcal{A}$ be a $\sigma$-algebra,
	we say  $(X, \mathcal{A})$ is a measurable space if
	\begin{itemize}
		\item $\emptyset \in \mathcal{A}$;
		\item If $A\in \mathcal{A}$, then $A^c\in \mathcal{A}$;
		\item If  $A_k\in \mathcal{A}$, then $\bigcup_{k=1}^{+\infty } \in \mathcal{A}$.
	\end{itemize}
\end{definition}

Outer measure $m^*$:
\begin{itemize}
	\item $m^*(A)\ge 0$;
	\item  $m^*(\bigcup_{k=1}^{\infty }A_k)\le \sum_{k=1}^{\infty} m^*(A_k)$;
	\item $m^*(A)\le m^*(B)$ when $A\subset B$.
\end{itemize}

Caratheodry condition:
\[
m^*(T) = m^*(T \cap E) + m^*(T \cap E^c),\quad \forall T \subset X
.\]
We define the measurable sets to be all the sets $E$ satisfying above condition.

This implies the Lebesgue measure space $(\mathbb{R}^n, \mathscr{U}, m)$.
It is a complete measure space, i.e. null sets are measurable.

\begin{proposition}[Properties of measurable sets]
	Let $E$ be a measurable set, there exists $G_{\delta}$ set and  $F_{\sigma}$ set
	such that
	 \[
		 E = G_{\delta}\backslash Z_1 = F_{\delta} \cup Z_2
	.\]
	where $Z_1,Z_2$ are null sets.

	$E_k\nearrow E$  $\implies$  $ \lim_{k \to \infty}m(E_k) = m(E)$ and

	 \[
		 m\left(\liminf_{k\to \infty}E_k\right)\le \liminf_{k\to \infty}m(E_k)
	.\]
	(Fauto's Lemma)
\end{proposition}

\begin{definition}[Measurable function]
	Let $f$ be a map from measurable space $(X, \mathcal{A})$ to $(Y, \mathcal{B})$.
	We say $f$ is measurable if
	 \[
	\forall B\in \mathcal{B}, \quad f^{-1}(B)\in \mathcal{A}
	.\]
	
	In the case of real functions, this is equivalent to
	\[
		\forall t\in \mathbb{R}, \quad \{x; f(x)>t\} \text{ is a measurable set}
	.\]
\end{definition}

\begin{proposition}
    Let $f$ be a non-negative measurable function,
	$\exists \varphi_k \nearrow f$,
	where $\varphi_k$ are simple functions.

	For a general measurable function $f$, decompose it to $f=f_+-f_-$. 
\end{proposition}

\begin{theorem}[Egorov]
    Let $E$ be a measurable set and $m(E)<\infty$, $f_n\to f, a.e.$,
	Then  $\forall\varepsilon>0$, there exists a closed set $F_\varepsilon$ s.t.
	$m(E\backslash F_{\varepsilon})<\varepsilon$ and
	$f_n\to f$ uniformly on $F_{\varepsilon}$.
\end{theorem}

\begin{theorem}[Lusin]
    Let $E$ be a measurable set and  $m(E)<\infty$.
	Then $\forall \varepsilon>0, \exists F_{\varepsilon}$ such that
	$f \big|_{F_\varepsilon}$ is continuous.
\end{theorem}

Convergence patterns:
\begin{itemize}
	\item $f_n\to f, a.e.$
	\item  $f_n\to f, a.u.$
	\item  $f_n\xrightarrow{m} f$
\end{itemize}

\section{Lebesgue integrals}
\label{sec:Lebesgue integrals}
The critical question of this section:
If a sequence of functions $f_n$ converges to  $f$, when does their
integrals $\int f_n$ converge to  $\int f$? 

\subsection{Definition of Lebesgue integrals}
\label{sub:Definition of Lebesgue integrals}

\begin{itemize}
	\item Simple functions:
		$f=\sum_{k=1}^N a_k\chi_{E_k}$, define
		\[
			\int f = \sum_{k=1}^N a_km(E_k)
		.\]
	\item $f: E\to \mathbb{R}^n$, where $m(E)<\infty$, $f$ bounded.
		These functions form the set $\mathcal{L}_0$
		Then $\epsilon \varphi_k \to f$,  $\varphi_k$ simple, define
		 \[
		\int f = \lim_{k\to \infty}\int \varphi_k
        .\]
	\item Non-negative function:
		\[
			\int f = \sup \{\int g\mid 0\le g\le f, g\in \mathcal{L}_0\}
		.\]
	\item General functions:
		\[
		\int f = \int f_+ - \int f_-
		.\]
		Integrable $\iff \int f_+, \int f_- <\infty$.
\end{itemize}

Relations between Riemann integrals and Lebesgue integrals:
\begin{itemize}
	\item $f$ is integrable on $[a,b]$ iff
		$f$ bounded and the discontinuous points form a null set.
	\item If $f$ Riemann integrable on $[a,b]$, then
		two types of integral yield the same result.
\end{itemize}

\subsection{Dominatied convergence theorem}
\label{sub:Dominated convergence theorem}
\begin{theorem}
    Let $E$ be a bounded and measurable set, $f_n\to f, a.e.$ on $E$.
	Furthermore,  $f_n$ is uniformly bounded almost everywhere ($|f_n|<M,a.e.$)
	Then
	\[
	\int_E |f_n-f|\to 0 \implies \lim_{m\to \infty}\int_E f_n = \int_E f
	.\]
\end{theorem}
\begin{proof}[Proof]
    By Egorov's Theorem
\end{proof}

\begin{lemma}[Fatou's Lemma]
	 If $f_n\ge 0$, then
	 \[
		 \int \liminf_{n\to \infty}f_n\le \liminf_{n\to \infty}\int f_n
	.\]
\end{lemma}
\begin{proof}[Proof]
	For any $g\in \mathcal{L}_0$, $0\le g\le \liminf_{n\to \infty} f_n$,
	we need to prove $\inf g\le \liminf \int f_n$. 

	Let $g_k=\min\{f_k, g\}$, then  $g_k\in \mathcal{L}_0$.
    We'll prove $g_k\to g$:

	Assmue by contrsdiction that  $\exists\varepsilon_0>0, \exists x_0$ s.t.
	\[
		g(x_0) - g_{k'}(x_0) > \varepsilon_0
	.\]
	then $g(x_0) - f_{k'}(x_0) > \varepsilon_0$,
\end{proof}

\begin{theorem}[Beppo-Levi]
    $f_n \nearrow f$, non-negative,
	 \[
	\lim_{n\to \infty}\int f_n = \int f
	.\]
\end{theorem}
\begin{proof}[Proof]
    \[
    f_n\le f\implies \lim_{n \to \infty}\int f_n \le \int f
    .\]
    By Fatou's Lemma,
	\[
	\int \liminf_{n\to \infty} f_n \le \liminf_{n\to \infty}\int f_n
	.\]
	\[
		\implies \int f \le \liminf_{n\to \infty} \int f_n
	.\]
\end{proof}
\begin{corollary}
    Let $f_n$ be non-negative functions,
	\[
	\int \sum_{k=1}^{\infty} f_k(x) = \sum_{k=1}^{\infty} \int f_k(x)
	.\]
\end{corollary}

\begin{proposition}
    Let $f$ be an integrable function, $\forall\varepsilon>0$, we have:
	 \begin{itemize}
		\item Exists finite measurable set $B$ s.t.
			\[
				\int_{B^c}|f|<\varepsilon
			.\]
	    \item (ab continuity) $\exists\delta>0$ s.t.
			$\forall B$, $m(B)<\delta$, we have
			 \[
			 \int_B |f| < \varepsilon
			.\]
			This is equivalent to
			\[
				\lim_{m(B)\to 0}\int_B f = 0
			.\]
	\end{itemize}
\end{proposition}
\begin{proof}[Proof]
    ...
\end{proof}
