%!TEX root=./main.tex
\section{Introduction}
\label{sec:Introduction}
\begin{center}
	\sffamily\large\bfseries Teacher: Yang Shiwu
\end{center}

Grading: Homework-Midterm-Endterm: 30-30-40

Contents of this course: Real analysis

\subsection{Recap}
\label{sub:Recap}
\begin{definition}[Measurable space]
	Let $X$ be a set and $\mathcal{A}$ be a $\sigma$-algebra,
	we say  $(X, \mathcal{A})$ is a measurable space if
	\begin{itemize}
		\item $\emptyset \in \mathcal{A}$;
		\item If $A\in \mathcal{A}$, then $A^c\in \mathcal{A}$;
		\item If  $A_k\in \mathcal{A}$, then $\bigcup_{k=1}^{+\infty } \in \mathcal{A}$.
	\end{itemize}
\end{definition}

Outer measure $m^*$:
\begin{itemize}
	\item $m^*(A)\ge 0$;
	\item  $m^*(\bigcup_{k=1}^{\infty }A_k)\le \sum_{k=1}^{\infty} m^*(A_k)$;
	\item $m^*(A)\le m^*(B)$ when $A\subset B$.
\end{itemize}

Caratheodry condition:
\[
m^*(T) = m^*(T \cap E) + m^*(T \cap E^c),\quad \forall T \subset X
.\]
We define the measurable sets to be all the sets $E$ satisfying above condition.

This implies the Lebesgue measure space $(\mathbb{R}^n, \mathscr{U}, m)$.
It is a complete measure space, i.e. null sets are measurable.

\begin{proposition}
	\label{mfatou}
	
	Properties of measurable sets:
	\begin{itemize}
		\item Let $E$ be a measurable set, there exists 
	a $G_{\delta}$ set $G$ and a $F_{\sigma}$ set $F$
	such that
	 \[
		 E = G\backslash Z_1 = F \cup Z_2
	.\]
	where $Z_1,Z_2$ are null sets.
		\item (Fatou's Lemma)
	
	 Measurable sets $E_k\nearrow E$  $\implies$  $ \lim_{k \to \infty}m(E_k) = m(E)$
	 and
	 \[
		 m\left(\liminf_{k\to \infty}E_k\right)\le \liminf_{k\to \infty}m(E_k)
	.\]

	\end{itemize}
\end{proposition}

\begin{definition}[Measurable function]
	Let $f$ be a map from measurable space $(X, \mathcal{A})$ to $(Y, \mathcal{B})$.
	We say $f$ is measurable if
	 \[
	\forall B\in \mathcal{B}, \quad f^{-1}(B)\in \mathcal{A}
	.\]
	
	In the case of real functions, this is equivalent to
	\[
		\forall t\in \mathbb{R}, \quad \{x; f(x)>t\} \text{ is a measurable set}
	.\]
\end{definition}

\begin{proposition}
    Let $f$ be a non-negative measurable function,
	$\exists \varphi_k \nearrow f$,
	where $\varphi_k$ are simple functions.

	For a general measurable function $f$, decompose it to $f=f_+-f_-$. 
\end{proposition}

\begin{theorem}[Egorov]
    Let $E$ be a measurable set and $m(E)<\infty$, $f_n\to f, a.e.$,
	Then  $\forall\varepsilon>0$, there exists a closed set $F_\varepsilon$ s.t.
	$m(E\backslash F_{\varepsilon})<\varepsilon$ and
	$f_n\to f$ uniformly on $F_{\varepsilon}$.
\end{theorem}

\begin{theorem}[Lusin]
    Let $E$ be a measurable set and  $m(E)<\infty$.
	Then $\forall \varepsilon>0, \exists F_{\varepsilon}$ such that
	$f \big|_{F_\varepsilon}$ is continuous.
\end{theorem}

Convergence patterns:
\begin{itemize}
	\item Converge \vocab{almost everywhere}: $f_n\to f, a.e.$
	\item Converge \vocab{almost uniformly}: $f_n\to f, a.u.$
	\item Converge \vocab{in measure}: $f_n\xrightarrow{m} f$
\end{itemize}

\section{Lebesgue integrals}
\label{sec:Lebesgue integrals}

\subsection{Recap: Definition of Lebesgue integrals}
\label{sub:Definition of Lebesgue integrals}

\begin{itemize}
	\item Simple functions:
		$f=\sum_{k=1}^N a_k\chi_{E_k}$, define
		\[
			\int f = \sum_{k=1}^N a_km(E_k)
		.\]
	\item $f: E\to \mathbb{R}^n$, where $m(E)<\infty$, $f$ bounded.
		These functions form the set $\mathcal{L}_0$.
		Then $\exists \varphi_k \to f$,  $\varphi_k$ simple, define
		 \[
		\int f = \lim_{k\to \infty}\int \varphi_k
        .\]
	\item Non-negative function:
		\[
			\int f = \sup \left\{\int g\ \middle|\ 0\le g\le f, g\in \mathcal{L}_0\right\}
		.\]
	\item General functions:
		\[
		\int f = \int f_+ - \int f_-
		.\]
		Integrable $\iff \int f_+, \int f_- < +\infty$.
\end{itemize}

Relations between Riemann integrals and Lebesgue integrals:
\begin{itemize}
	\item $f$ is Riemann integrable on $[a,b]$ iff
		$f$ bounded and the discontinuous points form a null set.
	\item If $f$ is Riemann integrable on $[a,b]$, then
		two types of integral yield the same result.
\end{itemize}

\subsection{Dominated convergence theorem}
\label{sub:Dominated convergence theorem}
The critical question of this section:
If a sequence of functions $f_n$ converges to $f$ (almost everywhere),
when does their integrals $\int f_n$ converge to  $\int f$?

We'll discuss this issue under various conditions and reach the
famous Dominated Convergence Theorem.

\begin{theorem}
	\label{thm:bounded convergence}
    Let $E$ be a measurable set with finite measure.
    Measurable functions $f_n\to f, a.e.$ on $E$.
	Furthermore, $f_n$ is uniformly bounded almost everywhere ($|f_n|<M,a.e.$).
	Then we have
	\[
	\int_E |f_n-f|\to 0 \implies \lim_{m\to \infty}\int_E f_n = \int_E f
	.\]
\end{theorem}
\begin{proof}[Proof]
    By Egorov's Theorem, $\forall \varepsilon>0$,
    there exists $F_\varepsilon \subset E$ s.t.
    $f_n\to f$ uniformly on $F_\varepsilon$,
    and $m(E\backslash F_\varepsilon)<\varepsilon$.
    
    Hence
    \begin{align*}
    \int_E|f_n-f| &= \int_{F_\varepsilon} |f_n-f| +
    \int_{E\backslash F_\varepsilon} |f_n-f|\\
    &\le \varepsilon_0 m(E) + 2M \varepsilon,
    \end{align*}
	which proves the result.
\end{proof}

\begin{lemma}[Fatou's Lemma]
	\label{lem:fatou}
	If $f_n\ge 0$, then
	\[
		\int \liminf_{n\to \infty}f_n\le \liminf_{n\to \infty}\int f_n
	.\]
\end{lemma}
\begin{proof}[Proof]
	For any $g\in \mathcal{L}_0$,
	$0\le g\le \liminf_{n\to \infty} f_n$,
	we need to prove $\int g\le \liminf \int f_n$. 

	Let $g_k=\min\{f_k, g\}$, assmue $g$ is uniformly bounded
	so that $g_k\in \mathcal{L}_0$.
	
    We'll prove $g_k\to g$:
	Assmue by contradiction that $\exists\varepsilon_0>0, \exists x_0$ s.t.
	\[
		g(x_0) - g_{k'}(x_0) > \varepsilon_0
	.\]
	then $g(x_0) - f_{k'}(x_0) > \varepsilon_0$,
	which contradicts with $g\le \liminf_{n\to \infty} f_n$.
	
	Thus for sufficiently large $k$,
	$g_k(x)\le g(x)\le g_k(x) + \varepsilon_0$, $\implies g_k\to g$.
	
	Therefore by \autoref{thm:bounded convergence} (note $g_k\in \mathcal{L}_0$),
	\begin{align*}
		\int g &= \lim_{k \to \infty}\int g_k \\
		&\le \liminf_{k \to \infty} \int f_k,
	\end{align*}
	and we're done.
\end{proof}
\begin{remark}
	This is nearly indentical to the measure version of
	Fatou's Lemma (\autoref{mfatou}).
	It shows some similarities between measure and integrals.
\end{remark}

\begin{theorem}[Beppo-Levi]
	\label{thm:beppo-levi}
    If non-negative functions $f_n \nearrow f$, we have
	 \[
	\lim_{n\to \infty}\int f_n = \int f
	.\]
\end{theorem}
\begin{proof}[Proof]
    \[
    f_n\le f\implies \lim_{n \to \infty}\int f_n \le \int f
    .\]
    By Fatou's Lemma (\ref{lem:fatou}),
	\[
	\int \liminf_{n\to \infty} f_n \le \liminf_{n\to \infty}\int f_n
	= \lim_{n \to \infty}\int f_n,
	\]
	\[
		\implies \int f \le \lim_{n\to \infty} \int f_n.
	\]
	Combining the two inequalities we get the desired equality.
\end{proof}
\begin{corollary}
    Let $f_n$ be non-negative functions, then
	\[
	\int \sum_{k=1}^{\infty} f_k(x) = \sum_{k=1}^{\infty} \int f_k(x)
	.\]
\end{corollary}

\begin{proposition}
    Let $f$ be an integrable function, $\forall\varepsilon>0$, we have:
	 \begin{itemize}
		\item There exists a set $B$ with finite measure s.t.
			\[
				\int_{B^c}|f|<\varepsilon
			.\]
	    \item (\vocab{Absolute continuity} of integrals) $\exists\delta>0$ s.t.
			$\forall E$, if $m(E)<\delta$,
			 \[
			 \int_E |f| < \varepsilon
			.\]
			This is equivalent to
			\[
				\lim_{m(E)\to 0}\int_E |f| = 0
			.\]
	\end{itemize}
\end{proposition}
\begin{proof}[Proof]
    Let $f_N(x) = |f(x)|$ when $|x|\le N, |f(x)|\le N$, and $f_N(x)=0$ otherwise. 
	Then $f_N \nearrow |f|$, so by Beppo-Levi (\autoref{thm:beppo-levi}),
	we get
	\[
		\lim_{N\to \infty}\int f_N = \int |f|.
	\]
	Let $B = \{x\mid |x|\le N, |f(x)|\le N\}$, when $N$ gets sufficiently large,
	we must have $\int_{B^c}|f|<\varepsilon$.

	For the second part, when $N$ is sufficiently large we have
	$\int (|f|-f_N)<\frac{\varepsilon}{2}$, so
	\begin{align*}
		\int_{E}|f| &= \int_E f_N + \int_E(|f|-f_N)\\
		&\le N\cdot m(E) + \frac{\varepsilon}{2}.
	\end{align*}
	
	Let $\delta = \frac{\varepsilon}{2N}$ to finish.
\end{proof}
