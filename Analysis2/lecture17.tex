%! TeX root = ./main.tex
\begin{theorem}[Parametrization of manifolds]
    Let $\Omega$ be an open set in $ \mathbb{R}^{n}$,
	$f: \Omega\to \mathbb{R}^{n+p}$ is a smooth map.
	If $\forall x^*\in \Omega$, $\rank \dd f(x^*) = n$,
	then there exists an open set $U$, $x^*\in U$ s.t.
	$f(U) \subset \mathbb{R}^{n+p}$ is an $n$-dimensional manifold.
\end{theorem}
\begin{proof}[Proof]
    Let $x_i$ be a coordinate in $\mathbb{R}^{n+p}$.

	WLOG $(\pfr{f_i}{x_j})_{1\le i,j\le n}$ is non-degenerate,
	let $F = (f_1,\dots, f_n)$, $G = (f_{n+1},\dots, f_{n+p})$
	and apply inverse function theorem on $F$,
	there exists open neighborhoods $U\ni x, V\ni F(x)=:y$,
	s.t. $F: U\to V$ is a smooth homeomorphism.
	\begin{equation*}
		\begin{tikzcd}
			U \subset \Omega\dar["f"]\rar["F"] &V \subset \mathbb{R}^{n}\dlar["\phi"]
			\\\mathbb{R}^{n+p}
		\end{tikzcd}
	\end{equation*}

	So $f(x) = (F(x), G(x)) = (y, GF^{-1}(y))$.
	Let
	\[
		\phi: V\to \mathbb{R}^{n},\quad
		y\mapsto (y, GF^{-1}(y)).
	\]
	We can see that $\phi$ is a homeomorphism $V \to f(U)$.
	(Indeed it's a bijection)
	So by definition we know $f(U)$ is a manifold.
\end{proof}
\begin{example}
    Let
	\[
	\phi(\theta,r) = \left\{
	\begin{aligned}
		x &= \left(1 + r\cos\frac{\theta}{2}\right)\cos\theta\\
		y &= \left(1 + r\cos\frac{\theta}{2}\right)\sin\theta\\
		z &= r\sin\frac{\theta}{2}
	\end{aligned}\right., \quad
	I = [0, 2\pi] \times (-1, 1).
	\]
	Then $M = \phi(I)$ is a Mobius strip, which is a
	two dimensional smooth manifold in $\mathbb{R}^{3}$,
	as $\dd \phi$ has rank 2 everywhere.

	Besides, there doesn't exist a function $f: \mathbb{R}^{3}\to \mathbb{R}$
	s.t. $M = f^{-1}(0)$. Basically this is because $M$ is not
	orientable, but $\nabla f$ and $-\nabla f$ are ``normal'' directions
	of $M$, which makes it orientable.
	Below we give a sketch:
	\begin{proof}[Proof]
		Let $v(\theta) = \cos\frac{\theta}{2}e_2(\theta) -
		\sin\frac{\theta}{2}e_1(\theta)$, where
		$e_2(\theta) = (0,0,1), e_1(\theta) = (\cos\theta, \sin\theta, 0)$.

		Note that $e_1\perp e_2$, consider the curve $\beta: [0,2\pi]\to \mathbb{R}^{3}$
		\[
		\theta\mapsto (\cos\theta, \sin\theta, 0) + \varepsilon v(\theta).
		\]
		Let $\varepsilon$ be sufficiently small, when $\varepsilon\ne 0$ we
		can check $\beta$ and $M$ do not intersect.
		We can take $\varepsilon$ s.t. $f(\beta(0)) > 0$ as $\dd f\ne 0$.
		($\varepsilon$ can be negative)

		Since $\beta(0) = (1,0,\varepsilon), \beta(2\pi) = (1,0,-\varepsilon)$,
		when $f(\beta(0)) > 0$, we must have $f(\beta(2\pi)) < 0$.
		By continuity, $\exists \theta_0$ s.t. $f(\beta(\theta_0)) = 0$,
		which means $\beta(\theta_0) \in M$, contradiction!
	\end{proof}
\end{example}
