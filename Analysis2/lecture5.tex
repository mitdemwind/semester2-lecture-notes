%! TeX root = ./main.tex
% I missed this lecture, so this is the notes I copied from my teacher's handout.
\subsection{Applications of Fubini's theorem}
\label{sub:Applications of Fubini's theorem}

\begin{definition}[Product measure]
	Let $(X, \mathscr{F}, m)$ and $(Y, \mathscr{G}, m)$ be
	measure spaces, define a measure on $X\times Y$:
	The measure $m$ induces an outer measure on  $X\times Y$,
	and complete it to a normal measure by using
	Caratheodory conditions. This measure is called
	the \vocab{product measure} on $X\times Y$.
\end{definition}

\begin{theorem}
	Let $ \mathbb{R}^d = \mathbb{R}^{d_1}\times \mathbb{R}^{d_2}$,
	$E_1,E_2$ are subsets of $\mathbb{R}^{d_1}, \mathbb{R}^{d_2}$,
	respectively.
	\begin{itemize}
		\item If $ E_1,E_2$ are measurable, then $E$ is measurable as well,
			and $m(E) = m(E_1)m(E_2)$.
		\item If $E$ is measurable, then  $E_1,E_2$ are measurable,
			and $m(E)=m(E_1)m(E_2)$, unless one of $E_1,E_2$ is null set,
			which means $E$ is null as well.
	\end{itemize}
\end{theorem}
\begin{proof}[Proof]
    First it's easy to note that
	\[
	m^*(E)\le m^*(E_1)m^*(E_2).
	\]
	So we directly conclude that if one of $E_1,E_2$ is null set, $E$ must be null.

	Thus we may assume below that $E_1,E_2$ have finite nonzero measure.
	By taking the equimeasure hull of $E_1,E_2$ (denoted by $F_1,F_2$),
	let $Z_1=F_1 \backslash E_1, Z_2=F_2\backslash E_2$, we have
	\[
		(F_1\times F_2)\backslash(Z_1\times F_2\cup F_1\backslash Z_2)\subset
		E \subset F_1\times F_2,
	\]
	so $E$ is measurable.

	Conversely, if  $E$ is measurable, consider the measurable function $\chi_E$,
	by definition  $\chi_E = \chi_{E_1}\chi_{E_2}$, hence by Tonelli's theorem,
	for $x$ almost everywhere, $\chi_{E_1}(x)\chi_{E_2}$ is measurable
	on $\mathbb{R}^{d_2}$ $ \implies E_2$ is measurable.

	Therefore we have the equation
	\[
	m(E) = \int_{\mathbb{R}^d} \chi_E
	= \int_{\mathbb{R}^{d_1}} \left( \int_{\mathbb{R}^{d_2}} \chi_{E_1}\chi_{E_2} \right)
	= m(E_1)m(E_2).
	\]
	This proves the theorem.
\end{proof}
\begin{corollary}
    Let $f(x)$ be a measurable function on  $\mathbb{R}^{d_1}$,
	we have $g(x,y)=f(x)$ is measurable on  $\mathbb{R}^{d_2}$.
\end{corollary}
\begin{proof}[Proof]
    It's sufficient to prove that $\{(x,y)|f(x)>t\}$ is measurable in $\mathbb{R}^{d}$.
	This follows from the fact that
	\[
	\{(x,y)| f(x)>t\} = \{x|f(x)>t\}\times \mathbb{R}^{d_2},
	\]
	and the previous theorem.
\end{proof}

\begin{proposition}
	Let $L$ be a linear map $\mathbb{R}^{d}\to \mathbb{R}^{d}$,
	$E \subset \mathbb{R}^{d}$ a measurable set, then  $L(E)$ is measurable, and
	\[
	m(L(E)) = |\det L|m(E).
	\]
\end{proposition}
\begin{proof}[Proof]
    In fact we only need to prove it for cuboids $E$ and
	elementary linear transformation $L$.

	Now we only need to look at the case where
	$L = \begin{pmatrix}
		1 &c &\cdots &0\\
		0 &1 &\cdots &0\\
		\vdots &\vdots &\ddots &\vdots\\
		0 &0 &\cdots &1
	\end{pmatrix}$ since the other cases are trivial or similar to this case.

	Thus by Fubini's theorem, WLOG $E$ is the unit cube,
	\[
	m(L(E)) = \int \chi_{L(E)} = \int_{\mathbb{R}^{d-1}}
	\left( \int_{\mathbb{R}} \chi_{L(E)\dd x_1} \right)
	= \int_{\mathbb{R}^{d-1}} \chi_{E'}\cdot 1 = 1 = |\det L|m(E),
	\]
	where $E' = \{(x_2,\dots,x_n)| 0\le x_i\le 1\}$.
\end{proof}

From this transformation formula we deduce the integral version:

Let $f$ be an integrable function on  $\mathbb{R}^{d}$, then
$f(L(x))$ is also integrable, and
\[
\int f(L(x)) = \frac{1}{|\det L|}\int f(x).
\]
Here we require $L\in \GL(n)$, since if  $\det L = 0$, the function
 $f(L(x))$ need not be measurable.

At last we take a look at Fubini's theorem with the convolution product.
\begin{definition}[Convolution]
	Let $f, g$ be smooth functions with compact support, define
	their \vocab{convolution} to be
	\[
	f*g = \int f(x-y)g(y)\dd y.
	\]
	Then $f*g$ is also a smooth function with compact support.
\end{definition}

In fact we can generalize this definition for $f,g\in L^1$.

First note that $f(x-y),g(y)$ are measurable functions on $\mathbb{R}^{2d}$,
by Tonelli's theorem,
\[
\int_{\mathbb{R}^{2d}} |f(x-y)||g(y)|\dd x\dd y
=\int _{\mathbb{R}^{d}}\left( \int_{\mathbb{R}^{d}}|f(x-y)||g(y)|\dd x \right)\dd y
=\lVert f \rVert _{L^1} \lVert g \rVert _{L^1} < +\infty.
\]
This shows that $f(x-y)g(y)$ is integrable on $\mathbb{R}^{2d}$.
Hence by Fubini's theorem $f(x-y)g(y)$ is integrable as a function of $y$,
and $f*g$ is integrable on $\mathbb{R}^{d}$.

Moreover we have
\[
\lVert f*g \rVert _{L^1} \le \lVert f \rVert _{L^1} \lVert g \rVert _{L^1}.
\]
The equality holds when both $f$ and $g$ are non-negative.
