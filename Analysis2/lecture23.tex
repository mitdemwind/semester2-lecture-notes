%! TeX root = ./main.tex
\subsection{Stolkes' formula}
\label{sub:Stolkes' formula}
Intuitively, Stolkes' formula states that:
Let $D$ be a region, $\dd \omega$ be a differential form,
then
\[
\int_D \dd \omega = \int_{\partial D} \omega.
\]
Here $\partial D$ means the ``boundary'' of $D$.

Of course we need some ``regularity'' requirements of $D$ and $\omega$,
and it's the generalization of Newton-Lebniz formula into higher dimensions.

\begin{definition}[Bounded regions with boundary]
	Let $\Omega \subset \mathbb{R}^{n}$ be a compact set,
	we say it's a \vocab{bounded region with boundary} if
	$\forall x\in \partial \Omega$,
	there exists open sets $U, V \subset \mathbb{R}^n$,
	$x\in U$ and a continuous homeomorphism $\Phi: U\to V$,
	such that
	\[
	\Phi(U\cap \Omega) = V\cap \{x_n\ge 0\}, \quad
	\Phi(U\cap \partial\Omega) = V\cap \{x_n = 0\}.
	\]
	If $\Phi$ is also $C^1$, we say $x\in \partial\Omega$ is a \vocab{regular point},
	otherwise a \vocab{singular point}.
\end{definition}

\begin{lemma}
	Let $\Omega$ be a bounded region with boundary,
	for all regular $p\in \partial\Omega$, there exists a unique unit vector
	$\nu(p) \in \mathbb{R}^{n}$, and $\varepsilon > 0$, s.t.
	\[
	\nu(p) \perp T_p\partial\Omega, |\nu(p)| = 1,
	p - t\nu(p) \in \Omega, \forall 0<t<\varepsilon.
	\]
	We call $\nu(p)$ the \vocab{outward unit normal vector} of $p$.
\end{lemma}
\begin{proof}[Proof]
    By the definition of regular points, we may assume that:
	\[
	\Omega\cap V = \{x\in V\mid f(x)\ge 0\}, \quad
	\partial\Omega \cap V = \{x\in V\mid f(x) = 0\}.
	\]
	Where $f$ is a $C^1$ function.

	Since $\nabla f$ is nonzero,
	the tangent space $T_p\partial\Omega = \{v\mid v\cdot \nabla f = 0\}$.

	Let $\nu(p) = - \frac{\nabla f}{|\nabla f|}$, then it's obvious $\nu(p)$ points
	outside of $\Omega$.
\end{proof}

Now for a cuboid $I$ and a $C^1$ function $\phi$,
 \begin{align*}
\int_I \frac{\dd \phi}{\dd x_n} \dd x
&= \int_{I_{n-1}} \phi(x_1, \dots, x_{n-1}, b_n)\dd x_1\cdots \dd x_{n-1}
- \int_{I_{n-1}} \phi(x_1, \dots, x_{n-1}, a_n)\dd x_1\cdots \dd x_{n-1}\\
&= \int_{\partial I} \phi \cdot \nu_n \dd \sigma.
\end{align*}
Where $\sigma$ is the measure on the boundary, $\nu$ is the outward unit
normal vector.

\begin{lemma}
	Let $K$ be a compact set in $\mathbb{R}^{n}$, $U \supset K$ is open,
	there exists a smooth function $f$ such that
	$\supp f \subset U$, and $f\big|_K > 0$.
\end{lemma}
\begin{proof}[Proof]
    Let $\rho(x)$ be a smooth function s.t. $\rho(x) = 1$ for $|x|\le 1$ and
	$\rho(x) = 0$ for $|x|\ge 2$.
	Let
	\[
	g(x) = \int_{|y|\le 2} f(x - \delta y)\rho(y)\dd y.
	\]
	Then $g$ is a smooth non-negative function.
\end{proof}

\begin{theorem}[Unit decomposition on compact sets]
	Let $K$ be a compact set, $\{U_1, \dots, U_k\}$ is an open covering of $K$.
	There exists smooth functions $f_1,\dots, f_k$ s.t.
	\[
	1 = f_1(x) + f_2(x) + \dots + f_k(x),\quad \supp f_i(x) \subset U_i.
	\]
\end{theorem}
\begin{proof}[Proof]
    For $1 \le i\le k$, $\delta > 0$, define
	\[
		K_i^\delta = \{x\in U_i \mid d(x, U_i^c) > \delta\}.
	\]
	Note that $\{\bigcup_{i=1}^k K_i ^{\frac{1}{m}}\}_{m = 1}^{\infty}$
	is also an open covering of $K$, thus there exists $N$ s.t.
	\[
	K \subset \bigcup_{i=1}^k K_i^{\frac{1}{N}}.
	\]
	Hence by lemma we have $g_i$ s.t. $\supp g_i \subset U_i$ and
	$g_i > 0$ on the closure of $K_i^{\frac{1}{N}}$.
	Similarly we have a smooth function $g$ s.t. $g(x) = 0$ on $K$,
	and $g > 0$ outside of the closure of $\bigcup_{i=1}^k K_i^{\frac{1}{N}}$.

	Let $G(x) = g_1(x)+\cdots+g_k(x)+g(x) > 0$ on $\bigcup_{i=1}^k U_i$,
	then we can define $f_i(x) = \frac{g_i(x)}{G(x)}$ which satisfy the condition.
\end{proof}

\begin{theorem}
	Let $\Phi$ be a $C^1$ homeomorphism from a cuboid $I$ to $\Omega$,
	then $\Omega$ satisfies Stolkes' formula:

	$\forall \phi\in C^1(\mathring{D}) \cap C(\overline{D})$, we have
	\[
	\int_D \nabla \phi \dd x = \int_{\partial D}\phi \nu \dd \sigma.
	\]
\end{theorem}
