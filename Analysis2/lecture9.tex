%! TeX root = ./main.tex
\paragraph{Step 2}

\begin{proposition}
	The jump function $J(x)$ is differentiable almost everywhere, and $J'(x)=0, a.e.$.
\end{proposition}
\begin{proof}[Proof]
    The Dini derivatives of $J(x)$ exist and are non-negative (since $J$ is increasing).
	\[
	\overline{D}(J)(x) = \max\{D^+(J)(x), D^-(J)(x)\}.
	\]
	Let $E_\varepsilon = \{\overline{D}(J)(x) > \varepsilon > 0\}$.
	We'll prove $E_\varepsilon$ is null for all $\varepsilon$.
	If $x\in E_\varepsilon$,  $\exists h$ s.t.
	\[
		\frac{J(x+h) - J(x)}{h}>\varepsilon \implies J(x+h)-J(x-h) > \varepsilon h.
	\]

	Let $N\in \mathbb{N}$ s.t. $\sum_{n>N}\alpha_n < \frac{\varepsilon\delta}{10}$.
	Define $J_N(x) = \sum_{n>N} j_n(x)$.
	\[
	E_{\varepsilon, N} = \{\overline{D}(J_N)(x) > \varepsilon\},\quad
	E_\varepsilon \subset E_{\varepsilon, N}\cup \{x_1,\dots,x_N\},
	\]

	Since for $x\ne x_i$,
	\begin{align*}
		\overline{D}(J)(x) &= \limsup_{h\to 0} \frac{J(x+h)-J(x)}{h}\\
		&= \limsup_{h\to 0} \left( \frac{J_N(x+h)-J_N(x)}{h} + \frac{1}{h}
		\sum_{n=1}^{N} (j_n(x+h) - j_n(x))\right)
		= \overline{D}(J_N)(x).
	\end{align*}

	Next we need to control the measure of $E_{\varepsilon,N}$.

	For all $y\in E_{\varepsilon,N}$, there exists sufficiently
	small $h$ s.t. $J_N(y+h)-J_N(y)>h\varepsilon$.
	So the intervals $(y-h, y+h)$ is a covering of $E_{\varepsilon,N}$,
	and it can be controled using the value of $J_N$.
	Therefore we hope to find some \textit{disjoint} intervals which
	cover certain ratio of $E_{\varepsilon,N}$.

	\begin{lemma}
		Let $\mathcal{B}$ be a collection of balls with bounded radius in $\mathbb{R}^{d}$.
		There exists countably many disjoint balls $B_i$ s.t.
		\[
		\bigcup_{B\in \mathcal{B}} B \subset\bigcup_{i=1}^\infty 5B_i.
		\]
	\end{lemma}
	\begin{proof}[Proof]
		Let $r(B)$ denote the radius of $B$.
		Take $B_1$ s.t. $r(B_1)>\frac{1}{2} \sup_{b\in \mathcal{B}}r(B)$.

		The rest is the same as before.
	\end{proof}

	By lemma, there exists countably many disjoint intervals
	$(x_i+h_i, x_i-h_i)$ s.t.
	\begin{align*}
	m^*(E_{\varepsilon,N})&\le 5 \sum_{i=1}^{\infty}2h_i\\
	&\le 10 \sum_{i=1}^{\infty} \varepsilon^{-1}(J_N(x_i+h_i)-J_N(x_i-h_i))\\
	&\le 10 \varepsilon^{-1} (J_N(b)-J_N(a)) < \delta.
	\end{align*}

	Hence $m^*(E_\varepsilon)\le m^*(E_{\varepsilon, N})
	<\delta \implies m^*(E_\varepsilon) = 0$,
	which gives $\overline{D}(J) = 0, a.e.$.
\end{proof}

\paragraph{Step 3}
First we prove $D^+(F)<\infty, a.e.$.

Let $E_\gamma = \{x: D^+(F)(x)>\gamma\}$.

When $h\in [\frac{1}{n+1}, \frac{1}{n}]$ :
\begin{align*}
	\frac{F(x+h) - F(x)}{h}\le \frac{n+1}{n} \frac{F(x+\frac{1}{n})-F(x)}{\frac{1}{n}},\\
	\ge \frac{n}{n+1}\frac{F(x+\frac{1}{n+1})-F(x)}{\frac{1}{n+1}}.
\end{align*}

Thus
\[
D^+(F)(x) = \limsup_{h\to 0}\frac{F(x+h)-F(x)}{h}
=\limsup_{n\to \infty} \frac{F(x+\frac{1}{n})-F(x)}{\frac{1}{n}}
\]
is measurable.

\begin{lemma}[Riesz sunrise lemma]
	\label{lem:sunrise}
	Let $G(x)$ be a continuous function on $\mathbb{R}$. Define
	\[
	E = \{x: \exists h>0, s.t.\ G(x+h)>G(x)\}.
	\]
	Then $E$ is open and $E = \bigcup_{i=1}^\infty (a_i, b_i)$,
	where $(a_i,b_i)$ are disjoint finite intervals s.t. $G(a_i) = G(b_i)$.

	When $G$ is defined on finite interval $[a,b]$, we also have $G(a)\le G(b_1)$.
\end{lemma}
\begin{proof}[Proof]
    Note that $E$ is open since $G$ is continuous.

	Take a maximum open interval $(a,b) \subset E$,
	i.e. $a,b\notin E$, so $G(a)\ge G(b)$.

	Since $b\notin E, G(x)\le G(b), \forall x>b$.
	If $G(a)>G(b)$,
	Let  $G(a+\varepsilon) > G(b)$, as $a+\varepsilon\in E$,
	exists $h>0$ s.t. $G(a+\varepsilon+h)>G(a+\varepsilon)$.

	But $G$ has a maximum on $[a+\varepsilon, b]$, say $G(c)$,
	we must have $c\ne a+\varepsilon, b$.
	This leads to a contradiction.
\end{proof}
\begin{remark}
    This lemma provides a better estimation than
	previous covering lemmas, since it directly
	claims that $E$ can be broken into disjoint
	intervals.
\end{remark}

For $x\in E_\gamma$,  $\exists h>0$ s.t. $F(x+h)-F(x) > \gamma h$,
by \autoref{lem:sunrise} on  $F(x)-\gamma x$,
\[
	m(E_\gamma) \le \sum_{k=1}^{\infty} (b_k - a_k)
	\le \gamma^{-1}\sum_{k=1}^{\infty} (F(b_k)-F(a_k))\le \gamma^{-1}(F(b)-F(a)).
\]
Therefore when $\gamma\to \infty$, $m(E_\gamma)\to 0$.

The last part is $D^+(F)\le D_-(F),a.e.$.

Similarly it's sufficient to prove the following set is null
for all rational numbers $r<R$:
 \[
E_{r, R} = \{D^+(F)(x)>R, D_-(F)(x)<r\}.
\]

Since $D^+(F)$ is measurable, $E_{r,R}$ is measurable.
If $m(E_{r,R})>0$, we can restrict it to
a smaller interval $[c,d] \subset [a,b]$ such that
$d-c<\frac{R}{r}m(E_{r,R})$.

Let $G(x) = F(-x) + rx$, by \autoref{lem:sunrise} on $[-d,-c]$,
 \[
\{s: \exists h>0, G(x+h)>G(x)\} = \bigcup_k (-b_k, -a_k).
\]
Note that $-E_{r,R}$ is contained in the above set,
and $G(-b_k)\le G(-a_k)\iff F(b_k) - F(a_k)\le r(b_k - a_k)$,

We use \autoref{lem:sunrise} again on each $(a_k, b_k)$ and $F(x)-Rx$,
 \[
E_{r,R}\cap (a_k,b_k) \subset \{x: \exists h>0, F(x+h)-F(x)\ge Rh\}
= \bigcup_{l=1}^\infty (a_{k,l}, b_{k,l}).
\]

Hence
\begin{align*}
	m(E_{r,R}) &\le \sum_{k,l=1}^{\infty}(b_{k,l}-a_{k,l})\\
	&\le R^{-1} \sum_{k,l=1}^{\infty}(F(b_{k,l}) - F(a_{k,l}))
	\le R^{-1} \sum_{k=1}^{\infty}(F(b_k) - F(a_k))\\
	&\le R^{-1} r \sum_{k=1}^{\infty}(b_k - a_k) \le R^{-1}r (d-c),
\end{align*}
which gives a contradiction!
So $m(E_{r,R}) = 0$ for all rationals $r<R$.
Therefore we're done by
\[
m(\{D^+(F)>D_-(F)\})\le \sum_{r,R} m(E_{r,R}) = 0
\]

