%! TeX root = ./main.tex
Returning to the proof of \autoref{Lebesgue differentiation thm p1},
we can assmue $g$ is continuous with compact support,
 \[
\frac{1}{m(B)} \int_B |f(y)-g(y)|\dd y \le M(f-g)(x)
\]
Hence by taking $B$ sufficiently small s.t. $|g(y)-g(x)|\le \varepsilon_0$
for all $x,y\in B$,
\[
\frac{1}{m(B)}\int _B f(y)\dd y \ge 3\varepsilon_0
\]
\[
\impliedby |f(x)-g(x)| + M(f-g)(x) \ge 2\varepsilon_0.
\]
But
\[
m\{|f(x)-g(x)|\ge \varepsilon_0\} + m\{M(f-g)>\varepsilon_0\}
\le \frac{\lVert f-g \rVert _{\mathcal{L}^1}}{\varepsilon_0} +
\frac{3^d}{\varepsilon_0}\lVert f-g \rVert _{\mathcal{L}^1}
\le \frac{3^d+1}{\varepsilon_0}\varepsilon.
\]
This completes the proof.

\begin{definition}[Lebesgue points]
	Let $|f(x)|<\infty$, $f$ is \textit{locally integrable}. If $x$ satisfies
	\[
	\lim_{|B|\to 0, B\ni x}\frac{1}{|B|} \int_B |f(y)-f(x)|\dd y = 0,
	\]
	we say $x$ is a \vocab{Lebesgue point} of $f$.
\end{definition}
\begin{remark}
    Here ``locally integrable'' means for all bounded measurable sets $E$,
	$f\chi_E\in \mathcal{L}^1$.
	This is denoted by $f\in \mathcal{L}_{loc}^1$.
\end{remark}

Let $E$ be a measurable set,  $\chi_E$ locally integrable,
If point $x$ is called a \vocab{density point} of $E$
if it's a Lebesgue point of  $\chi_E$.

\begin{theorem}
    Let $E$ be a measurable set, then almost all the points in $E$
	are density points of $E$, almost all the points outside of $E$
	are not density points of $E$.
\end{theorem}
\begin{proof}[Proof]
    This is a direct corollary of \autoref{Lebesgue differentiation thm p1}.
\end{proof}

The differentiation theorem has some applications in convolution:
\begin{align*}
    \frac{1}{|B|}\int _B f(y)\dd y
	&= c_d^{-1} \varepsilon^{-d}\int_{B(x,\varepsilon)} f(y)\dd y\\
	&= \int f(x-y) \cdot c_d^{-1}\varepsilon^{-d}\chi_{B(0,\varepsilon)}(y)\dd y\\
	&= f*K_\varepsilon.
\end{align*}
where $K_\varepsilon=c_d^{-1} \varepsilon^{-d}\chi_{B(0,\varepsilon)}$,
$c_d$ is the measure of a unit sphere in  $\mathbb{R}^d$.

By differentiation theorem, $\lim_{\varepsilon\to 0}f*K_\varepsilon = f(x), a.e.$.
In the homework we proved that there doesn't exist a function $I$ s.t.
$f*I=f$ for all $f\in \mathcal{L}^1$, but the functions $K_\varepsilon$
is approximating this ``convolution identity''.

\begin{definition}
	In general, if $\int K_\varepsilon =1$,
	$|K_\varepsilon|\le A\min\{\varepsilon^{-d},\varepsilon |x|^{-d-1}\}$,
	we say $K_\varepsilon$ is an \vocab{approximation to the identity}.
\end{definition}

``convolution kernel''

Let $\varphi$ be a smooth function whose support is in  $\{|x|\le 1\}$,
and $\int \varphi = 1$.
The function  $K_\varepsilon := \varepsilon^{-d}\varphi(\varepsilon^{-1}x)$
is called the Friedrichs smoothing kernel.

\begin{theorem}
    If $K_\varepsilon$ is an approximation to the identity, $f$ integrable,
	\[
	\lim_{\varepsilon \to 0} \lVert f*K_\varepsilon - f \rVert _{\mathcal{L}^1} = 0.
	\]
\end{theorem}
\begin{proof}[Proof]
    \begin{align*}
        |(f*K_\varepsilon)(x) - f(x)|
		&= \left|\int f(x-y)K_\varepsilon(y)\dd y - f(x)\right|\\
		&\le \int |f(x-y)-f(x)| |K_\varepsilon(y)|\dd y\\
		&\le \int _{|y|\le R} |f(x-y)-f(x)| A\varepsilon^{-d}\dd y
		+ \int _{|y|>R} |f(x-y)-f(x)| \varepsilon |y|^{-d-1}\dd y.
    \end{align*}

	Taking the integral over $\mathbb{R}^d$ :

	\begin{align*}
		&\le A\varepsilon^{-d} \int \int_{|y|\le R} |f(x-y)-f(x)|\dd y \dd x
		+\varepsilon\int \int_{|y|>R} |f(x-y)-f(x)||y|^{-d-1}\dd y\dd x\\
		&\le A\varepsilon^{-d} \int\int_{|y|\le R}|\tau_{-y}f(x)-f(x)|\dd y\dd x
		+ \varepsilon \int_{|y|>R}|y|^{-d-1}\int |\tau_{-y}f(x)|+|f(x)|\dd x\dd y\\
		&\le A\varepsilon^{-d}\int_{|y|\le R}\lVert \tau_{-y}f-f \rVert _{\mathcal{L}^1}
		\dd y + \varepsilon \int_{|y|>R} |y|^{-d-1} 2\lVert f \rVert _{\mathcal{L}^1}\dd y.
	\end{align*}

	By the continuity of translation, let $R$ sufficiently small we have
	\[
	\lVert \tau_{-y}f-f \rVert _{\mathcal{L}^1} < \varepsilon_0, \forall|y|<R.
	\]
\end{proof}

\begin{theorem}
    Let $K_\varepsilon$ be an approximation to the identity, $f$ integrable,
	\[
	\lim_{\varepsilon\to 0} f*K_\varepsilon = f(x)
	\]
	holds for Lebesgue points $x$ of $f$.
\end{theorem}
\begin{proof}[Proof]
    WLOG $x=0$, let
	\[
	\omega(r) = \frac{1}{c_d r^d}\int_{B(0,r)} |f(y)-f(0)| \dd y,
	\]
	we have  $\lim_{r\to 0}\omega(r) = 0$, and $\omega$ is continuous.
\end{proof}
