%! TeX root = ./main.tex
\begin{proposition}
	Let $\Omega \subset \mathbb{R}^{n}$, and $f: \Omega \to \mathbb{R}^{m}$ is
	a smooth map. Let $S \subset \mathbb{R}^{m}$ be a differential manifold,
	if for all $x\in f^{-1}(S)$, we have $\rank\dd f(x) = m$, then $f^{-1}(S)$ is
	a differential manifold with codimension same as $S$.
\end{proposition}
\begin{proof}[Proof]
    For any $x\in S$, let $\Phi$ be the homeomorphism
	from an open neighborhood of $x$ to $\mathbb{R}^{m}$.

	Suppose $\dim S = d$,
	let
	\[
	F(x) = ((\Phi \circ f)_{d+1}, \dots, (\Phi\circ f)_{m}).
	\]
	Note that $\dd (\Phi \circ f)$ is an $m\times n$ matrix,
	and its rank is $m$. Since
	\[
	\dd (\Phi\circ f) = \begin{pmatrix}
		\dd(\Phi\circ f)_1\\\vdots\\\dd(\Phi\circ f)_d\\\dd F
	\end{pmatrix}
	\]
	Thus $\dd F$ is a $(m-d)\times n$ matrix with rank $m - d$.
	So $F^{-1}(0) = f^{-1}(S)$ is a manifold with dimension $n - (m - d)$.
\end{proof}

\subsection{Tangent space}
\label{sub:Tangent space}
Since the differentiation relies on the choice of coordinates, if we
want to study the manifold more geometrically, we should look at the tangent lines
or tangent planes of the manifold.
\begin{definition}[Tangent vectors]
	Let $M$ be a differential manifold. Let $p\in M$,
	for all parametrized curve $\gamma: (-\varepsilon, \varepsilon)\to M$
	with $\gamma(0) = p$, we say the vector $\gamma'(0)\in \mathbb{R}^{n}$ is
	the \vocab{tangent vector} of $\gamma$ at point $p$.

	Let $T_pM$ denote the \vocab{tangent space} at $p$, which is defined as
	\[
	T_pM = \{\gamma'(0)\in \mathbb{R}^{n} \mid \gamma(0) = p\}.
	\]
\end{definition}
It's clear that $T_pM$ should be a vector space of dimension $\dim M$,
next we'll prove this fact.

\begin{proposition}[Push forward of tangent spaces under differential homeomorphism]
	Let $\Phi: U\to V$ be a differential homeomorphism, $M \subset U$ be
	a manifold, then
	\[
		T_{\Phi(p)}\Phi(M) = (\dd \Phi)\big|_{p} \cdot T_pM.
	\]
\end{proposition}
\begin{proof}[Proof]
    Let $\gamma$ be a parametrized curve on $M$ with $\gamma(0) = p$.
	Note that $\Phi\circ \gamma$ is a curve on  $\Phi(M)$ passing through $\Phi(p)$.
	Since
	\[
	\frac{\dd}{\dd t} \Phi\circ \gamma(t)\Big|_{t=0} = \dd \Phi(p) \cdot \gamma'(0).
	\]
	Thus $\dd \Phi(p)\cdot T_p M \subset T_{\Phi(p)}\Phi(M)$.

	Now we do the same thing for $\Phi^{-1}$, we can get the desired equality.
\end{proof}

Now we can easily calculate the tangent space:
since $M$ is locally homeomorphic to $\mathbb{R}^{d}$, and obviously
$T_{\Phi(p)}(\mathbb{R}^{d}\times \{0\}) = \mathbb{R}^{d} \times \{0\}$,
by above proposition,
$T_p M = (\dd \Phi)\cdot T_{\Phi(p)}(\mathbb{R}^{d} \times \{0\})$ is
a vector space of dimension $d$.

\begin{theorem}
    Let $M$ be a manifold, $T_p M$ is a vector space of dimension $\dim M$.
\end{theorem}
\begin{proposition}
	Let $f: \mathbb{R}^{n+d}\to \mathbb{R}^{n}$ be a smooth map,
	$\rank \dd f = n$. Let $M = f^{-1}(f(p))$, then $T_pM = \ker \dd f(p)$.
\end{proposition}
\begin{proof}[Proof]
    Let
	\[
		F(x, y) = (x, f(x, y)), x\in \mathbb{R}^{d}, y\in \mathbb{R}^{n}.
	\]
	$F$ is a homeomorphism, so $T_pM = (\dd F^{-1}) T_{F(p)}F(M)$.

	Note that $F(M) = \{(x, p)\mid \exists y, f(x, y) = f(p)\}$, it must
	be a vector space of dimension $d$,
	so $T_{F(p)} F(M) = \mathbb{R}^{d}\times \{0\}$,
	\[
	T_p M = (\dd F^{-1}) T_{F(p)}F(M) = \ker \dd f(p).
	\]
\end{proof}
\begin{example}
    Let $M$ be a manifold determined by $f: \mathbb{R}^{n}\to \mathbb{R}$,
	\[
	T_p M = \ker \dd f = \{v\in \mathbb{R}^{n}\mid \dd f(p) v = 0\}.
	\]
	Here $\dd f = (\pfr{f}{x_1}, \dots, \pfr{f}{x_n}) = \nabla f$.
	So $v\in T_p M \iff \nabla f \cdot v = 0$, the dot means the inner product.
	In this case the vector $\nabla f$ is called \vocab{normal direction vector}.
\end{example}

\subsection{Smooth maps between manifolds}
\label{sub:Smooth maps between manifolds}
Next we'll briefly introduce the differential maps between manifolds.
Since the differentiation on manifolds is hard to define, so what we
do is actually regarding manifolds as $\mathbb{R}^{d}$ locally and
define the differentiablity using the maps between Eucild spaces.
\begin{definition}
	Let $M, N$ be manifolds in $\mathbb{R}^{m}, \mathbb{R}^{n}$, respectively.
	$f: M\to N$ is a map, if $\forall p\in M$,
	there exists $p\in U \subset \mathbb{R}^m,
	V \subset \mathbb{R}^d$, $\Phi: U\to V$ s.t.
	\[
	f_\Phi = f\circ \Phi^{-1}
	\]
	is a smooth map from $V$ to $N$.
	We say $f$ is a smooth map from $M$ to $N$.
\end{definition}
We need to check this definition is well-defined:
if there's another homeomorphism $\Phi'$,
$f \circ \Phi' = (f\circ \Phi^{-1})\circ(\Phi \circ \Phi'^{-1})$ is indeed
a smooth map.

\begin{lemma}[Smooth maps are locally restrictions of smooth maps in Eucild spaces]
	Let $f: M\to N$ be a map,
	then $f$ is smooth $\iff$ $\forall p\in M,
	\exists p\in U \subset \mathbb{R}^{m}$ and
	a smooth map $F: U\to \mathbb{R}^{n}$ s.t.
	\[
	f\big|_{U\cap M} = F\big|_{U\cap M}.
	\]
\end{lemma}
\begin{proof}[Proof]
    Let $\tau$ denote the embedding from $M\cap U$ to $U$.
	Since $f\circ \Phi^{-1} = F\circ \tau\circ \Phi^{-1}$,
	so $F$ smooth $ \implies f_\Phi$ smooth $ \implies f$ smooth.
	\begin{equation*}
	\begin{tikzcd}
		V \subset \mathbb{R}^{d}\drar["f\circ \Phi^{-1}"']
		& M\cap U\rar["\tau"]\lar["\Phi"']\dar["f"]
								& U\dlar["F"]\\
		& N \subset \mathbb{R}^{n}
	\end{tikzcd}
	\end{equation*}
	TODO: fix this

	On the other hand,
	let $\wt{\tau}$ be the projection from $U$ to $V$,
	then $F = f\circ \Phi^{-1}\circ \wt{\tau}\circ \Phi$ satisfies
	the desired condition.
\end{proof}

\begin{example}
    Let $A$ be an orthogonal map in $\mathbb{R}^{3}$,
	then $A$ can be restricted to $S^2 \to S^2$.
\end{example}

