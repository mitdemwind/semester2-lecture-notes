%! TeX root = ./main.tex
Now we take a look at what we get so far:
\begin{itemize}
	\item If bounded functions $f_n\in \mathcal{L}_0$, $f_n\to f$,
		then $\int f_n\to \int f$.
	\item If $f_n$ is non-negative, then $\int \liminf f_n \le \liminf \int f_n$. (Fatou)

		This corresponds to: $m(\liminf E_n)\le \liminf m(E_n)$.
	\item If $f_n \nearrow f$, then $\int f_n \nearrow \int f$. (Beppo-Levi)

		This corresponds to: $E_n \subset E_{n+1} \implies m(\bigcup E_n) = \lim m(E_n)$.
\end{itemize}

Finally we come to the famous Lebesgue dominated convergence theorem:
\begin{theorem}[Lebesgue Dominated Convergence Theorem]
	\label{thm:dominated}
    Functions $f_n \to f, a.e.$, if there exists a function  $g$ s.t.
	$|f_n|\le g, a.e.$, then we have:
	 \[
	\int |f-f_n| \to 0. \left(\lim_{n\to \infty} \int f_n = \int f\right)
    \]
\end{theorem}
\begin{proof}[Proof]
    By Fatou's lemma (\ref{lem:fatou}), $2g - |f_n-f|$ is non-negative,
	 \[
	\int \liminf (2g-|f_n-f|) \le \liminf \int (2g - |f_n-f|)
	\]
	\[
	\implies 0 \le \liminf \left(- \int |f_n - f|\right)
	\]
	$\implies \limsup \int |f_n - f| \le 0$, hence it must equal to $0$.
\end{proof}
\begin{example}
    Non-examples of lebesgue dominated convergence theorem:
	\begin{itemize}
		\item Let $f_n = \chi_{[n, n+1]}$, $g = 1$, note that $g$ is not
			integrable, so $\int f_n = 1$ while  $f_n \to 0$.
		\item $f_n = \frac{1}{n}\chi_{[0,n]}$, $f_n\to 0$,  $\int f_n =1\not\to 0$.
			Since  $g(x)= \min\{\frac{1}{x}, 1\}$, which isn't integrable.
		\item $f_n = n\chi_{(0, \frac{1}{n})}$, $f_n \to 0$,  $\int f_n =1\not\to 0$.
			Here  $g(x) = \frac{1}{x}\chi_{[0,1]}$ is not integrable.
	\end{itemize}
\end{example}
\begin{example}
    Suppose that
	\[
	\lim_{n\to \infty}\int_E f_n = \int_E f
	\]
	holds for any measurable set $E$. Then
	\[
	\liminf_{n\to \infty} f_n \le f\le \limsup_{n\to \infty} f_n, a.e..
	\]
\end{example}
\begin{proof}[Proof]
	We only need to prove the case when $f=0$.

	For  $\forall\varepsilon>0$, define
	\[
	E_n^{\varepsilon} = \{x: f_n(x)< -\varepsilon\}.
	\]
	Note that
	\[
	\liminf E_n^{\varepsilon} \subset \{x: \limsup f_n\le -\varepsilon\}
	\subset \liminf E_n^{\frac{\varepsilon}{2}}.
	\]
	Because when $\limsup f_n(x)\le -\varepsilon$,  $\exists N$ such that
	$\sup_{n>N} f_n(x) < -\frac{\varepsilon}{2}$
	\[
		\implies f_n(x)<-\frac{\varepsilon}{2}, \forall n>N
	\]

	This implies $x\in E_n^{\frac{\varepsilon}{2}}, \forall n>N$,
	so $x\in\liminf E_n^{\frac{\varepsilon}{2}}$.

	We proceed with the proof,
	by using the condition ($E = \bigcap_{k\ge N}E_k^{\varepsilon}$),
    \[
	0 = \lim_{n\to \infty} \int_{\bigcap_{k\ge N}E_k^{\varepsilon}} f_n.
	\]
	Since $x\in \bigcap_{k\ge N} E_k^{\varepsilon}\implies f_k(x)<-\varepsilon$,
	we deduce
	\[
	0 = \lim_{n\to \infty} \int_{\bigcap_{k\ge N}E_k^{\varepsilon}} f_n
	\le (-\varepsilon)\cdot m\left(\bigcap_{k\ge N}E_k^{\varepsilon}\right)
	\]
	Hence $E = \bigcap_{k\ge N}E_k^{\varepsilon}$ is a null set.
\end{proof}

\subsection{Integrable function space $\mathcal{L}^1(E)$}
\label{sub:Integrable function space L1}

\begin{definition}[$\mathcal{L}^1$ space]
	Denoted by $\mathcal{L}^1(E)$ the space consisting of
	all the integrable functions on $E$.

	If $f=g, a.e.$, then  $\int |f-g| = 0$, we regard them as equivalent elements
	in  $\mathcal{L}^1(E)$.

	Observe that $\mathcal{L}^1(E)$ is a vector space, define the norm:
	\[
	\lVert f \rVert = \int_E |f|.
	\]
	It's easy to check that $\mathcal{L}^1(E)$ becomes a normal vector space.

	Moreover, it's also a \vocab{Banach space} (complete normal vector space).
\end{definition}
\begin{theorem}
    $\mathcal{L}^1(E)$ is a Banach space.
\end{theorem}
\begin{proof}[Proof]
    Let $f_n$ be a Cauchy sequence in  $\mathcal{L}^1(E)$,
	suppose $\lVert f_{n_k} - f_{n_{k+1}} \rVert < 2^{-k}$.

	Let $f = \sum_{k=0}^{\infty} (f_{n_{k+1}} - f_{n_{k}})$, where $f_{n_0} = 0$.
	Because
	\[
	\int _E \sum_{k=0}^{\infty} |f_{n_{k+1}} - f_{n_{k}}|
	= \sum_{k=0}^{\infty} \int_E |f_{n_{k+1}-f_{n_k}}| \le \sum_{k=0}^{\infty} 2^{-k}
	< +\infty.
	\]
	so our $f$ is well-defined (convergent).
	Now we compute
	\begin{align*}
		\lVert f-f_m \rVert
		&= \lVert f_m - f_{n_l} \rVert +
		\left\lVert \sum_{k=l}^{\infty} (f_{n_{k+1}} - f_{n_k}) \right\rVert\\
		&\le \lVert f_m - f_{n_l} \rVert +
		\sum_{k=l}^{\infty} \lVert f_{n_{k+1}}-f_{n_k}\rVert\\
		&\le \lVert f_m - f_{n_l} \rVert + 2^{-l+1}.
	\end{align*}
	As $m$ gets large, $\lVert f_m - f_{n_l} \rVert$ and $2^{-l+1}$ both
	converge to $0$, so $f_n\to f$ in $\mathcal{L}^1(E)$.
\end{proof}
