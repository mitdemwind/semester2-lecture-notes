%! TeX root = ./main.tex
Note that $\omega_1\wedge \omega_2 = (-1)^{k_1k_2}\omega_2\wedge \omega_1$,
so when $k_1k_2$ is even, the wedge product may not be anti-symmetrical.

If we have a coordinate transformation  $\Phi: (x_1,\dots,x_n)\to (y_1,\dots,y_n)$,
we have
\[
	\dd y_i = \sum_{j=1}^{n} \pfr{y_i}{x_i} \dd x_i,
\]
thus $\dd y_1\cdots \dd y_n = J_\Phi \dd x_1 \cdots\dd x_n$.
Here $J_\Phi$ can be negative, so the differential forms already contains
the information of orientation.

\begin{theorem}
    Let $\omega$ be a differential form, $\dd (\dd \omega) = 0$.
\end{theorem}
\begin{proof}[Proof]
    Partial derivatives commute.
\end{proof}

\begin{definition}
	Let $\omega$ be a differential form, if $\dd \omega = 0$, we say
	$\omega$ is a \vocab{closed form}, if there exists $\omega_1$ s.t.
	$\dd \omega_1 = \omega$, then $\omega$ is a \vocab{exact form}.
\end{definition}

The theorem above tells us that exact forms must be closed,
but in general closed forms may not be exact, it depends on the topology
structure of $\Omega$.

\begin{theorem}[Poincare]
	The closed forms on $\mathbb{R}^{n}$ must be exact.
\end{theorem}
\begin{proof}[Proof]
	Use induction, when $\omega$ is an $n$-form this can be proved by computation.

	For a generic form $\omega = \omega_1 + \dd x_1\wedge \omega_2$,
	where $\omega_1, \omega_2$ do not contain $\dd x_1$.
	We want to find $\omega_3$ s.t.
	$\dd \omega_3 = \dd x_1\wedge \omega_2 + \omega_4$,
	where $\omega_3, \omega_4$ don't contain $\dd x_1$ as well.
	(The construction is direct)

	Since $\omega - \dd \omega_3 = \omega_1 - \omega_4$,
	and
	\[
		\dd(\omega - \dd\omega_3) = \dd \omega = 0 \implies
		\dd(\omega_1 - \omega_4) = 0.
	\]
	Since $\dd (\omega_1 - \omega_4) = 0$ and it doesn't contain $\dd x_1$,
	hence all its coefficients can't contain $\dd x_1$.
	Thus we can view it as a differential form in $\mathbb{R}^{n-1}$.
\end{proof}
\begin{remark}
    When $\Omega$ is simply connected, then all the closed $1$-forms are exact.
	Also this is equivalent to the integral on any closed curves are $0$.
\end{remark}

We can rewrite Stolkes' formula using differential forms:
\begin{theorem}[Stolkes' formula]
    Let $D$ be a $k+1$ dimensional orientable manifold,
	$\omega \in \Lambda^k(\mathbb{R}^{n})$, we have
	\[
	\int_D \dd \omega = \int_{\partial D} \omega.
	\]
\end{theorem}
