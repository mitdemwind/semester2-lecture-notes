%! TeX root = ./main.tex
\begin{example}
    We can't grant that the global function exists.
	For example, let $D = \{x^2+y^2\in [a^2, b^2]\}$,
	and $M$ be a helicoid.

	Since there's a natural map
	$\phi: D\backslash([a,b]\times \{0\})\to M$ (projection),
	let $g, h$ be the fundamental forms of $\phi$,
	by the symmetry we can extend  $g,h$ to entire $D$.

	It's clear that there exists local solutions but
	the global solutions does't exist.
	(In theory of differential forms, this is similar to closed forms may
	not be exact)

	But if the region $D$ is \textit{simply connected},
	the global solution always exist.
\end{example}
\subsection{Isometric, conformal and area-perserving maps}
\label{sub:Isometric, conformal and area-perserving maps}
Let $U,\wt{U}\subset \mathbb{R}^2$, and $\phi: U\to \mathbb{E}^3,
\wt{\phi}: \wt{U}\to \mathbb{E}^3$ be two surfaces.
Let $f: \wt{U}\to U$ be a map between two surfaces.

Earlier we introduced isometric maps (isometry),
i.e. $f^*(g) = \wt{g}$.
Since the length depends only on the first fundamental form,
the isometry perserves the length, angles and areas on surfaces.

The \vocab{conformal} maps perserves the angles on the surfaces,
and it's easy to imply this is equivalent to $f^*(g) = \lambda \wt{g}$
for some $\lambda\in \mathbb{R}$.

As the name suggests, the \vocab{area-perserving} maps
perserves the areas on two surfaces,
which is saying $\det f^*(g) = \det \wt{g}$.
