%! Tex root = ./main.tex
Note that $X$ may be disconnected, so the fundamental group
is dependent of the base point $x_0$.
If $\gamma = \left<c \right>$ is a homotopy class of paths from $x_0$ to $x_1$,
then $\gamma$ induces a group homomorphism:
\[
\gamma_\sharp : \pi_1(X, x_0)\to \pi_1(X, x_1):
\left<a \right>\mapsto \left< \overline{c}ac \right>.
\]

It's easy to see $\gamma_\sharp$ is an isomorphism.

Hence $\pi_1(X, x_0)$ only depends on the path connected components of $x_0$.
Thus if $X$ is path connected, and $X, Y$ are homotopy equivalent,
then $\pi_1(X, x_0)\cong \pi_1(Y, y_0)$, or sometimes we can leave the base
point out, just write $\pi_1(X)\cong \pi_1(Y)$.
\begin{remark}
    If $x_0 = x_1$, then $\gamma\mapsto \gamma_\#$ gives a homomorphism
	$\pi_1(X, x_0)\to \Aut(\pi_1(X,x_0))$.
\end{remark}

\begin{example}
    If $X\simeq \{pt\}$, then $\pi_1(X) \cong \{1\}$.
	In this case, $X$ is called a \vocab{contractible space}.
	Note that the inverse is not true, e.g. $X = S^n$ for $n\ge 2$.
	Path connected spaces with trivial fundamental group
	are called \vocab{simple connected} spaces.

	Some classic contractible space includes convex sets in $\mathbb{R}^n$,
	trees in graph theory and cones $CX = X\times [0, 1]/ X\times \{1\}$.
\end{example}

Some more complex contractible examples including ``a house with two rooms'',
the equitorial inclusion $S^\infty = \bigcup_{n=0}^\infty S^n$ with
limit topology, i.e. the largest topology s.t. $S^n\to S^\infty$ continuous.

There are several concepts:
\begin{itemize}
	\item Retraction: $f:X \to A$, $A \subset X$, $f\big|_{A} = \id_A$.
	\item Deformation retraction: $f$ as above with $i\circ f \simeq \id_X$,
		where  $i: A\to X$ is the inclusion.
	\item Strong deformation retraction: $f$ as above with
		$i\circ f \simeq \id_X \ rel\ A$.
\end{itemize}
The set $A$ is called (strong) deformation kernel of $f$.

\begin{example}[Differences between deformation and strong deformation]
    Let $X$ be the follwing space:
	\[
		([0,1]\times \{0\}) \cup ([0, 1]_{\mathbb{Q}} \times [0, 1])
	\]
	We know $X\simeq \{pt\}$,
	but $\{q\}\times [0,1]$ is deformation kernel but not strong deformation kernel.
\end{example}

\subsection{Fundamental groups}
\label{sub:Fundamental groups}

After introducing the fundamental groups, a natural question arises:
how to compute the fundamental group of a given space?
We first state the main result of this section:
\begin{theorem}[Van Kampen]
	\label{thm:vankampen}
    Let $X = U'\cup U''$ be a topology space such that
	$U', U''$ are open and $W = U'\cap U''$ path connected, then for $x_0\in W$,
	we have
	\[
		\pi_1(X, x_0) \cong \pi_1(U', x_0) * \pi_1(U'', x_0) / N,
	\]
	where $N$ is the smallest normal subgroup generated by
	\[
	i'_\sharp(\delta) i''_\sharp(\delta^{-1}): \delta\in \pi_1(W, x_0),
	\]
	\begin{equation*}
	\begin{tikzcd}[sep=small, row sep=tiny, cramped]
		&U'\drar["j'"]\\ W\urar["i'"]\drar["i''"'] &&X\\ &U''\urar["j''"']
	\end{tikzcd}
	\end{equation*}
	and $*$ means free product.
\end{theorem}

Note that this theorem is useless when both $U', U''$ have trivial fundamental groups.
Thus we need to find a space with nontrivial fundamental group first.
One of the simplest example is $S^1$ :
\[
\pi_1(S^n, x_0) \cong \left\{\begin{aligned}
		\{&1\}, & n\ge 2,\\ &\mathbb{Z}, & n = 1.
\end{aligned}\right.
\]
Let $X\vee Y := X\sqcup Y / (x_0=y_0)$, then $\pi_1(X\vee Y) = \pi_1(X)*\pi_1(Y)$.
Thus $\pi_1(\underbrace{S^1\vee \dots\vee S^1}_{k}) = \mathbb{Z}*\dots*\mathbb{Z}
= \mathbb{F}_k$, the free group of rank $k$.

\begin{example}
    Since $nT^2$ is formed by $2n$ loops(borders of the polynomial representation)
	fused with a disk. Note that $W = U'\cap U'' \cong S^1$, so
	\[
	\pi_1(nT^2) = \left<a_1,b_1,\dots,a_n,b_n \mid [a_1,b_1]\cdots[a_n,b_n]=1\right>.
	\]
	Similarly,
	\[
	\pi_1(mP^2) = \left<c_1,\dots,c_m \mid c_1^2\cdots c_m^2 = 1 \right>.
	\]
\end{example}
\begin{example}
	For product spaces, we have
    \[
    \pi_1(X\times Y, (x_0, y_0)) \cong \pi_1(X, x_0) \times \pi_1(Y, y_0).
    \]
	In fact the isomorphism can be written down with $i_x, i_y, p_x, p_y$,
	i.e. $p_{x\sharp} \times  p_{y\sharp}$ and $(i_{x\sharp}, i_{y\sharp})$.
\end{example}
