%! TeX root = ./main.tex
At last, we'll combined what we've learned and prove a well-known theorem:
\begin{theorem}
    Simply connected surfaces with complete metric and constant curvature $-1$
	are globally isometrically isomorphic to $\mathbb{H}^2$.
\end{theorem}
\begin{remark}
    Here the curvature is the Gauss curvature. The proof is similar for
	$ \mathbb{S}^2(k=1)$ and $\mathbb{E}^2(k=0)$.
\end{remark}
\begin{proof}[Sketch of the proof]
    The surface can be viewed as a manifold, whose charts are assigned
	the first fundamental form. The proof can be spilt to 2 parts,
	one for local properties and one for global properties.

	If we have the local result, i.e. each point has an open neighborhood
	homeomorphic to an open disk in $\mathbb{H}^2$, we'll prove the theorem:

	$\bullet$ There exists a unique well-defined locally isometric
	extension $f: M\to \mathbb{H}^2$. (Here we need $M$ simply connected)

	Since $f$ is locally isometric, $f$ is a covering map.
	But $\pi_1(M) = \{1\}$, by the uniqueness of universal covering,
	there exists an isomorphism of coverings $\sigma$
	s.t. $f = \id\circ \sigma = \sigma$. Thus $f$ is a homeomorphism.

	$\bullet$ Locally, we'll take a geodesic parallel parameter,
	i.e. the $y$-axis and $x$-curves are geodesic lines.
	We have $I = \dd x^2 + G(x, y)\dd y^2$, where $G(0, y) = 1$, $G_x(0, y) = 0$.
\end{proof}
