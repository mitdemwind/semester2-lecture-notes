%! TeX root = ./main.tex
\begin{proof}[Proof]
    If $\wt{x}, \wt{X}'$ are both universal coverings,
	by map lifting theorem, since $\pi_1(\wt{X})$ is trivial,
	$p: \wt{X} \to X$ can be lifted to $\sigma: \wt{X}\to \wt{X}'$,
	similarly we have $\sigma'$, and it's easy to see $\sigma$ and
	$\sigma'$ are inverse maps, so they are isomorphic.

	For existence part, $X$ locally semi-simply connected means
	for $\forall x\in X$, there exists a neighborhood basis $\{U_i\}$ s.t.
	$\pi_1(U_i, x) \to \pi_1(X, x)$ is trivial.

	Let $P(X, x_0)$ be all paths in $X$ starting from $x_0$,
	and $\mathscr{X}$ is the homology equivalent classes
	(with fixed endpoints) of $P(X, x_0)$.

	Let $p: \mathscr{X}\to X$ by $\left<a \right>\mapsto a(1)$,
	and $\tilde x_0$ denote the constant path.

	Next we'll define the topology on $\mathscr{X}$ :

	Let $\{U_\alpha\}$ be a topology basis of $X$,
	consider the following sets:
	\[
	U(U_\alpha, a) = \{\left<ac \right>\mid c\in P(U_\alpha, a(1))\}.
	\]
	Let the topology basis on $\mathscr{X}$ be the above sets.
	We claim $p: \mathscr{X}\to X$ is indeed a covering.
\end{proof}

\begin{example}
    A counter example of above theorem when $X$ is not locally
	semi-simply connected:
	Hawaiian earrings (a family of tangent circles with radius $\to 0$).
\end{example}

Now we can view all these things from group actions.

Let $X$ be a topological space, $G$ is a group acting on $X$.
We say the action is \vocab{freely discontinuous} if
for all $x\in X$, there's a neighborhood $U$ s.t. $gU \cap U \ne \emptyset$ only
holds for $g = e$.

\begin{proposition}
	Let $G\curvearrowright X$ be a freely discontinuous action,
	then the quotient map $X \to X / G$ by $x\mapsto Gx$ is
	a regular covering, and the group action is just deck transformations.
\end{proposition}

\begin{example}
    The antipodal map in $S^n$ generates a group $\{\pm 1\}$,
	and the action is freely discontinuous,
	so $S^n \to S^n / \{\pm 1\} = \mathbb{R}P^n$ is a covering.

	Let $\alpha: (x, y)\mapsto (x, y+1)$ and $\beta: (x, y)\mapsto (x+1, -y)$ on
	$\mathbb{E}^2$ generates a group action $G \curvearrowright \mathbb{E}^2$.
	This is also freely discontinuous, and $\mathbb{E}^2 / G$ is a Klein bottle.
\end{example}

Let $X$ be a topological space, $G$ is a group acting on $X$.
We say the action is \vocab{properly discontinuous} if
for all compact set $K \subset X$, $gK \cap K \ne \emptyset$ only
holds for finitely many $g$.
