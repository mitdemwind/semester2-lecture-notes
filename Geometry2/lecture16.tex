%! TeX root = ./main.tex
\subsection{A bit of manifold}
\label{sub:Intro}

First we'll introduce a few concepts before we move on.
\begin{itemize}
	\item We say a topological space is an $n$-dimensional \vocab{topological manifold}
	if it's Hausdorff and locally homeomorphic to $\mathbb{R}^{n}$.
	Sometimes we also require manifolds to be compact / paracompact / $C_2$.
	Here paracompact means that any open covering has a locally finite subcovering.

	\item Manifolds with boundary:
		locally homeomorphic to $\mathbb{R}^{n-1}\times [0, +\infty)$.

	\item When we talk about the regularity of manifolds, we must
		appoint an atlas first.
		Let $\phi_i: U_i \to E_i \subset \mathbb{R}^{n}$ be
		homeomorphisms mentioned above,
		then each $\phi_i$ is a \vocab{chart}, and $\{(U_i, \phi_i)\}_{i\in I}$ is
		the \vocab{atlas}. The map
		\[
			\phi_j\circ\phi_i^{-1}: \phi_i(U_i\cap U_j)\to \phi_j(U_i\cap U_j)
		\]
		is called \vocab{transition functions}.

		The regularity of the manifold is actually the regularity of
		transition functions, such as $C^r, C^\infty$, piecewise linear, etc.
\end{itemize}
\begin{example}
    The sphere $\mathbb{S}^2$ and projective plane $\mathbb{R}P^2$ are 2d manifolds.
	But they're different since $\mathbb{R}P^2$ is not \textit{orientable}.
	In fact $\mathbb{R}P^2$ can be obtained by fusing the edge of a Mobius band
	to a disk(keep in mind that Mobius band has only one edge!).
\end{example}
There are many manifolds which looks wired, but I can't draw them on the computer ;)

\begin{example}[Projective curves]
    Consider a quadratic equation
	\[
	C: z^2 + w^2 = 1,\quad (z, w)\in \mathbb{C}^2.
	\]
	What does this surface look like?

	Let $Z = z+iw, W = z-iw$, the equation becomes $ZW = 1$,
	hence the surface is $(\zeta, \frac{1}{\zeta}), \zeta\in \mathbb{C}\backslash\{0\}$.
	So $C$ is homeomorphic to $\mathbb{C}\backslash\{0\}$.

	We can also discuss this in $\mathbb{C}P^2 = \mathbb{C}P^1 \cup \mathbb{C}^2$,
	where $\mathbb{C}P^1 = \{\infty\} \cup \mathbb{C}\cong \mathbb{S}^2$.

	So in homogeneous coordinate, the equation can be written as $ZW = T^2$.
	The surface is consisting of $(1,0,0), (0,1,0), (\zeta, \frac{1}{\zeta}, 0)$.
	Thus the projective completion of $C$ is homeomorphic
	to $\mathbb{S}^2$, which is $\mathbb{C}\backslash\{0\}$
	appending with two points.
\end{example}

\begin{example}[Elliptic curves]
    Let $\lambda_1, \lambda_2, \lambda_3\in \mathbb{C}$ pairwise different.
	\[
	E: w^2 = (z-\lambda_1)(z-\lambda_2)(z-\lambda_3).
	\]
	What does $E$ looks like in $\mathbb{C}P^2$?

	Observe that for $z\in \mathbb{C}\backslash\{\lambda_1,\lambda_2,\lambda_3\}$,
	there're 2 values for $w$.
	So the image of $E$ is two planes($\mathbb{C}$) fused together
	at $\lambda_1, \lambda_2, \lambda_3$ and $\infty$ with some adjust.

	In fact this can be realized as two cylinder fused together at their edges.

	$E\cong T^2 \backslash \{pt\}$ in $\mathbb{C}^2$, and $T^2$ in $\mathbb{C}P^2$.
\end{example}
