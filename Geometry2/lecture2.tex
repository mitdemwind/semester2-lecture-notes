%! TeX root = ./main.tex
\begin{remark}
    On how to construct regular manifolds:

	Let $f: \mathbb{R}^n\to \mathbb{R}^{n-m}$ be a smooth function,
	for a fixed $y\in \mathbb{R}^{n-m}$, if $\forall x\in f^{-1}(y)$,
	$Df(x)$ has rank $n-m$, then $M:=f^{-1}(y)$ is an $m$-dimensional
	regular manifold.

	In fact this is known as ``Regular Value Theorem'' in literarture,
	and $y$ is called a regular value of $f$. This leads to a branch in
	mathematics, namely \textit{differential topology}.
\end{remark}
\begin{remark}
    On real/complex analysis:
	Holomorphic (which is the complex version of differentiable) is way
	stronger than smooth condition.
\end{remark}

\section{Theory of space curve}
\label{sec:Theory of space curve}
In this section we mainly discuss the regular parametrized curves
$\gamma: J\to \mathbb{E}^3$.

Our goal is to find some identities to describe the ``shape'' of the curves.
Since the curve is 1-dimensional manifold in 3 dimensional space,
somehow we should find 3 identities to describe it, including length and
another two concerning how it ``bends''.

\subsection{Arc length}
\label{sub:arc length}
\begin{definition}[Arc length]
	Let $\gamma: J\to \mathbb{E}^3$ be a regular parametrized curve.
	In an interval $[a,b] \subset J$, we define its length to be
	\[
		Length_\gamma([a,b]) := \int_{a}^{b} \lVert \gamma'(t)\rVert \dd t.
	\]
	where $\gamma'(t)\in V(\mathbb{E}^3)=\mathbb{R}^3$.
\end{definition}
\begin{proposition}
	Arc length is a \vocab{geometry quantity}, i.e. fixed under reparametrization.
\end{proposition}

\begin{proof}[Proof]
    For an arbitary regular reparametrization $t = t(\tilde{t})$,
	$\tilde{\gamma}(\tilde{t}) = \gamma(t)$,
	by Chain rule we get
	\begin{align*}
		Length_{\gamma}([a,b]) &= \int _a^b |\gamma'(t)| \dd t\\
		&= \int_{\tilde{a}}^{\tilde{b}} |\gamma'(t(\tilde{t}))|
		\frac{\dd t}{\dd \tilde{t}} \dd \tilde{t}\\
		&=\int_{\tilde{a}}^{\tilde{b}} |\tilde\gamma'(\tilde t)| \dd \tilde{t}
		= Length_{\tilde{\gamma}}([\tilde a,\tilde b]).
	\end{align*}
	However, here we used the fact that $\frac{\dd t}{\dd \tilde t}$
	is positive, so when the reparametrization reverses the orientation,
	we need to take extra care of it.

	\begin{align*}
		Length_{\gamma}([a,b]) &= \int _a^b |\gamma'(t)| \dd t\\
		&= \int_{\tilde{a}}^{\tilde{b}} |\gamma'(t(\tilde{t}))|
		\frac{\dd t}{\dd \tilde{t}} \dd \tilde{t}\\
		&=\int^{\tilde{a}}_{\tilde{b}} |\tilde\gamma'(\tilde t)| \dd \tilde{t}
		= Length_{\tilde{\gamma}}([\tilde b, \tilde a]).
	\end{align*}
\end{proof}

The arc length induces a parametrization for regular curves, namely the
\vocab{arc length parameter} $\gamma(s)$, with $\lVert \frac{\dd\gamma}{\dd s}\rVert = 1$
everywhere.

\subsection{Curvature}
\label{sub:Curvature}
\begin{definition}[Curvature]
	Let $\gamma(s)$ be a regular curve with arc length parameter,
	define its \vocab{curvature} to be
	\[
	\curv_\gamma(s) = \kappa(s) := \lVert \gamma''(s)\rVert.
	\]
	Since it is deduced from arc length (which is a geometry quantity), it
	must be a geometry quantity as well.
\end{definition}
\begin{remark}
    Sometimes $\gamma''(s)$ is called the curvature vector.
	It's parallel to the normal vector and can be interpreted as
	centripedal force.
\end{remark}

\begin{example}
    For a straight line, its curvature is always $0$.

	For a circle with radius $R$,
	$\gamma(s)=(R\cos(\frac{s}{R}), R\sin(\frac{s}{R}))$,
	so $\curv_\gamma(s) = \frac{1}{R}$.
\end{example}

\begin{proposition}
    When the parameter is a general parameter $\gamma(t)$, the curvature is equal to:
	\[
	\curv_\gamma(t) = \frac{\lVert\gamma''(t) \times \gamma'(t)\rVert}
	{\lVert\gamma'(t)\rVert^3}.
	\]
\end{proposition}
\begin{example}
    Let $\Gamma: x^2+k^2y^2 = 1$, calculate curvature of $\Gamma$ at point $(x,y)$.
\end{example}
\begin{proof}[Solution]
    First we take a parametrization for $\Gamma$:
	$(x, y) = (\cos t, \frac{1}{k}\sin t)$.

	Then compute the derivatives:
	\[(x', y') = (-\sin t, \frac{1}{k}\cos t) = (-ky, \frac{1}{k}x),\]
	\[(x'', y'') = (-\cos t,-\frac{1}{k}\sin t) = (-x, -y).\]
	\[
	\curv_\Gamma = \frac{|ky^2 + \frac{1}{k}x^2|}
	{(k^2y^2+\frac{1}{k^2}x^2)^{\frac{3}{2}}}
	= \frac{1}{k(\frac{1}{k^2}x^2 + k^2y^2)^{\frac{3}{2}}}.
	\]
	When $(x,y)=(1,0)$,  $\curv = k^2$; when $(x,y)=(0,\frac{1}{k})$,
	$\curv = \frac{1}{k}$.
\end{proof}
\begin{remark}
    Osculating circle: A circle tangent to the curve with
	the same curvature as the curve at the tangent point.
	Specifically, its radius is equal to $\frac{1}{\curv}$.

	This is useful in engineering to indicate the curvature
	of a curve.
\end{remark}

\subsection{Torsion and Frenet frame}
\label{sub:Torsion and Frenet frame}
\begin{definition}[Torsion]
	Let $\gamma(s)$ be a curve with arc length parameter.

	Let $\vec{t}:=\gamma'(s), \vec{n}:=\frac{\gamma''(s)}{\lVert\gamma''(s)\rVert}$
	be the tangent vector and normal vector.

	Let $\vec{b} = \vec{t}\times \vec{n}$ be the \vocab{binormal vector}.
	Define the \vocab{torsion} to be
	\[
	\tors_\gamma(s) = \tau(s) := - \vec{b}' \cdot \vec{n}.
	\]
	In fact $\vec{b}'$ is parallel to $\vec{n}$:
	\[
	\vec{b}'= \vec{t}'\times \vec{n} + \vec{t}\times \vec{n}'
	= \vec{t}\times\vec{n}'\perp\vec{t},
	\]
	and $\lVert \vec{b}\rVert = 1$, so $\vec{b}\perp\vec{b}'$,
	so $\vv{b}' \parallel \vec{n}$.
\end{definition}

The torsion's geometric meaning is less intuitive than the previous ones.
It describes how much the curve is moving ``out'' the plane it currently
lies in.
