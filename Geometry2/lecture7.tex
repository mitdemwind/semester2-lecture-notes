%! TeX root = ./main.tex


\subsection{The second fundamental form}
\label{sub:The second fundamental form}
Since the first fundamental form is not sufficient to describe
all the properties of the surface (it can only describe the curves lying
on it and the area), we thereby introduce the second fundamental form.

\begin{definition}[The second fundamental form]
	Let $\phi:U\to \mathbb{E}^3$ be a regular parametrized surface.
	The normal vector at the point $u=(s,t)$ is defined as
	\[
		\vec{n} = \frac{\phi_s\times \phi_t}{\lVert \phi_s\times \phi_t \rVert }.
	\]
	Since the cross product cares about orientation, so
	the normal vector is only fixed under orientation-preserving
	reparametrization.

	Now we expand $\phi$ to the second derivative:
	\[
	\phi(s+\Delta s, t+\Delta t) =
	\phi(s,t) + (\phi_s, \phi_t)\colvec{\Delta s}{\Delta t}
	+ \frac{1}{2} (\Delta s, \Delta t)\begin{pmatrix}
		\phi_{ss} &\phi_{st}\\ \phi_{ts} &\phi_{tt}
	\end{pmatrix} \colvec{\Delta s}{\Delta t}
	+ o(|\Delta s|^2 + |\Delta t|^2).
	\]
	Hence we define
	\[
	L = \phi_{ss}\cdot\vec{n}, \quad M = \phi_{st}\cdot\vec{n},
	\quad N = \phi_{tt}\cdot\vec{n}.
	\]
	The second fundamental form is defined as
	$h = L\dd s^2 + M\dd s\dd t + N\dd t^2$.
\end{definition}
\begin{remark}
    Another expression of $L,M,N$:
	 \[
	L = -\phi_s \cdot \vec{n}_s,\quad M = -\phi_s\cdot \vec{n}_t
	= -\phi_t \cdot \vec{n}_s, \quad N = -\phi_t \cdot \vec{n}_t.
	\]
\end{remark}

Intuitively, the second fundamental form describes
how much the surface is ``going out'' of the tangent plane.

Since the first fundamental form $g$ gives an inner product of
the tangent space, so we can compute the ``canonical form'' of
$h$ with respect to $g$, this process will generate some geometric quantities.

\begin{definition}
	Define the \vocab{average curvature} and \vocab{Gaussian curvature}:
	\[
	H:= \frac{1}{2}\frac{LG-2MF+NE}{EG-F^2},\quad
	K:= \frac{LN-M^2}{EG-F^2}.
\]
\end{definition}

These expressions look complicated and ugly, the reason is that we didn't
choose the right parameters. Indeed, if at some point $u=(s,t)$ we have
\[
\begin{pmatrix} E &F\\F&G \end{pmatrix} =
\begin{pmatrix} 1&0\\0&1 \end{pmatrix},
\]
then
\[
H = \frac{1}{2} \tr \begin{pmatrix} L&M\\M&N \end{pmatrix}, \quad
K = \det \begin{pmatrix} L&M\\M&N \end{pmatrix}.
\]

\begin{definition}[Principal curvatures]
	The characteristic polynomial $\lambda^2 - 2H\lambda + K$
	has two real roots, they are called the
	\vocab{principal curvatures} of $\phi$.
	The \vocab{principal directions} are defined as
	the directions of eigenvectors of $h: T_uU\times T_uU\to \mathbb{R}$
	WRT the inner product $g$.
\end{definition}

Now we'll dig deeper into the geometric meaning of these formulas.
\begin{proposition}
	$H$ and $K$ are geometric quantities.
\end{proposition}
\begin{proof}[Proof]
    For any reparametrization $s=s(\tilde s, \tilde t), t=t(\tilde s,\tilde t)$,
	we have
	\[
		\begin{pmatrix}\tilde E&\tilde F\\ \tilde F&\tilde G \end{pmatrix}
		= J\begin{pmatrix} E&F\\F&G \end{pmatrix}J^{-1}.
	\]

	Similarly we can verify that
	\[
		\begin{pmatrix}\tilde L&\tilde M\\ \tilde M&\tilde N \end{pmatrix}
		= J\begin{pmatrix} L&M\\M&N \end{pmatrix}J^{-1}.
	\]

	Since
	\[
	H = \frac{1}{2}\tr\EFFG^{-1}\LMMN,\quad
	K = \det \EFFG^{-1}\LMMN
	\]
	Thus $H,K$ are fixed under orientation-preserving reparametrization.
\end{proof}
\begin{remark}
    When the reparametrization is orientation-reversing, $L,M,N$ all
	differ by a sign, thus $H$ will change while $K$ is still fixed.
\end{remark}

Some examples about curvatures:
\begin{example}
	A sphere with radius $R$:
	\[
	H = \frac{1}{R}, \quad K = \frac{1}{R^2}.
	\]
	Note that
	\[
	\int _{S^2} K \dd Area = \frac{1}{R^2}4\pi R^2 = 2\pi \chi(S^2).
	\]
	This is an example of Gauss-Bonnet formula which we'll cover later.
\end{example}
