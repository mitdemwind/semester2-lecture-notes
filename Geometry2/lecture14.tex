%! TeX root = ./main.tex
It's easy to prove that isometric = conformal + area-perserving.
These three properties induce Riemann geometry, complex geometry
and sympletic geometry, respectively (in two dimensional).

\subsubsection{Isometries}
Firstly by Gauss' Theorema Egregium, Isometries perserves
Gaussian curvature.

\begin{example}
    Let $S_{a,b}: \frac{x^2}{a} + \frac{y^2}{b} = 2z$ be a saddle surface.
	Let $(x,y,z) = (as, bt, \frac{as^2+bt^2}{2})$ be a parametrization.

	We can compute the fundamental forms:
	\[
	g = a^2(1+s^2)\dd s^2 + 2abst\dd s\dd t + b^2(1+t^2)\dd t^2,
	\]
	\[
	h = \frac{a\dd s^2 + b\dd t^2}{\sqrt{1+s^2+t^2}}.
	\]
	So $K = \frac{1}{ab(1+s^2+t^2)^2}$.
	In fact the Gaussian curvature of some different surfaces,
	say $S_{2,3}$ and $S_{1,6}$ are the same.

	But there is not an isometry between them:

	If $\tau: \mathbb{R}^2\to \mathbb{R}^2$ is an isometry, then $\tau$ fixes
	the circles centered at $(0,0)$ as their Gaussian curvature are the same.
	Then $\tau_*=\dd \tau: T_{(0,0)}\mathbb{R}^2\to T_{(0,0)}\mathbb{R}^2$ can
	only be rotation or reflection.
	(If $\tau_*$ is not orthogonal, it will map small circles to ellipse)

	While $g(0,0) = a ^2\dd s^2 + b^2\dd s^2$, which has eigenvalue $a^2$
	and $b^2$, and they're fixed under $\tau_*$, so $S_{2,3}$ isn't
	isometric to $S_{1,6}$.
\end{example}
\begin{remark}
    Given $E,F,G: D\to \mathbb{R}$ s.t. $g = E\dd s^2 + 2F\dd s\dd t + G\dd t^2$
	positive definite, is there a surface $D\to \mathbb{E}^3$ can have
	$g$ as its first fundamental form locally?

	When we require $E,F,G$ to be $C^{\omega}$(analytic), the answer is ``yes'',
	but if we only require $C^{\infty}$, it's still an open problem.
\end{remark}

Even though we don't know the situation in 3 dimensional space,
we can study the case in higher dimensions:
\begin{theorem}
    It's always possible to construct $\phi: D\to \mathbb{E}^4$ to
	have $E,F,G$ as its first fundamental form.
\end{theorem}

Surfaces with Gaussian curvature 0 everywhere
are called \vocab{developable surfaces}.
Developable surfaces can only be cylinder, cone, tangent surface of a curve
and their concatenation.

\begin{example}[Pseudosphere]
	Let $\phi(x,y) = (\frac{\cos x}{y}, \frac{\sin x}{y},
	\cosh^{-1}(y) - \frac{\sqrt{y^2-1}}{y})$,
	where $(x,y)\in (-\pi, \pi)\times [1,+\infty)$.

	It's obtained by rotating a \textit{tractrix} around its asymptote.
	We can calculate its Gaussian curvature, whichis a constant $-1$.
	This is where the name comes from.

	Recall that hyperbolic plane also has constant curvature $-1$,
	in fact they are locally isometric.
	In 1901, Hilbert proved a theorem that there exists an isometry
	$\mathbb{H}^2\to \mathbb{E}^3$.
\end{example}
