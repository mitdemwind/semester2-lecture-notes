\section{Introduction}
\label{sec:Introduction}
\begin{center}
	\sffamily\large\bfseries Teacher: Liu Yi

	Email: liuyi@bicmr.pku.edu.cn

	homepage: \url{scholar.pku.edu.cn/liuyi} $\Rightarrow$``Teaching''
\end{center}

This course covers the topic of elementary \textit{differential geometry}
and \textit{fundamental groups in algebraic topology}.

Grading: Homework-Midterm-Final: 20-40-40

Midterm: Wednesday, week 8

\subsection{Intro}
\label{sub:Intro}
\begin{definition}[Manifold]
	Let $M$ be an open subset of $\mathbb{R}^n$, we call $M$ an $m$-dimensional regular
	manifold of $\mathbb{R}^n$, if $\forall p\in M$, exists an open neighborhood
	$W \subset \mathbb{R}^n$ such that there exists open set $U \subset \mathbb{R}^m$
	and homeomorphism $\phi: U\to M \cap W$, satisfying the Jabobi matrix of $\phi$
	is injective everywhere, i.e.
	\[
	\D\phi(x) = \begin{pmatrix}
		\frac{\partial\phi_1}{\partial u_1} &\cdots &\frac{\partial\phi_1}{\partial u_m}\\
		\vdots &\ddots &\vdots \\
		\frac{\partial\phi_n}{\partial u_1} &\cdots &\frac{\partial\phi_n}{\partial u_m}
	\end{pmatrix}
	\]
	has rank $m$ for all  $x\in U$.
\end{definition}
When $n=3$, we say $M$ is a curve for $m=1$, and a surface for $m=2$.
\begin{remark}
    The term ``$C^r$ regular manifold'' means $\phi$ is a  $C^r$ function.
\end{remark}

\begin{example}
    Some quadratic surfaces like cylinders, biparted hyperboloids and saddle surfaces
	are all regular 2-manifolds, but a cone is not a regular manifold.
\end{example}
\begin{example}
    The curve $\phi(r) = (\cos(2\pi r), \sin (2\pi r), r)$ is a 1-manifold in $\mathbb{R}^3$.
	This curve is called a \textit{helix}.
\end{example}
\begin{example}[Cycloid]
    A cycloid is the locus of a point on a circle while the circle ``rolls'' along a line.
	When the point lies inside resp. outside the circle, the curve is called
	curtate cycloid resp. procolate cycloid.

	{\bfseries TODO: Images here.}

	The cycloid is not a manifold because it has singularity where it touches the line,
	and procolate cycloids are also not manifolds as they have self-intersections.
\end{example}
\begin{remark}
    The regular manifolds we talk about are also called ``embedded manifolds'',
	the ones with self-intersections can be discribed as ``immersed manifolds'',
	such as the curves in the previous example. The immersed manifolds are complex
	and hence beyond the scope of this class.
\end{remark}

However, it turns out that the curves or surfaces with self-intersections also have some
properties, so we need to find a way to describe them. This induces:

\begin{definition}[Regular parametrized curve]
	Let $\gamma: J\to \mathbb{R}^3$ be a function, where $J$ is an open interval.
	if for every point $p\in J$, there exists open neighborhood  $J'$ s.t.
	$\gamma\big|_{J'}$ is a regular 1-manifold, then we say $\gamma(J)$
	is a regular parametrized curve, and  $\gamma\big|_{J'}$ is
	called its regular parametrization.
\end{definition}
Likewise, we have:
\begin{definition}[Regular parametrizd surface]
	Let $\phi:U\to \mathbb{R}^3$ be a function, where $U\in \mathbb{R}^2$ is an open set.
	If for every point $p\in U$, there exists open neighborhood  $U'$ s.t.
	$\phi\big|_{U'}$ is a regular 2-manifold, we say $\phi(U)$ is
	a regular parametrized surface.
\end{definition}

\subsection{Prerequisites}
\label{sub:Prerequisites}
\paragraph{Vector calculus}
Let $\vec{v}(t)$ be a 3-dimensional vector function,
its derivative resp. integration is the vector formed by taking
derivative resp. integration of each component,
and the derivative satisfies Leibniz's rule with respect to both dot product
and cross product.

\paragraph{Multi-variable calculus}
If $f: \mathbb{R}^m\to \mathbb{R}^n$ is $C^2$, then the partial derivative
can change order with each other:
 \[
	\frac{\partial}{\partial x_i} \frac{\partial}{\partial x_j} f =
	\frac{\partial}{\partial x_j} \frac{\partial}{\partial x_i} f.
\]

Integration of surfaces has two types:
\begin{enumerate}
	\item $\iint f \dd x\dd y$, multiple integrals.
	\item $\iint f \dd x\wedge \dd y$, integrals with orientation.
\end{enumerate}
