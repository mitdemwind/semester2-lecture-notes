%! TeX root = ./main.tex
\begin{example}
	The conical and cylindrical surfaces have Gaussian curvature $0$.

	For a general ruled surface, we can prove that $K\le 0$ everywhere.
\end{example}
\begin{example}
    Minimal surfaces(like soap bubbles) have $H=0$ and $K\le 0$ everywhere.
\end{example}
\begin{example}[Dupin canonical form]
	Let $\phi:U\to \mathbb{E}^3$ be a regular surface,
	then at the neighborhood of any point, there exists a
	parameter s.t. $\phi(s,t) = (s, t, \kappa_1s^2+\kappa_2t^2) + o(|s|^2+|t|^2)$,
	where $\kappa_1,\kappa_2$ are principal curvatures of $\phi$.

	In this case we can talk about concepts like ``elliptic point'',
	``parabolic point'' and ``hyperbolic point''.
\end{example}

Next we'll going to switch to a more intrinsic view to study the meaning
of those definitions again.

If we look at a curve $\gamma$ on a surface $\phi$,
let $r$ be the arc length parameter, then $\lVert \gamma' \rVert = 1$,
$\lVert \gamma'' \rVert = \kappa(r)$, note that $\gamma''$ can be decompose
with respect to the normal vector and tangent plane:
\[
	\gamma''=\kappa_n \vec{n}+\kappa_g\vec{n}\times \gamma'.
\]
Here $\kappa_n$ is called \vocab{normal curvature},
and $\kappa_g$ is called \vocab{geodesic curvature} of $\gamma$ WRT $\phi$.

Moreover we have \textit{Euler's formula}:
\[
\kappa^2 = \kappa_n^2 + \kappa_g^2.
\]

If we compute the normal curvature in terms of $u=(s,t)$:
 \[
\gamma' = \phi_s s' + \phi_t t'
\]
\[
\gamma'' = (s',t')\begin{pmatrix}
	\phi_{ss} &\phi_{st}\\\phi_{ts}&\phi_{tt}
\end{pmatrix}
\colvec{s'}{t'} + \phi_s s'' + \phi_t t''
\]
Hence
\[
\kappa_n = \gamma''\cdot \vec{n}
= (s', t')\LMMN \colvec{s'}{t'} = L(s')^2+2Ms't'+N(t')^2.
\]
This is the formula under the arc length parameter.
\begin{remark}
    The general formula of $\kappa_n$:
	\[
	\kappa_n = \frac{Ls'^2 + 2Ms't' + Nt'^2}{Es'^2 + 2Fs't' + Gt'^2}.
	\]
\end{remark}

The normal plane of $\gamma$ intersects the surface  $\phi$, the section curve
is called a \vocab{normal section}.

Oberserve that: if $ \lVert \gamma' \rVert = 1$, and the tangent vector is $\vec{t}$,
then $\kappa_n(r)$ is the curvature of the normal section at $u$ in the plane
spanned by $\vec{n},\vec{t}$.

Hence $\kappa_n$ can be viewed as a quadratic form $\vec{n}^\perp \to \mathbb{R}$ which
sends a vector $\vec{t}$ to the curvature of the normal section with
tangent vector $\vec{t}$.

Furthermore, the principal directions are the ``eigen-directions'' of $\kappa_n$,
which are the directions where the curvature of normal section attains
its extremum.

 \begin{example}
    Consider the helix and the cylinder
	\[
	\gamma(t)=(\cos t, \sin t, at),\quad S : x^2+y^2=1.
	\]
	It's easy to verify that $\kappa=\kappa_n=\frac{1}{1+a^2}$
	as $\gamma''$ is always perpendicular to $z$-axis.

	Note that $\kappa_g = 0$ everywhere, curves satisfying  $\kappa_g=0$ are
	called \vocab{geodesic line}.
\end{example}
