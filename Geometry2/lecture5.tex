%! TeX root = ./main.tex
% Missing lecture, notes are from the handout.
\section{Theory of surfaces}
\label{sec:Theory of surfaces}

\subsection{The first fundamental form}
\label{sub:The first fundamental form}

Let $\phi: U\to \mathbb{E}^3$ be a regular parametrized surface.
We denote the point in $U \subset \mathbb{R}^2$ as $u=(s,t)$.
Hence the partial derivative of  $\phi$ gives:
 \[
\phi_s(u) = \frac{\partial\phi}{\partial s}(u),\quad
\phi_t(u) = \frac{\partial\phi}{\partial t}(u).
\]
Define the first fundamental quantities
\[
E(u) = \phi_s(u)\cdot \phi_s(u), \quad
F(u) = \phi_s(u)\cdot \phi_t(u), \quad
G(u) = \phi_t(u)\cdot \phi_t(u).
\]

The first fundamental form of the surface $\phi$ is the
real bilinear form $T_u\phi(U)\times T_u\phi(U)\to \mathbb{R}$ :
 \[
g(u) = E(u)\dd s^2 + 2F(u) \dd s \dd t + G(u)\dd t^2.
\]

It can also be written as $\begin{pmatrix}
	E &F\\ F &G
\end{pmatrix}$.

\begin{remark}[On partial detivatives]
	When using the notation $\frac{\partial}{\partial x}$, we must
	declare the parameters we're using, e.g. $(x,y)$,
	For example when $x' = x, y' = x + y$, the meaning
	of $\frac{\partial}{\partial x}$ is different from
	$\frac{\partial}{\partial x}$.
\end{remark}

\begin{definition}[Length, angle and area]
	Let $\gamma(r) = \phi(u(r))$ be a curve on $\phi$, then its length is equal to
	\[
	Length = \int_{a}^{b} \lVert s'(r)\phi_s(u(r))+t'(r)\phi_t(u(r)) \rVert \dd r.
	\]

	Let $\alpha,\beta: [0,\epsilon]\to \mathbb{E}^3$ be two curves on $\phi$.
	Then the angle between  $\alpha$ and  $\beta$ (in $\mathbb{E}^3$)
	is equal to
	\[
	Angle = \arccos \left( \frac{\phi_s(u)\cdot \phi_t(u)}{\lVert \phi_s(u)\cdot\phi_t(u) \rVert } \right).
	\]

	Lastly, let $R \subset U$ be a closed region whose boundary
	is a regular curve, the area of $\phi$ on  $R$ is defined as
	\[
	Area = \iint_R \lVert \phi_s(u)\times \phi_t(u) \rVert \dd s\dd t.
	\]
\end{definition}
