%! TeX root = ./main.tex
\subsection{Gauss map and Weingarten map}
\label{sub:Gauss map and Weingarten map}

The strange definition of those curvatures don't
come from nothing, in this section we'll cover this topic
and give a geometric interpretation.

\begin{definition}[Gauss map]
	Let $\Sigma$ be a regular surface in $\mathbb{E}^3$,
	denote its normal vector at $x$ by $\vec{n}(x)$.
	Then this map $\mathcal{G}: \Sigma\to \mathbb{S}^2$ by
	$x\mapsto \vec{n}(x)$ is called the \vocab{Gaussian map}.

	In terms of a parametrized surface $\phi:U\to \mathbb{E}^3$,
	we can compute that
	\[
	\mathcal{G}: U\to \mathbb{S}^2:\quad
	\vec{n}(u) = \frac{\phi_s\times \phi_t}{\lVert \phi_s\times \phi_t \rVert}
	\]
\end{definition}

But each vector has a normal plane, namely $\vec{n}^\perp$,
and this derives the \vocab{Weingarten map}:
\begin{definition}[Weingarten map]
	For all $u\in U$, define $W: \vec{n}(u)^\perp\to \vec{n}(u)^\perp$:
	$\vec{v}\mapsto W(\vec{v})$,
	where
	\[
	W(\vec{v}) = - \frac{\dd (\mathcal{G}\circ \gamma)}{\dd u}\Bigg|_{u=0},
	\quad \gamma:=\phi(u(r))\text{ is a curve on the surface}.
	\]
\end{definition}
\begin{remark}
    In the language of modern differetial manifolds,
	Weingarten map is just the tangent map of Gauss map with a negative sign.
\end{remark}

Since $\vec{n}^\perp$ has a basis $\phi_s,\phi_t$, we can compute the matrix
of Weingarten map:
 \[
(\phi_s, \phi_t)W = (-\vec{n}_s, -\vec{n}_t).
\]
Note that $-\vec{n}_s \cdot \phi_s = \vec{n}\cdot \phi_{ss} = L$,
so if we take the inner product of $(\phi_s, \phi_t)$ on both sides,
we get
 \[
\EFFG W = \LMMN \implies W = \EFFG^{-1}\LMMN.
\]
Since $W$ is clearly a geometric quantity, so its trace and determinant are
also geometric:
 \[
\tr W = \frac{GL - 2FM + EN}{EG - F^2} = 2H,\quad
\det W = \frac{LN - M^2}{EG - F^2} = K,
\]
which gives the average curvature and Gauss curvature.

Moreover, the principal curvatures are the eigenvalues of $W$,
and  principal directions are just the eigenspaces of $W$.

Let $\vec{v}=(\phi_s,\phi_t)X$, then its normal section has curvature
\[
\kappa_n = \frac{X^T \LMMN X}{X^T \EFFG X}.
\]
When $ \lVert \vec{v} \rVert = 1$, we can change a parameter s.t.
$\EFFG = I_2$, in this case we can observe that
when $\kappa_n$ attains its extremum, $\vec{v}$ is precisely
the eigenvector of $W$, i.e. lies on the principal directions.

\begin{definition}[Curvature line]
	A curve is called a \vocab{curvature line} if its tangent vector
	is the same as principal directions everywhere.
\end{definition}
\begin{example}
    Every curve on a sphere is curvature line.

	Around a point where the principal curvatures are different,
	there exists a orthogonal grid of curvature lines.
\end{example}
\begin{example}
    monkey saddle surface, ``prong singularity''
	%TODO: add a picture here
\end{example}

In the case when the $s$-curve and  $t$-curve are precisely the
curvature lines, then we say this is a \vocab{curvature grid parameter},
and here we have $g = E\dd s^2 + G\dd t^2$ and $h = L\dd s^2 + N\dd t^2$.

\begin{remark}
    The geometric interpretation of Gauss curvature:
	For $u\in D \subset U$,
	\[
	|K(u)|=\lim_{``D\to u"}
	\frac{Area_{\mathbb{S}^2}(\mathcal{G}(D))}{Area_{\mathbb{E}^3}(\phi(D))}
	\]
	while $sgn(K(u))$ is the orientation of $\mathcal{G}$ at point $u$.
\end{remark}
