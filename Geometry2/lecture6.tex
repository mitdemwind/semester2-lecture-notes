%! TeX root = ./main.tex

We say $\phi(s,t_0)$ is an $s$-curve and $\phi(s_0,t)$ is a $t$-curve.
If $s$-curve and $t$-curve are orthogonal at every point $u\in U$,
i.e. $F = 0$, or the matrix is diagonal,
we say  $\phi$ is an \vocab{orthogonal parametrization},
and $s,t$ are \vocab{orthogonal parameters}.

Moreover, if $E=G, F=0$ for all  $u\in U$, (the matrix is a scalar)
then we call  $\phi$ an
\vocab{isothermal parametrization}, and $s,t$ are \vocab{isothermal parameters}.
(Sometimes also called \vocab{comformal parameters})

\begin{example}
    The longtitude and latitude on a sphere are orthogonal parameters,
	but not isothermal parameters;
	While the stereographical projection is an isothermal parametrization of the sphere.
\end{example}

\begin{remark}
    The word ``isothermal'' is connected to thermology in a rather complicated way.
	The word ``conformal'' provides a more intuitive comprehension.
\end{remark}
\begin{remark}
    A fun fact: Isothermal parameters always exist locally on
	regular parametrized surfaces. This only holds for 2-dimensional
	manifold.
\end{remark}

\subsection{Linear algebra review}
\label{sub:Linear algebra review}

Let $V$ be a vector space over $\mathbb{F}$.

\textbf{Symmetrical bilinear form vs. quadratic form}

A symmetrical bilinear form is a linear map $B: V\times V\to \mathbb{F}$
with $B(v,w)=B(w,v)$.
A quadratic form is a map $Q:V\to \mathbb{F}$ with $Q(v) = B(v,v)$ for
some symmetrical bilinear form  $B$.

By taking a basis of $V$, we can use the matrix to express them:
 \[
B(v,w) = vAw^T, \quad Q(v) = vAv^T.
\]
where $A$ is a symmetrical matrix.
When we change the basis, the matrix  $A$ differs by a congruent transformation.

We could also write $B\in V^*\otimes V^*$ for a bilinear form $B$.
All the symmetrical bilinear forms constitude a subspace of  $V^*\otimes V^*$
of dimension  $\frac{n(n+1)}{2}$. This is denoted by $\sym^2(V)$.
\begin{remark}
    The subspace of anti-symmetric matrices is denoted by $\alt^2(V)$,
	and  $\alt^2(V)\oplus \sym^2(V) = V^*\otimes V^*$.
\end{remark}

\subsection{Tangent spaces}
\label{sub:Tangent spaces}

A surface $\phi:U\to \mathbb{E}^3$ has a tangent space at every point $\phi(u)$,
which is just the space (in this case, a plane) spanned by $\phi_s(u)$ and $\phi_t(u)$.
We can prove that this tangent space is independent to the parameters.
Furthermore, we can equip it with the inner product in $\mathbb{E}^3$,
the matrix of this product is exactly $\begin{pmatrix}
	E &F \\ F &G
\end{pmatrix}$.

\begin{remark}
    In modern differential manifold theory, there's an intrinsic definition of
	tangent spaces, but this definition is too abstract.
\end{remark}

Here we present one of these intrinsic definitions.
\begin{definition}[Tangent vectors]
	Define an equivalence relation on smooth curves in $\phi(U)$:

	Let $\gamma(r) = \phi(s(r),t(r))$ be
	a smooth curve $(-\epsilon,\epsilon)\to \phi(U)$.
	Two curves $\gamma_1,\gamma_2$ are equivalent
	iff $s_1'(0)=s_2'(0)$ and $t_1'(0)=t_2'(0)$.

	Each equivalence class is a ``tangent vector'' at point $\phi(s_0,t_0)$.
\end{definition}
