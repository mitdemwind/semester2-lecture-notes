%! TeX root = ./main.tex
\begin{theorem}
    $\pi_1(S^1) \cong \mathbb{Z}$, where the generating element is $\id$.
\end{theorem}
\begin{proof}[Proof]
    Consider the map $p: \mathbb{R}\to S^1$, with $t\mapsto e^{2\pi it}$.

	Given any path $\gamma: [0,1]\to S^1$, we can find a unique
	path $\tilde \gamma: [0,1]\to \mathbb{R}$, s.t. $\tilde\gamma(0)\in \mathbb{Z}$ is
	any given base point.
	We denote this map by $\Phi$, $\gamma\mapsto \tilde\gamma(1)$,
	where we require $\tilde\gamma(0) = 0$.

	We can prove that $\Phi(\alpha\beta) = \Phi(\alpha)+\Phi(\beta)$,
	and $\Phi$ only depends on the homotopy class of $\gamma$,
	so $\Phi$ induces a homomorphism of $\pi_1(S^1) \to \mathbb{Z}$.
	\begin{remark}
		Since every homotopy $[0,1]\times [0,1]\to S^1$ can be lifted uniquely,
		and the endpoints of each path form a path in $\mathbb{R}$,
		but it's always contained in $\mathbb{Z}$, hence it must be constant.
	\end{remark}

	Note that
	\begin{itemize}
		\item $\Phi$ is surjective since $s\mapsto e^{2\pi i m s}$ is mapped
			to $m$ under $\Phi$;
		\item $\Phi$ is injective since $\ker \Phi = \{1\}$:
			if $\tilde\gamma(1) = 0$, then $\tilde\gamma \simeq const$,
			so $\gamma = p\circ \tilde\gamma \simeq const$.
	\end{itemize}

	So $\Phi$ is an isomorphism, $\pi_1(S^1) \cong \mathbb{Z}$.
\end{proof}

Next we'll prove Van Kampen theorem (\ref{thm:vankampen}).
In fact we only need to prove that:
\begin{claim}
	The map
	\[
	j'_\sharp * j''_\sharp : \pi_1(U', x_0) * \pi_1(U'', x_0)\to \pi_1(X, x_0)
	\]
	is a surjective homomorphism, and its kernel is the normal closure generated by
	$i'_\sharp(\delta)i''_\sharp(\overline{\delta})$.
\end{claim}

CLearly it's a group homomorphism.

For any $\gamma \in \pi_1(X, x_0)$, it can be decompose to $a_1b_1a_2\cdots a_kb_k$,
where $a_i \subset U', b_i \subset U''$, let the partition points be
$p_1, \dots, p_k, q_1, \dots, q_k\in W$,
and denote $s_i, t_i$ the path from $x_0$ to $p_i, q_i$.
So we have
\[
	\gamma = \underbrace{a_1\overline{s}_1}_{\in \pi_1(U', x_0)}
	\cdot \underbrace{s_1b_1\overline{t}_1}_{\in \pi_1(U'', x_0)} \cdots
\]
Thus $j'_\sharp * j''_\sharp$ is indeed surjective.

At last we'll study its kernel, let $\gamma\in \ker j'_\sharp * j''_\sharp$.
Since $\gamma \simeq \{x_0\}$,
say the homotopy is $H: [0,1]\times [0,1] \to U'\cup U''$.

We can partition $[0,1] \times [0,1]$ to many small cells such that
each cell's image is completely contained in either $U'$ or $U''$.

TODO
