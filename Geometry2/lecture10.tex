%! TeX root = ./main.tex
\begin{example}
    Consider the Gauss map of a torus, the ``outer'' part
	and the ``inner'' part of the torus maps to $\mathbb{S}^2$ bijectively.
	If we compute
	\[
	\int _{T^2} K \dd Area_E = \int _{\mathbb{S}^2} (1 + (-1))\dd Area_S
	= 0 = 2\pi \chi(T^2),
	\]
	as Gauss-Bonnet formula implies.
\end{example}

\subsection{Fundamental equation of surfaces}
\label{sub:Fundamental equation of surfaces}

Like the Fundamental theorem and Frenet frame in curve theory, we want to develop a
theorem for describing surfaces using only fundamental forms.

Given a parameter on a surface, there's a natural frame $(\phi_s, \phi_t, \vec{n})$.
If we take the derivative of the frame, we'll get
\[
	(\phi_s, \phi_t, \vec{n})_{st} = (\phi_s, \phi_t, \vec{n})_{ts}.
\]
Taking the inner product with $(\phi_s,\phi_t,\vec{n})^T$ and apply the product rule:
\[
\left( \begin{pmatrix}\phi_s\\\phi_t\\\vec{n}\end{pmatrix}
(\phi_s,\phi_t,\vec{n})_s \right)_t
- \begin{pmatrix}\phi_s\\\phi_t\\\vec{n}\end{pmatrix}_t
(\phi_s,\phi_t,\vec{n})_s
= \left( \begin{pmatrix}\phi_s\\\phi_t\\\vec{n}\end{pmatrix}
(\phi_s,\phi_t,\vec{n})_t \right)_s
- \begin{pmatrix}\phi_s\\\phi_t\\\vec{n}\end{pmatrix}_s
(\phi_s,\phi_t,\vec{n})_t
\]
This equation will give us some relations between the fundamental quantities.
In literature these relations are known as Gauss equation and Codazzi equations.

Gauss equation can be written as:
\[
	(\phi_s\cdot \phi_{ts})_t - (\phi_s\cdot \phi_{tt})_s
	= \phi_{st}\cdot \phi_{st} - \phi_{ss}\cdot \phi_{tt}.
\]
Codazzi equations are related to $\vec{n}$ and more complicated.

From Gauss equation we can deduce a famous theorem:
\begin{theorem}[Gauss' Theorema Egregium]
    The Gauss curvature $K$ is determined by the first fundamental form.
\end{theorem}
\begin{proof}[Proof]
    Note that $(\phi_s\cdot \phi_{ts})_t = \frac{1}{2}E_{tt}$, and
	$(\phi_s\cdot \phi_{tt})_s = (F_t - \frac{1}{2}G_s)_s = F_{ts} - \frac{1}{2}G_{ss}$.

	Suppose $\phi_{ss} = x\phi_s + y\phi_t + L\vec{n}$,
	then
	\[
		\frac{1}{2}E_s = \phi_s\cdot \phi_{ss} = Ex + Fy,\quad
		F_s - \frac{1}{2}G_t = \phi_t\cdot \phi_{ss} = Fx + Gy
	\]
	So $x,y$ is determined by $E,F,G$.

	Similarly, we get
	\[
	\begin{aligned}
		\phi_{ss} &= *\phi_s + *\phi_t + L\vec{n}\\
		\phi_{st} &= *\phi_s + *\phi_t + M\vec{n}\\
		\phi_{tt} &= *\phi_s + *\phi_t + N\vec{n}
	\end{aligned}
	\]
	where $*$ are determined by $E,F,G$.

	By Gauss equation, we get $* = -(LN - M^2) + *$,
	and $*$ is determined by $E,F,G$ and their partial derivatives.
\end{proof}
\begin{remark}
    The computation looks messy, but in modern mathematics,
	we have a systematic notation which is more simplified.
\end{remark}
