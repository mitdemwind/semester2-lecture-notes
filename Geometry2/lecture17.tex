%! TeX root = ./main.tex
In fact $\mathbb{C}P^2$ is a 4-dimensional closed manifold,
and it's also a 2-dimensional complex manifold.
$PSL(3, \mathbb{C})$ acts transitively on $\mathbb{C}P^2$.

\begin{example}
    We can fuse the edges of polygons to get manifolds:
	%% TODO: pictures
	By fusing together opposite edges of a square,
	we can get torus or Klein bottle.
\end{example}

We'll use the word ``fuse'' frequently in the future,
so here we'll make it clear what we mean by ``fusing'' things together.

\begin{definition}[Quotient maps]
	A continuous map $f: X\to Y$ is called a \vocab{quotient map},
	if it's surjective, and $\forall B \subset Y$,
	$f^{-1}(B)$ open $\implies B$ open.

	This is saying that the topology on $Y$ is the ``largest''
	topology (or quotient topology) while keeping $f$ continuous.
\end{definition}

So when we ``fusing'' things together, we're actually giving an
equivalence relation on the original space, and the result
is the quotient topology induced from the natural projection map.

Now we look at the elliptic curves again,
let $U = \mathbb{C} \backslash ([\lambda_1,\lambda_2]\cup [\lambda_3, \infty])$.
Let $X$ be the path end compactification of $U$, then $X\simeq S^1\times [0,1]$.

Let $X_1, X_2$ be two copies of $X$, and fusing the corresponding circles
at the end in the reversed direction, we'll get a torus without 4 points,
by adding $\lambda_1, \lambda_2, \lambda_3$ back we'll get $T^2\backslash\{pt\}$.

\begin{remark}
    The quotient topology may have some bad properties, like not being Hausdorff:
	Consider $\mathbb{R}^2\backslash\{(0,0)\}$ with connected vertical lines
	as equivalence class, then we'll get a line with 2 points at the origin,
	which is a typical non-Hausdorff space.
\end{remark}

A closed surface is a connected compact 2-dimensional manifold with no edges.
We have the following classification theorem:
\begin{theorem}
    All the closed surfaces must be homeomorphic to $nT^2 (n\ge 0)$ or $mP^2 (m\ge 1)$.
	Here $n$ is called the \vocab{genus} of orientable surfaces.

	$nT^2$ can be viewed as $S^2$ fused with $n$ handles (torus),
	and $mP^2$ can be viewed as $S^2$ fused with $m$ crosscaps (Mobius strip).
\end{theorem}
In this course we mainly talk about surfaces with triangulation, i.e.
we take it for granted that all surfaces has triangulation.

