%! TeX root = ./main.tex
\begin{proposition}
	Torsion can be represented in general parameter:
	\[
	\tors_\gamma(t) = \frac{(\gamma'(t)\times \gamma''(t))\cdot \gamma'''(t)}
	{\lVert \gamma'(t)\times \gamma''(t) \rVert^2 }.
	\]
\end{proposition}
\begin{remark}
    The torsion can be negative (while curvature is always non-negative),
	and it is only defined at the points where the curvature is nonzero.
\end{remark}

Note that the vectors $\vv{t},\vv{n},\vv{b}$ form a right-handed
orthonormal basis in $\mathbb{R}^3$, and it's called the curve
$\gamma(s)$'s \vocab {Frenet frame}.

The plane containing $\vv{n}$ and $\vv{b}$ is called \vocab{normal plane},
the plane containing $\vv{t}$ and $\vv{n}$ is called \vocab{osculating plane},
and the last plane which contains $\vv{t},\vv{b}$ is called \vocab{rectifying plane}.

The Frenet frame is not a fixed frame, it's moving with the point along the curve.
So we can compute its derivative (with respect to $s$, the arc length parameter):
\[
	(\vv{t}',\vv{n}',\vv{b}') = (\vv{t}, \vv{n}, \vv{b})\cdot
	\begin{pmatrix}
		0&-\kappa&0\\ \kappa&0&-\tau\\0&\tau&0
	\end{pmatrix}.
\]
\begin{example}
    When $\gamma$ lies on the surface of a sphere, assmue
	$\kappa>0$ on  $\gamma\big|_J$, then
	\[
		\left(\frac{1}{\kappa}\right)^2 +
		\left(\frac{1}{\tau}\left(\frac{1}{\kappa}\right)'\right)^2 = c^2.
	\]
	where $c$ is the radius of the sphere.
	\begin{proof}[Proof]
		Let $\vv{u} = \gamma(s) - p$,
		then $\vv{u}\cdot\vv{u} = c^2$.
		To get a relation of $\kappa$ and  $\tau$, we only need to
		represent $\vv{u}$ in terms of $\vv{t},\vv{n}$ and $\vv{b}$.

		Taking derivative WRT $s$:
		\[
		0 = 2\vv{u}'\cdot\vv{u} = 2\vv{t}\cdot\vv{u}.
		\]
		and then by taking the second and third derivative,
		\[
		0 = \vv{t}'\cdot\vv{u} + \vv{t}^2 = \kappa\vv{n} \cdot \vv{u} + 1.
		\]
		We get $\vv{u}\cdot\vv{n}  = -\frac{1}{\kappa}$.
		\[
			(\kappa\vv{n})' = \kappa'\vv{n} + \kappa(-\kappa\vv{t} + \tau\vv{b}),
		\]
		so the third derivative should be
		\[
		0 = \kappa\vv{n}\cdot\vv{t} +
		\kappa'\vv{n}\cdot\vv{u} +
		\kappa(-\kappa\vv{t} + \tau\vv{b})\cdot\vv{u}
		= -\frac{\kappa'}{\kappa} + \kappa\tau\vv{u}\cdot\vv{b},
		\]
		hence $\vv{u}\cdot\vv{b} = \frac{1}{\tau}(\frac{1}{\kappa})'$.

		At last we just plugged everything into $\vv{u}^2 = c^2$ to conclude.
	\end{proof}

	Note: the inverse statement does not hold, e.g. helix
	(which has constant curvature).
\end{example}

This example shows that Frenet frame is a powerful tool
for handling the local properties of a curve.
In fact, we could totally ``determine'' a curve near a point
given the curvature and torsion.

\begin{example}
    We can expand the curve $(\gamma(s), \vv{t},\vv{u},\vv{b})$ around $s=0$:
	\[
	\left\{\begin{aligned}
			x(s) &= x(0) + s - \frac{\kappa^2(0)}{6}s^3 + o(s^3)\\
			y(s) &= y(0) + \frac{\kappa(0)}{2}s^2 + \frac{\kappa^2(0)}{6}s^3 + o(s^3)\\
			z(s) &= z(0) + \frac{\kappa'(0)\tau(0)}{6}s^3 + o(s^3)
	\end{aligned}\right.
	\]
\end{example}
\begin{remark}
    By Frenet's formula, we can expand it to higher degrees, but the expansion
	need not converge to the original curve (similar reason as Taylor's formula).
	Also we can expand the curve with any parameter instead of arclength.
\end{remark}

\subsection{Fundamental theorem of curve theory}
\label{sub:Fundamental theorem of curve theory}

\begin{theorem}[Fundamental theorem of curve theory]
    Let $\kappa,\tau : J\to \mathbb{R}$ be smooth functions, $\kappa(s)>0$ on $J$.
	There exists a curve with arc length parameter $\gamma: J\to \mathbb{E}^3$,
	such that $\curv_\gamma = \kappa$, $\tors_\gamma = \tau$ holds
	on $J$.

	Moreover, if $\tilde \gamma$ also satisfies above conditions,
	then exists $\sigma: \mathbb{E}^3\to \mathbb{E}^3$ perserving orientation and
	distance s.t.  $\tilde\gamma = \sigma\circ\gamma$.
\end{theorem}
\begin{claim}
	Let $H = \begin{pmatrix}
		0&-\kappa&0\\\kappa&0&-\tau\\0&\tau&0
	\end{pmatrix}
	: J\to \Mat_{3 \times 3}(\mathbb{R})$.

	The ODE about $F: J\to \Mat_{3 \times 3}(\mathbb{R})$ :
	\[
	\left\{\begin{aligned}
			\frac{\dd F}{\dd s}(s) &= F(s)H(s)\\
			F(s_0) &= F_0\in \Mat_{3 \times 3}(\mathbb{R})
	\end{aligned}
	\right.
	\]
	always has unique solution.
	Moreover if $F(s_0)\in \SO(3)$, then $F(s)\in \SO(3)$ always holds.
\end{claim}
\begin{proof}[Proof of the theorem]
	Since this claim requires some knowledge in ODE, which is beyond the scope of
	this course, we'll directly use it without proving.

	WLOG $0\in J$, let  $s_0=0$ and $F(0)=I_3$.

	Let $\mathcal{F} = (\vv{t},\vv{n},\vv{b}) := (\vv{e}_1,\vv{e}_2,\vv{e}_3)F(s)$
	be a frame of $\mathbb{R}^3$.

	Now we construct $\gamma$ to be
	\[
	\gamma(s_1) := \int_{0}^{s_1} \vv{t}\dd s.
	\]
	It's sufficient to prove that $\curv_\gamma=\kappa$ and  $\tors_\gamma=\tau$.

	Since $\mathcal{F}(0)=(e_1,e_2,e_3)$ is orthonormal frame,
	$\mathcal{F}(s)$ is orthonormal for all $s$.

	Thus  $|\vv{t}| = 1$, $s$ is the arc length parameter.
	Some computation yields $\mathcal{F}$ is Frenet frame of $\gamma$.
	Compare its Frenet matrix to $H$, we get the desired result.

	On the other hand, if  $\tilde\gamma$ is as stated,
	take its Frenet frame $\tilde {\mathcal{F}}(s)$.

	Let $\sigma$ be the map which maps $\mathcal{F}(0)$ to $\tilde {\mathcal{F}}(0)$,
	$\gamma(0)$ to $\tilde\gamma(0)$.
	Then the Frenet frame of $\sigma\circ\gamma$ and  $\tilde\gamma$
	are the solution of the same ODE
	$\implies \sigma\circ\gamma=\tilde\gamma$ for all $s\in J$.
\end{proof}
