\usepackage{amsmath,amssymb,amsthm}
\usepackage{mathrsfs}
\usepackage{enumerate}
\usepackage[usenames,svgnames,dvipsnames,hyperref,table]{xcolor}
\usepackage{hyperref}
\usepackage[margin = 1.3 in]{geometry}
\usepackage{titlesec}
\usepackage{tikz-cd}
%%----section headers----
\titleformat{\section}{\sffamily\bfseries\Large\centering}
{\color{purple}\S\thesection\enskip}{0em}{}

\titleformat{\subsection}{\sffamily\bfseries\large}
{\color{purple}\S\thesubsection\enskip}{0em}{}

\titleformat{\subsubsection}{\sffamily\bfseries}
{\color{purple}\S\thesubsubsection\enskip}{0em}{}

\hypersetup{
	colorlinks=true,
	linkcolor=RoyalBlue,
	urlcolor=Cyan!80!Black,
}
\urlstyle{same}

%%----page formats----
\usepackage{fancyhdr}
\pagestyle{fancy}

\rhead{\footnotesize \normalfont\leftmark}
\chead{}
\cfoot{\thepage}

%%----theorem environments----

%%fakesection New Theorem Styles
\usepackage{amsthm}
\usepackage{thmtools}

\usepackage[framemethod=TikZ]{mdframed}
\mdfdefinestyle{mdbluebox}{%
	roundcorner=10pt,
	linewidth=1pt,
	skipabove=12pt,
	innerbottommargin=9pt,
	skipbelow=2pt,
	linecolor=blue,
	nobreak=true,
	backgroundcolor=TealBlue!6,
}
\declaretheoremstyle[
headfont=\sffamily\bfseries\color{MidnightBlue},
mdframed={style=mdbluebox},
headpunct={\\[3pt]},
postheadspace={0pt}
]{thmbluebox}
\mdfdefinestyle{mdredbox}{%
	linewidth=0.5pt,
	skipabove=12pt,
	frametitleaboveskip=5pt,
	frametitlebelowskip=0pt,
	skipbelow=2pt,
	frametitlefont=\bfseries,
	innertopmargin=4pt,
	innerbottommargin=8pt,
	nobreak=true,
	backgroundcolor=Salmon!5,
	linecolor=RawSienna,
}
\declaretheoremstyle[
headfont=\bfseries\color{RawSienna},
mdframed={style=mdredbox},
headpunct={\\[3pt]},
postheadspace={0pt},
]{thmredbox}
\mdfdefinestyle{mdgreenbox}{%
	skipabove=8pt,
	linewidth=2pt,
	rightline=false,
	leftline=true,
	topline=false,
	bottomline=false,
	linecolor=ForestGreen,
	backgroundcolor=ForestGreen!5,
}
\declaretheoremstyle[
headfont=\bfseries\sffamily\color{ForestGreen!70!black},
bodyfont=\normalfont,
spaceabove=2pt,
spacebelow=1pt,
mdframed={style=mdgreenbox},
headpunct={ --- },
]{thmgreenbox}
\mdfdefinestyle{mdblackbox}{%
	skipabove=8pt,
	linewidth=3pt,
	rightline=false,
	leftline=true,
	topline=false,
	bottomline=false,
	linecolor=black,
	backgroundcolor=RedViolet!5!gray!5,
}
\declaretheoremstyle[
headfont=\bfseries,
bodyfont=\normalfont\small,
spaceabove=0pt,
spacebelow=0pt,
mdframed={style=mdblackbox}
]{thmblackbox}
\newcommand{\listhack}{$\empty$\vspace{-2em}}

%%fakesection Theorem setup

\theoremstyle{definition}
%Branching here: the option secthm changes theorems to be labelled by section

\declaretheorem[%
style=thmbluebox,name=Theorem,numberwithin=subsection]{theorem}

\declaretheorem[style=thmbluebox,name=Lemma,sibling=theorem]{lemma}
\declaretheorem[style=thmbluebox,name=Proposition,sibling=theorem]{proposition}
\declaretheorem[style=thmbluebox,name=Corollary,sibling=theorem]{corollary}
\declaretheorem[style=thmbluebox,name=Theorem,numbered=no]{theorem*}
\declaretheorem[style=thmbluebox,name=Lemma,numbered=no]{lemma*}
\declaretheorem[style=thmbluebox,name=Proposition,numbered=no]{proposition*}
\declaretheorem[style=thmbluebox,name=Corollary,numbered=no]{corollary*}

\declaretheorem[style=thmgreenbox,name=Algorithm,sibling=theorem]{algorithm}
\declaretheorem[style=thmgreenbox,name=Algorithm,numbered=no]{algorithm*}
\declaretheorem[style=thmblackbox,name=Claim,sibling=theorem]{claim}
\declaretheorem[style=thmblackbox,name=Claim,numbered=no]{claim*}

\declaretheorem[style=thmredbox,name=Example,sibling=theorem]{example}
\declaretheorem[style=thmredbox,name=Example,numbered=no]{example*}

% Remark-style theorems
%\theoremstyle{remark}

\declaretheorem[style=thmgreenbox,name=Remark,sibling=theorem]{remark}
\declaretheorem[style=thmgreenbox,name=Remark,numbered=no]{remark*}

\declaretheorem[name=Conjecture,sibling=theorem]{conjecture}
\declaretheorem[name=Conjecture,numbered=no]{conjecture*}
\declaretheorem[name=Definition,sibling=theorem]{definition}
\declaretheorem[name=Definition,numbered=no]{definition*}
\declaretheorem[name=Exercise,sibling=theorem]{exercise}
\declaretheorem[name=Exercise,numbered=no]{exercise*}
\declaretheorem[name=Fact,sibling=theorem]{fact}
\declaretheorem[name=Fact,numbered=no]{fact*}
\declaretheorem[style=thmblackbox,name=Problem,sibling=theorem]{problem}
\declaretheorem[style=thmblackbox,name=Problem,numbered=no]{problem*}
\declaretheorem[name=Question,sibling=theorem]{ques}
\declaretheorem[name=Question,numbered=no]{ques*}
%\Crefname{claim}{Claim}{Claims}
%\Crefname{conjecture}{Conjecture}{Conjectures}
%\Crefname{exercise}{Exercise}{Exercises}
%\Crefname{fact}{Fact}{Facts}
%\Crefname{problem}{Problem}{Problems}
%\Crefname{ques}{Question}{Questions}

%%----Asymptote definitions----
\usepackage{asymptote}
\begin{asydef}
    import olympiad;
	import cse5;
	size(6cm);
	usepackage("amsmath");
	usepackage("amssymb");
	defaultpen(fontsize(11pt));
	settings.tex="latex";
	settings.outformat="pdf";
\end{asydef}
\def\asydir{asy}
