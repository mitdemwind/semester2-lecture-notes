%! TeX root = ./main.tex
\subsection{Primary cyclic decomposition and Jordan canonical forms}
\label{sub:Primary cyclic decomposition}

\begin{theorem}
    For $T\in L(V)$, $T$ irreducible $\iff T$ is primary and cyclic.
\end{theorem}
\begin{proof}[Proof]
    If $T$ is irreducible, then both the primary and cyclic decomposition
	have only one term, i.e. $T$ is primary and cyclic.

	Conversely, if $V = V_1\oplus V_2$ is a nontrivial decomposition.
	Since $T$ is cyclic and primary, assume $f_T = p_T = p^r$,
	where $p$ is a irreducible polynomial.

	Suppose $f_{T_1} = p^s, f_{T_2} = p^t$, then $s + t = r$, $s,t < r$.
	Since $p_{T_1}\mid p^s, p_{T_2}\mid p^t$,
	\[
	p_T = \lcm(p_{T_1}, p_{T_2}) \mid p^{\max\{s, t\}},
	\]
	contradiction!
\end{proof}

\begin{theorem}[Primary cyclic decomposition]
    Let $T\in L(V)$.
	\begin{itemize}
		\item There exists a decomposition $V = \bigoplus_{i=1}^s V_i$,
			each $V_i$ is $T$-invariant, $T_{V_i}$ primary and cyclic.
			Let $q_i = p_{T_{V_i}}$.
		\item $q_1,\dots,q_s$ are uniquely determined by $T$ (ignoring the permutation).
			They are called the \vocab{elementary divisors} of $T$.
	\end{itemize}
\end{theorem}
\begin{proof}[Proof]
    Existence follows immediately from the previous theorem.

	Uniqueness: Let $V = \bigoplus_{i=1}^t W_i$ s.t. $T_{W_i}$ is primary and cyclic.
	Let $\{u_1,\dots, u_k\}$ be the set of all the monic prime factors of
	the minimal polynomials of $T_{W_1},\dots,T_{W_t}$.

	We can group $W_i$'s by $u_i$, and each group can be placed in a row
	in descending order wrt the degree of $p_{T_{W_i}}$.

	Let $Z_j$ be the direct sum of the $j$-th column, note that $Z_j$ is
	a cyclic decomposition of $T$.

	Now since the cyclic decomposition and primary decomposition are unique,
	$p_{T_{W_i}}$'s must be unique as well.
\end{proof}
\begin{remark}
    The elementary factors depend on the base field.
\end{remark}

Since the invariant subspaces of primary subspace are primary,
and invariant subspaces of cyclic subspace are cyclic,
we can apply both decomposition (in any order) to get the primary
cyclic decomposition of any operators.

For a primary cyclic map $T$, if we choose the base field to
be \textit{algebraically closed} (e.g. $\mathbb{C}$), we can write
$f_T = p_T = (x-c)^n$.
Let $N = T - c\id_V$, then $f_T = p_T = x^n$,
from rational canonical form we know that $N$ is similar to
$\begin{pmatrix}
	0 &0 \\ I_{n-1} &0
\end{pmatrix}$.
Hence $T$ is similar to
 \[
J_n(c) :=
\begin{pmatrix}
	c \\ 1 &c \\ &1 &\ddots \\ &&\ddots &c \\ &&&1 &c
\end{pmatrix},
\]
such matrix is called a \vocab{Jordan block}.
Jordan matrices are the blocked diagonal matrices with
each block being a Jordan block.

\begin{theorem}[Jordan canonical forms]
    If $f_T$ can be decompose to product of polynomials of degree 1, then
	\begin{itemize}
		 \item $\exists \mathcal{B}$ s.t. $[T]_{\mathcal{B}}$ is a Jordan matrix,
			 this is called the \vocab{Jordan canonical form} of $T$.
		\item The canonical form is unique under permutations of each Jordan blocks.
	\end{itemize}
\end{theorem}
\begin{proof}[Proof]
    This follows immediately from the primary cyclic decomposition of $T$.
\end{proof}
