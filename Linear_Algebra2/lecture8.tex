%! TeX root = ./main.tex
\subsection{Cyclic decomposition}
\label{sub:Cyclic decomposition}
In the following contents we'll assmue $R = F[x]$ if it's
not specified.

\begin{definition}[Cyclic maps]
	Let $V$ be a finite dimensional vector space and $T\in L(V)$.
	For $\alpha\in V$, $R\alpha = \{f\alpha \mid f\in R\}=
	\Span\{\alpha, T\alpha, \dots\}$ is the smallest $T$-invariant subspace
	containing $\alpha$.

	We say $T$ is \vocab{cyclic} if $\exists \alpha$ s.t. $V = R\alpha$.
	In this case $\alpha$ is called a \vocab{cyclic vector}.

	Here $R\alpha$ is called the cyclic subspace spanned by $\alpha$.
\end{definition}
\begin{remark}
    The word ``cyclic'' comes from the theory of modules.
\end{remark}

Note that $\dim R\alpha = 1$ $\iff \alpha$ is an eigenvector.

\begin{example}
    Let $A = E_{21} \in F^{2\times 2}$.
	Then $A$ is cyclic because $A\varepsilon_1 = \varepsilon_2$,
	$A\varepsilon_2 = 0$. This means $\varepsilon_1$ is a cyclic vector of $A$,
\end{example}

Now there's a natural question: When is $T$ cyclic and how to find its
cyclic vectors?

For a given vector $\alpha$, let $M_\alpha = \{f\in R\mid f\alpha = 0\}$ is
an ideal of $R$.

Note that $M_T \subset M_\alpha$ as $f\in M_T \implies f(T)\alpha = 0$,
so $M_\alpha$ is nonempty, it has a generating element $p_\alpha$,
called the \vocab{annihilator} of $\alpha$.

\begin{proposition}
	Let $d = \deg p_\alpha$, then $\{\alpha, T\alpha, \dots, T^{d-1}\alpha\}$ is
	a basis of $R\alpha$.
	In particular, $\dim R\alpha = \deg p_\alpha$.
\end{proposition}
\begin{proof}[Proof]
    Linear independence:

	If $\sum_{i=0}^{d - 1} c_iT^i\alpha = 0$, let $g = \sum_{i=0}^{d - 1} c_ix^1$.
	\[
	g\alpha = 0 \implies g \in M_\alpha \implies p_\alpha \mid g.
	\]
	But $\deg g \le d - 1 < d = \deg p_\alpha \implies g = 0$.

	Spanning:

	Clearly $T^i \alpha \in R\alpha$. $\forall f\in R$,
	let $f = qp_\alpha + r$ with $\deg r < \deg p_\alpha$.
	Hence $f\alpha = r\alpha \in \Span\{\alpha, T\alpha, \dots, T^{d - 1}\alpha\}$.
\end{proof}

Since $\alpha$ is a cyclic vector $\iff \dim R\alpha = \dim V$,
and $\deg p_\alpha \le \deg p_T \le \deg f_T = \dim V$,
so we care whether these two inequalities can attain the equality.

\begin{proposition}
	There exists $\alpha\in V$ s.t. $p_\alpha = p_T$.
\end{proposition}
\begin{proof}[Proof]
    Let $p_T = \prod_{i = 1}^k p_i^{r_i}$.
	\[
	W_i = \ker(p_i^{r_i}(T)) \implies V = \bigoplus_{i = 1}^k W_i.
	\]
	We claim that $\ker(p_i^{r_i - 1}) \subsetneq W_i$ as
	$p_{T_{W_i}} = p_i^{r_i}$.

	Take a vector $\alpha_i\in W_i \backslash \ker(p_i^{r_i - 1}(T))$.
	By definition $p_{\alpha_i} \mid p_i^{r_i}, p_{\alpha_i} \nmid p_i^{r_i - 1}
	\implies p_\alpha = p_i^{r_i}$.

	Let $\alpha = \sum_{i = 1}^{k} \alpha_i$. If $f\alpha = 0$,
	then $f\alpha_i = 0$ for $i = 1,\dots,k$ as $f\alpha_i \in W_i$.
	\[
	f\alpha_i = 0\implies p_{\alpha_i} \mid f \implies p_T \mid f.
	\]
	This means we must have $p_\alpha = p_T$.
\end{proof}

Now we come to a conclusion:
\begin{corollary}
    $T$ is cyclic $\iff$ $\deg p_T = \dim V \iff p_T = f_T$.

	In this case, $\alpha$ is a cyclic vector $\iff p_\alpha = p_T$.
\end{corollary}

Let $n = \dim V$, $T$ be a cyclic map, $\alpha$ be a cyclic vector.
By previous proposition, $\{\alpha, T\alpha, \dots, T^{n - 1}\alpha\}$ is
a basis of $V$. Denote the basis by $\mathcal{B}$.

Observe that $[T]_{\mathcal{B}}$ is equal to
\[
A = \begin{pmatrix}
	0 &0 &0 &\cdots &0 &-c_0\\
	1 &0 &0 &\cdots &0 &-c_1\\
	0 &1 &0 &\cdots &0 &-c_2\\
	\vdots &\vdots &\vdots &\ddots &\vdots &\vdots\\
	0 &0 &0 &\cdots &0 &-c_{n - 2}\\
	0 &0 &0 &\cdots &1 &-c_{n-1}
\end{pmatrix}
\]
where $c_i$ are the coefficients of $p_\alpha = p_T = f_T = \sum_{i=0}^{n} c_ix^i$.
For a monic polynomial $f$, define $C_f$ to be the matrix as above,
called the \vocab{companion matrix} of $f$.

\begin{proposition}
	If exists a basis $\mathcal{B}$ s.t. $[T]_{\mathcal{B}} = C_f$ for some
	monic polynomial $f$, then $T$ is cyclic and $p_T = f$.
\end{proposition}
\begin{proof}[Proof]
    Let $\mathcal{B} = \{\alpha_1,\dots,\alpha_n\}$,
	we have $T^i \alpha_1 = \alpha_{i + 1} \implies R\alpha_1 = V$ and
	$p_{\alpha_1} = f$.
\end{proof}

\begin{remark}
    In fact we can check directly that $f$ is the
	characteristic polynomial of $C_f$.

	This gives another proof of Cayley-Hamilton theorem: 
	\begin{proof}[Proof]
	    For any $\alpha\in V$, consider $T_{R\alpha}$:
		\[
		f_{T_{R\alpha}} = f_{C_{p_\alpha}} = p_\alpha \mid f_T
		\]
		This implies that $f_T$ is an annihilating polynomial of $\alpha$,
		which means $f_T(\alpha) = 0, \forall \alpha\in V$,
		i.e. $f_T(T) = 0$.
	\end{proof}
\end{remark}

\begin{theorem}[Cyclic decomposition]
    Let $T\in L(V)$, $\dim V = n$. There exists $\alpha_1,\dots,\alpha_r\in V$
	s.t. $V = \bigoplus_{i=1}^r R\alpha_i$.

	Furthermore, $p_{\alpha_r}\mid\dots\mid p_{\alpha_1} = p_T$,
	$f_T = \prod_{i=1}^r p_{\alpha_i}$.

	Here $p_{\alpha_i}$'s are called the \vocab{invariant factors} of $T$.
	The invariant factors are \textit{totally determined} by $T$.
\end{theorem}
