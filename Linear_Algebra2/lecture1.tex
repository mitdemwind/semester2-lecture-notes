\section{Introduction}
\label{sec:Introduction}
\begin{center}
	\sffamily\large\bfseries
	Teacher: An Jinpeng

	Homepage: \url{https://www.math.pku.edu.cn/teachers/anjp/algebra}
\end{center}

\subsection{recap}
\label{sub:recap}
\paragraph{Direct sums of vector spaces}
Given a field $F$, let $V_1,\dots,V_k$ be vector spaces over $F$.
The set
 \[
	V_1 \times \dots \times V_k = \{(v_1,\dots,v_k)\mid v_i\in V_i\}
\]
forms a vector space by the operations
\[
	(v_1,\dots,v_k) + (w_1,\dots,w_k) = (v_1+w_1,\dots,v_k+w_k)
\]
and
\[
c\cdot (v_1,\dots,v_k) = (cv_1,\dots,cv_k).
\]
We call this vector space the \vocab{external direct sum} of $V_1,\dots,V_k$,
denoted by $\bigoplus_{i=1}^k V_i$.

Obviously  $(U\oplus V)\oplus W\simeq U\oplus (V\oplus W)$.

For every $i$, we have an injective linear map:
 \begin{align*}
	\tau_i: &V_i \to \bigoplus_{j=1}^k V_j\\
	&v_i \mapsto (0,\dots,v_i,\dots,0)
\end{align*}
\begin{lemma}
	If $\mathcal{B}_i$ are the bases of $V_i$, then
	$\bigcup_{i=1}^k \tau_i(\mathcal{B}_i)$ is
	a basis for  $\bigoplus_{i=1}^k V_i$.

	In paricular,
	\[
		\dim \bigoplus_{i=1}^k V_i = \sum_{i=1}^k \dim V_i.
	\]
\end{lemma}
\begin{proof}[Proof]
    Spanning part:

	For any $(v_1,\dots,v_k)\in \bigoplus_{i=1}^kV_i$,
	\[
		v_i\in V_i = \Span(\mathcal{B}_i)\implies
		\tau_i(v_i)\in \Span\left(\tau_i(\mathcal{B})_i\right)\implies
		(v_1,\dots,v_k)\in \Span\left(\bigcup_{i=1}^k \tau_i(\mathcal{B}_i)\right)
	\]

	Linearly independent part:

	If $\bigcup_{i=1}^k \tau_i(\mathcal{B}_i)$ is linearly dependent,
	i.e. exists $e_{ij}\in \mathcal{B}_i$ satisfying $\exists c_{ij}\in F$,
	\[
		\sum_{i,j} c_{ij} \tau_i(e_{ij}) = 0.
	\]
	This expands to
	\[
		\left( \sum_{j=1}^{m_1} c_{1j}e_{1j}, \dots, \sum_{j=1}^{m_k} c_{kj}e_{kj} \right)=0.
	\]
	but $e_{1j}$ are linear independent, which implies $c_{1j}=0$.
\end{proof}

\begin{remark}
    Let $V$ be a vector space over $F$, and $V_1,\dots,V_k$ are subspaces of $V$.

	Consider a linear map  $\Phi: V_1\oplus \dots\oplus V_k\to V$ by
	$(v_1,\dots,v_k)\mapsto v_1+\dots+v_k$.

	Then $\Img(\Phi) = V_1+\dots+V_k$.
	If $\Phi$ is injective, i.e.  $V_1,\dots,V_k$ are independent, we say
	$V_1+\dots+V_k$ the \vocab{internal direct sum} of  $V_1,\dots,V_k$.

	In this case $\Phi$ gives an isomorphism of external and internal sums:
	\[
		\Phi: \bigoplus_{i=1}^k V_i \xrightarrow{\sim} \sum_{i=1}^k V_i.
	\]
\end{remark}

\begin{lemma}
	The following statements are equivalent:
	\begin{enumerate}
		\item $V_1,\dots, V_k$ are independent;
		\item For $v_i\in V_i, (i=1,\dots,k)$, if $\sum_{i=1}^kv_i =0$, then  $v_i=0$.
		\item For any  $1\le i\le k$,  $V_i\cap (V_1+\dots+V_{i-1})=\{0\}$.
		\item Given arbitary bases  $\mathcal{B}_i$ of $V_i$,
			they are disjoint and their union is a basis of $\bigoplus_{i=1}^k V_i$.
		\item If $\dim V<+\infty$, they are also equivalent to:
		\[
			\dim \sum_{i=1}^{k} V_i = \sum_{i=1}^{k} \dim V_i.
		\]
	\end{enumerate}
\end{lemma}

\begin{proof}[Proof]
    It's easy but verbose so I leave it out.
\end{proof}

\begin{example}
	Let $\Char F \ne 2$,  $V=F^{n \times n}$, $V_1 = \{A\in V\mid A^t=A\}$,
	$V_2 = \{A\in V\mid A^t = -A\}$.

	Note that $V_1\cap V_2 = \{0\}$, and $V_1+V_2 = V$,
	hence $V_1\oplus V_2=V$ is an internal direct sum.
\end{example}

\section{Diagonization}
\label{sec:Diagonalization}

Example: google page rank?

Given a linear map $T$, it can be represented as different matrices
under different bases. Thus a question arises: What's the simpliest
matrix representation of a linear map?

\begin{definition}[Diagonizable maps]
	Let $V$ be a vector space over $F$,
	$T\in L(V)$ is a linear map from $V$ to itself.
	If the matrix of $T$ under a certain basis is diagonal, we say
	$T$ is \vocab{diagonizable}.
\end{definition}

In this case the linear map $T$ can be simply described as a diagonal matrix,
thus we'll study under what condition is $T$ diagonizable.

\subsection{Eigen-things}
\label{sub:Eigen-things}

\begin{definition}[Eigenvalue]
	Let $T: V\to V$ be a linear map, for $c\in F$,
	let
	\[
		V_c= \{v\in V\mid Tv = cv\} = \ker (T- c\cdot \id_V).
	\]
	If $V_c\ne \{0\}$, we call $c$ an \vocab{eigenvalue} of $T$,
	and $V_c$ the \vocab{eigenspace} of $T$ with respect to $c$.
	the vectors in $V_c$ are called \vocab{eigenvectors}.

	All the eigenvalues of $T$ are called the \vocab{spectrum} of $T$,
	denoted by  $\sigma(T)$.
\end{definition}

\begin{proposition}
	Let $\mathcal{B}$ be a basis of $V$, then  $[T]_{\mathcal{B}}$ is
	diagonizable $\iff$ vectors in $\mathcal{B}$ are all eigenvectors.
\end{proposition}
\begin{proof}[Proof]
	Let $\mathcal{B}=\{e_1,\dots,e_k\}$, $A = [T]_{\mathcal{B}}$.
	\[
	Te_j = \sum_{i=1}^k A_{ij} e_i.
	\]
	So $A$ is diagonal $\iff$ $A_{ij}=0$ when $i\ne j$,

	$\iff \exists c_j\in F, Te_j = c_j e_j$,

	$\iff$ all the vectors  $e_j$ are eigenvectors.
\end{proof}
\begin{example}
    Let $V=F^{n\times n}$, then $V_{sym}$ is the eigenspace of $1$,
	and  $V_{antisym}$ is the eigenspace of $-1$.
\end{example}

\begin{lemma}
	\label{lem:eigenpoly}
	Let $T$ be a linear operator, then
	\[
	\sigma(T) = \{c\in F\mid \det (c\cdot id_V - T) = 0\}.
	\]
\end{lemma}
\begin{proof}[Proof]
    $V_c=\ker (c\cdot \id_V - T)$,
	\[
	c\in \sigma(T) \iff V_c \ne \{0\} \iff \det(c\cdot \id_V -T) = 0.
	\]

\end{proof}
