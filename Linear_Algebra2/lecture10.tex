%! TeX root = ./main.tex
So far we've proved that $A\sim B \iff A,B$ have the same
rational canonical form. Note that this canonical form
does not require any extra properties of the base field $F$.

Next we'll see some applications of it.
Different from Jordan canonical forms, rational canonical forms
focus more on theory than computaion.

\begin{proposition}[Rational canonical forms don't depend on fields]
	Let $A\in F^{n\times n}$ has rational canonical form $A'$,
	and the invariant factors are $p_1,\dots,p_r\in F[x]$.

	If $K \subset F$ is a smaller field s.t. $A\in K^{n\times n}$,
	then $A'$ is still the rational canonical form of $A$ in $K$.
	i.e. $A'\in K^{n\times n}$, and $\exists P\in K^{n\times n}, A' = PAP^{-1}$.
\end{proposition}
\begin{proof}[Proof]
    Let $A''$ be the rational form of $A$ on $K$.
	By the uniqueness of rational canonical forms,
	we must have $A' = A''$, since they are both the
	rational form of $A$ on $F$.
\end{proof}

\begin{proposition}[Similarity in larger fields implies similarity in smaller fields]
	Let $A,B$ be matrices on $F$, and $A\sim B$ in $F$.
	If $A,B\in K^{n\times n}$, where $K$ is a subfield of $F$,
	then $A\sim B$ in $K$ as well.
\end{proposition}
\begin{proof}[Proof]
    Let $C$ be the rational canonical form of $A,B$,
	since $A,B\in K^{n\times n}$, by the previous proposition,
	$C\in K^{n\times n}$ and $A\sim C\sim B$ in $K$.
\end{proof}

\begin{proposition}
	$\forall A\in F^{n\times n}$, $A\sim A^t$.
\end{proposition}
\begin{proof}[Proof]
	Firstly when $A = C_f$ for some $f\in F[x]$,
	$A$ has only one invariant factor $f$.
	Note that $f_{A^t} = p_{A^t} = f_A = p_A = f$, so
	the invariant factor of $A^t$ is also $f$,
	by rational canonical forms we're done.

	Next for generic matrix $A$,
	just take the rational canonical form $B$.
	By above we have
	\[
	A \sim B\implies A\sim B\sim B^t \sim A^t.
	\]
\end{proof}

\begin{example}[How to compute the rational canonical forms (in low dimensions)]
    Let $A = \begin{pmatrix}
		5 &-6 &-6\\ -1 &4 &2 \\ 3 &-6 &-4
    \end{pmatrix}\in \mathbb{Q}^{3\times 3}$.
	First observe that $f_A = (x-1)(x-2)^2$.

	Since $(x-1)(x-2)$ is the minimal polynomial of $A$,
	so the invariant factors are $p_1 = (x-1)(x-2), p_2 = (x-2)$.
	Hence the rational canonical form of $A$ is
	\[
	\begin{pmatrix}
		0 &-2 &0\\ 1 &3 &0\\ 0 &0 &2
	\end{pmatrix}
	\]

	Next we'll find vectors $\alpha_1,\alpha_2$ s.t. $p_{\alpha_i} = p_i$.
	So $P = (\alpha_1, A\alpha_1, \alpha_2)$ will be the transition matrix.
\end{example}

\begin{proposition}
	Let $T$ be a diagonalizable map, $\sigma(T)=\{c_1,\dots,c_k\}$.
	Let $V_1,\dots,V_k$ be the primary decomposition of $V$,
	\begin{itemize}
		\item Let $\alpha = \sum_{i=1}^{k} \beta_i, \beta_i\in V_i$,
			then $R\alpha = \Span\{\beta_1,\dots,\beta_k\}$,
			$p_\alpha = \prod_{\beta_i\ne 0}(x-c_i) $.
		\item Let $d_i=\dim V_i$, then $p_j = \prod_{d_i\ge j}(x-c_i)$.
	\end{itemize}
\end{proposition}
