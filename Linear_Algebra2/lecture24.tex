%! TeX root = ./main.tex
Here we'll present multiple proofs to emphasize some
intermediate result.

\begin{proposition}
	Let $T$ be a normal map, if $W \subset V$ is $T$-invariant,
	then $T_W$ is also normal.
\end{proposition}
\begin{proof}[Proof]
    First note that $W, W^\perp$ are $T^*$-invariant.
	For $\alpha, \beta\in W$, we have
	\[
	\left<(T_W)^*\alpha, \beta \right> = \left<\alpha, T_W\beta \right>
	= \left<\alpha, T\beta \right> = \left<T^*\alpha, \beta \right>.
	\]
	Thus $(T_W)^* = T^*_W$. The conclusion follows.
\end{proof}

\begin{proposition}
	Let $T$ be a normal map, there exists an orthogonal decomposition
	$V = \bigoplus_{i=1}^k V_i$, such that each $V_i$ is $T$-invariant,
	and $T_{V_i}$ simple.
\end{proposition}
\begin{proof}[Proof]
    Note that if $W$ is $T$-invariant, then $W^\perp$ is
	also $T$-invariant.
	By induction and the previous proposition this is trivial.
\end{proof}

Therefore to prove \autoref{thm:normaldecom}, we only need to prove the case
when $T$ is simple.

\begin{proof}[Proof of \autoref{thm:normaldecom}]
    WLOG $\dim V > 1$.

	Since $T$ simple $ \implies f_T\in \mathbb{R}[x]$ prime,
	thus $\deg f_T = 2$, $\dim V = 2$ and $f_T = (x - c)(x - \overline{c})$.

	Take any orthonormal basis $\mathcal{B} = \{\alpha_1, \alpha_2\}$,
	let $r = |c|$, $A = r^{-1}[T]_{\mathcal{B}}$.
	Clearly $A$ normal and $\sigma(A) = \{r^{-1}c, r^{-1}\overline{c}\}$,
	so $A$ is unitarily similar to $\diag(r^{-1}c, r^{-1}\overline{c})$,
	$A$ is unitary.

	Moreover $A$ is a real matrix so $A$ orthogonal, and $\det A = 1$,
	thus $A = Q_\theta, \theta\in [0, 2\pi]$.

	At last by $T$ has no eigenvector, and we can change $\alpha_2$ to
	$-\alpha_2$, so we can require $\theta \in (0, \pi)$.
\end{proof}

\begin{proposition}
	Let $T\in L(V)$, then $\ker(T)^\perp = \img(T^*), \img(T)^\perp = \ker(T^*)$.
\end{proposition}
\begin{proof}[Proof]
    Trivial, just some computation.
\end{proof}

\begin{proposition}
	Let $T\in L(V)$, $\sigma(T^*) = \overline{\sigma(T)}$,
	\[
		\forall c\in \sigma(T), \quad
		\dim\ker(T - cI) = \dim\ker(T^* - \overline{c}I).
	\]
\end{proposition}
\begin{proof}[Proof]
    By the previous proposition,
	\[
	\dim \ker(T - cI) = n - \dim\img(T^* - \overline{c}I)
	= \dim\ker(T^* - \overline{c}I)
	\]
	which also implies $\sigma(T) = \overline{\sigma(T^*)}$.
\end{proof}
\begin{proposition}
	If $T$ normal, then $\ker(T - cI) = \ker(T^* - \overline{c}I)$.
\end{proposition}
\begin{proof}[Proof]
    Let $W = \ker(T - cI)$, $T_W^*$ is just $(c\id_W)^* = \overline{c}\id_W$.
	Thus $W \subset \ker(T^* 0 \overline{c}I)$, by dimensional reasons they
	must be equal.
\end{proof}

\begin{proposition}
	Let $T$ be a normal map, $f, g\in F[x]$ coprime
	$\implies \ker(f(T))\perp \ker(g(T))$.
\end{proposition}
\begin{proof}[Proof]
    Since $g(T)^* = \overline{g}(T^*)$, $g(T)$ is normal,
	thus $\ker(g(T))^\perp = \img(g(T))$.

	Let $W = \ker(f(T))$, let $a, b\in F[x]$ s.t. $af+bg = 1$,
	so $a(T)f(T) + b(T)g(T) = \id_V$.
	Restrict this equation to $W$, we get $b(T)_W g(T)_W = \id_W$,
	hence  $W \subset \img(g(T))$.
\end{proof}

\begin{proposition}
	Let $T$ be a normal map,
	\begin{itemize}
		\item The primary decomposition of $T$ are orthogonal decomposition;
		\item The cyclic decomposition of $T$ can be orthogonal.
	\end{itemize}
\end{proposition}
\begin{proof}[Proof]
    The first one is trivial by previous proposition.

	For cyclic decomposition, we proceed by induction on $\dim V$.

	Let $\alpha_1\in V$ s.t. $p_{\alpha_1} = p_r$,
	then $(R\alpha_1)^\perp$ are $T$-invariant, use induction hypo
	on it and we're done.
\end{proof}

\begin{remark}
    This means the primary cyclic decomposition of $T$ can also be orthogonal.
\end{remark}

This gives the second proof of \autoref{thm:normaldecom}:
\begin{proof}[Proof]
    WLOG $T$ normal and primary cyclic,
	then $p_T$ is primary, and $T$ normal $ \implies T$ semisimple,
	so $p_T$ has no multiple factors, thus $p_T$ prime, which proves the result.
\end{proof}

Next we present the third proof:
\begin{proposition}
	If $A, B\in \mathbb{R}^{n\times n}$ are unitarily similar,
	then they are orthogonally similar.
\end{proposition}

\begin{proposition}[QS decomposition]
	For any unitary matrix $U$, $U = QS$ where $Q$ real orthogonal, $S$ unitary
	and symmetrical. Moreover $\exists f\in \mathbb{C}[x]$ s.t. $S = f(U^tU)$.
\end{proposition}
