%! TeX root = ./main.tex
\begin{proof}[Proof]
    Let $\sigma(U^tU) = \{c_1, \dots, c_k\}$.
	We can take a polynomial $f\in \mathbb{C}[x]$ s.t. $f(c_i)^2 = c_i$.

	Since $U$ is unitary, $|c_i| = 1 \implies |f(c_i)| = 1$.

	Let $S = f(U^tU)$, we claim that $S$ unitary and $S^2 = U^tU$.

	Let $U^tU = P\diag(c_1, \dots, c_k)P^{-1}$, where $P$ is unitary,
	then $S = P\diag(f(c_1), \dots, f(c_k))P^{-1}$ is unitary,
	and clearly $S^2 = U^tU$.

	Let $Q = US^{-1}$, then $Q$ unitary. Since $S$ symmetrical,
	$S^{-1} = S^*\implies \overline{S^{-1}} = S^t = S$,
	\[
	\overline{Q}Q^{-1} = \overline{U} S S U^{-1} = \overline{U}U^tU U^{-1} = I_n.
	\]
	Hence $\overline{Q} = Q$, $Q$ is real orthogonal.
\end{proof}

\begin{proof}[Return to the original proposition]
    Let $A, B$ be real matrices unitarily similar,
	let $B = UAU^{-1}$, taking the conjuate we get
	\[
	UAU^{-1} = \overline{U}AU^t \implies U^tUA = AU^tU.
	\]
	Let $U = QS$, then $AS = SA$. We have
	\[
	B = UAU^{-1} = QSAS^{-1}Q^{-1} = QAQ^{-1}.
	\]
	Therefore $A, B$ are orthogonally similar.
\end{proof}

\begin{corollary}
    Let $A, B$ be normal matrices, TFAE:
	\begin{enumerate}[\indent(1)]
		\item $A, B$ are unitarily similar (or orthogonally similar);
		\item $A, B$ are similar;
		\item $f_A = f_B$.
	\end{enumerate}
\end{corollary}
\begin{proof}[Proof]
    We only need to prove $(3)\implies (1)$.

	When $F = \mathbb{C}$, $A, B$ are unitarily similar to
	diagonal matrices $D_1, D_2$.
	Since $f_A = f_B$, $D_1, D_2$ only differ by a permutation,
	hence unitarily similar.

	When $F = \mathbb{R}$, by the previous proposition and proof for $\mathbb{C}$,
	we get the result.
\end{proof}

The third proof of \autoref{thm:normaldecom} is to
factorize $f_T \in \mathbb{R}[x]$ and use the above corollary.

At last we prove another property of normal maps:
\begin{proposition}
	Let $A$ be a normal matrix, then $A^*$ is a complex polynomial of $A$.
\end{proposition}
\begin{proof}[Proof]
    Use the spectral decomposition.
\end{proof}

\section{Bilinear forms}
\label{sec:Bilinear forms}
In this section we study the bilinear forms on generic fields.
Let $M^2(V)$ denote all the bilinear forms on $V$.

For $f\in M^2(V)$,
Let $(f(\alpha_i, \alpha_j))_{ij}$ be the matrix of $f$ under basis $\{\alpha_i\}$.
(Note that this differs by a transpose with previous section)

Obviously $M^2(V) \to F^{n\times n}$ by $f\mapsto [f]_{\mathcal{B}}$ is
a linear isomorphism.

\begin{proposition}
	Let $ \mathcal{B}, \mathcal{B}'$ be two basis,
	$P$ is the transformation matrix between them, for all $f\in M^2(V)$ we
	have $[f]_{\mathcal{B}'} = P^t[f]_{\mathcal{B}}P$.
\end{proposition}
\begin{proof}[Proof]
    Trivial.
\end{proof}

If $A = P^tBP$ for some $P\in \GL(V)$, we say $A, B$ are \vocab{congruent}.

A bilinear form will induce two linear maps $V\to V^*$, namely $L_f, R_f$:
\[
L_f(\alpha)(\beta) = R_f(\beta)(\alpha) = f(\alpha, \beta).
\]
\begin{proposition}
	For any basis $ \mathcal{B}$,
	we have $\rank L_f = \rank R_f = \rank [f]_{\mathcal{B}}$.
	This number is called the rank of $f$, denoted by $\rank f$.
\end{proposition}

If $\rank f = n$, we say $f$ is non-degenrate, this is equivalent to
$L_f$ invertible or $R_f$ invertible.

\subsection{Some special bilinear forms}
\label{sub:Some special bilinear forms}
\begin{definition}
	For $f\in M^2(V)$,
	\begin{itemize}
		\item If $f(\alpha, \beta) = f(\beta, \alpha), \forall \alpha, \beta\in V$,
			then we say $f$ is \vocab{symmetrical}.
		\item If $f(\alpha, \beta) = -f(\beta,\alpha), \forall \alpha, \beta\in V$,
			we say $f$ is \vocab{anti-symmetrical}.
		\item If $f(\alpha, \alpha) = 0, \forall \alpha\in V$,
			we say $f$ is \vocab{alternating}.
	\end{itemize}
	We denote the above functions by $S^2(V), A^2(V), \Lambda^2(V)$.
\end{definition}
We can see that $\Lambda^2(V) \subset A^2(V)$,
and they are all subspaces of $M^2(V)$.

\begin{proposition}
	If $\Char F\ne 2$, then $A^2(V) = \Lambda^2(V)$,
	and $M^2(V) = A^2(V) \oplus S^2(V)$.
\end{proposition}
\begin{proof}[Proof]
    Already proved in last semester.
\end{proof}

\begin{proposition}
	Let $\mathcal{B}$ be any basis of $V$,
	\begin{itemize}
		\item $f$ symmetrical $\iff [f]_{\mathcal{B}}$ symmetrical;
		\item $f$ anti-symmetrical $\iff [f]_{\mathcal{B}}$ anti-symmetrical;
		\item $f$ alternating $\iff [f]_{\mathcal{B}}$ anti-symmetrical and
			the diagonal entries are all zero.
	\end{itemize}
\end{proposition}
