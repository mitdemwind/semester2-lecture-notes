%! TeX root = ./main.tex
For an invariant subspace $W \subset V$,
there may not exist a subspace $W'$ s.t.  $W\oplus W' = V$, so we can instead
study the quotient space.

Define $T_W = T|_W\in L(W)$, $T_{V/W}\in L(V/W)$:
$T_{V /W}(\alpha+W) = T(\alpha) + W$. It's clear that $T_{V /W}$ is well-defined.

However, this decomposition loses some imformation about $T$, i.e. they can't
determine  $T$ completely. For example when $T=\begin{pmatrix}
	A &B \\ 0 &C
\end{pmatrix}$, the matrix $B$ will not be carried to
$T_W$ and $T_{V /W}$ as their matrices are $A, C$ respectively.

Since $\det T = \det T_W \det T_{V /W}$, $f_T = f_{T_W}\cdot f_{T_{V /W}}$.
The minimal polynomials satisfy
\[\lcm(p_{T_W}, p_{T_{V/W}}) \mid p_T, \quad
p_T \mid p_{T_W} p_{T_{V /W}}.\]

This follows by the definition of $T_W, T_{V /W}$, readers can check it manually.
Hint: The image of $p_{T_{V /W}}(T)$ is in $W$.
So by \autoref{p_t decomposition}, $T$ is diagonalizable $\iff$ $T_W,T_{V /W}$
are both diagonalizable and their minimal polynomials are coprime.

\begin{definition}[Simultaneous diagonalization]
	Let $\mathcal{F} \subset L(V)$, if there exists $\mathcal{B}$ s.t.
	$\forall T\in \mathcal{F}$, $[T]_{\mathcal{B}}$ is diagonal matrix,
	then we say $\mathcal{F}$ can be simultaneously diagonalized.
\end{definition}

\begin{proposition}
	Let $\mathcal{F} \subset L(V)$, TFAE:
	\begin{itemize}
		\item $\mathcal{F}$ can be simultaneously diagonalized;
		\item Any element in $\mathcal{F}$ is diagonalizable, and any two elements
			commute with each other.
	\end{itemize}
\end{proposition}
\begin{proof}[Proof]
    It's obvious the first statement implies the second.

	On the other hand, we proceed by induction on the dimension of the space $V$.

	Assume  $\dim V = n\ge 2$, WLOG $T\in \mathcal{F}$ is not a scalar matrix.

	Let $\sigma(T)=\{c_1,\dots,c_r\}, V=\bigoplus_{i=1}^r V_{c_i}$, where $r\ge 2$,
	$V_{c_i}\ne V$.
	Since $T$ commutes with other elements in  $\mathcal{F}$,
	so $V_{c_i} = \ker(T - c_i\id_V)$ is invariant under all the maps in $\mathcal{F}$.

	Hence we can restrict $\mathcal{F}$ to $V_{c_i}$ and apply
	induction hypothesis, i.e. for any $U\in \mathcal{F}$, $U|_{V_{c_i}}$ can
	be simultaneously diagonalized.

	Therefore $\exists \mathcal{B}_i$ s.t. $[U|_{V_{c_i}}]_{\mathcal{B}_i}$ is diagonal
	$\implies$  $[U]_{\mathcal{B}}$ is diagonal, where $\mathcal{B}=\bigcup \mathcal{B}_i$.
\end{proof}

\begin{definition}[Triangulable matrix]
	Let $T\in L(V)$. If $[T]_{\mathcal{B}}$ is an upper
	triangular matrix for some basis $\mathcal{B}$,
	we say $T$ is \vocab{triangulable}.
\end{definition}

\begin{proposition}
	Let $\dim V=n$, for  $T\in L(V)$, TFAE:
	\begin{enumerate}[(1)]
		\item $T$ is triangulable;
		\item $f_T$(or $p_T$) can be decomposed to product of polynomials of degree 1.
		\item There exists a sequence of $T$-invariant subspaces
			$\{0\}=W_0 \subset W_1 \subset\dots \subset W_n = V$.

			This kind of sequence is called a \vocab{flag}.
			(Flag itself does not require $T$-invariant)
	\end{enumerate}
\end{proposition}
\begin{remark}
	In particular, when the base field is \textit{algebraically closed}, the above
	statements always holds.
\end{remark}
\begin{proof}[Proof]
	It's obvious that $(1) \implies (2)$.

	For $(2)\implies (3)$:
	We proceed by induction, for $W_1$ just take the space spanned by
	one of the eigenvectors of $T$.

	Assume that we have constructed $W_j$ for $0\le j\le i$.
	Instead of finding an invariant subspace of dimension $i+1$,
	we'll find an invariant subspace of dimension 1 in  $V / W_i$.


	Let $Q$ denote the quotient map  $V\to V / W_i$.
	Consider the map $T_{V / W_i}: \alpha+W_i \mapsto T(\alpha)+W_i$.

	We have
	\[
		T_{V / W_i}\circ Q = Q\circ T.
	\]
	\begin{center}
	    \begin{tikzcd}
			V\rar["T"]\dar["Q"] &V\dar["Q"]\\
			V/W_i\rar["T_{V/W_i}"] &V/W_i
		\end{tikzcd}
	\end{center}

	Since $p_{T_{V / W_i}}\mid p_T \implies p_{T_{V /W_i}}$ is product of
	polynomials of degree 1, $T_{V / W_i}$ must have an eigenvector.
	Let $L$ denote the subspace spanned by this vector,
	and $W_{i+1}=Q^{-1}(L)$.

	Clearly $\dim W_{i+1} = 1 + \dim W_i = i+1$.
	It suffices to check that  $W_{i+1}$ is $T$-invariant:
	\[
	T(W_{i+1}) = T(Q^{-1}(L)) = Q^{-1}(T_{V / W_i}(L)) \subset Q^{-1}(L) = W_{i+1}.
	\]

	Now for the last part $(3)\implies (1)$:

	Let $\mathcal{B} = \{\alpha_1,\dots,\alpha_n\}$,
	such that $\Span\{\alpha_1,\dots,\alpha_i\} = W_i$.
	The matrix of $T$ under  $\mathcal{B}$ is clearly an upper triangular matrix.
\end{proof}
