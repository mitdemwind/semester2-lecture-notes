%! TeX root = ./main.tex
\begin{theorem}[Jordan decomposition]
    Let $T\in L(V)$, $n = \dim V$, and $F$ is perfect.
	There exists unique $S,N\in L(V)$ s.t. $T = S+N$, where $S$ semisimple and
	$N$ nilpotent, and $SN = NS$.

	Moreover there exists $f,g\in F[x]$ s.t. $S = f(T), N = g(T)$.
\end{theorem}

To prove this generalized version, we need the following observation:
\begin{proposition}
	Let $F$ be a perfect field, $A\in F^{n\times n}$ is semisimple
	iff $A$ is diagonalizable in $\overline{F}^{n\times n}$.
\end{proposition}
\begin{proof}[Proof]
	$A$ semisimple $\iff$ $p_A$ has no multiple factors in $F[x]$

	$\iff$ $p_A$ has no multiple roots in $\overline{F}[x]$

	$\iff$ $p_A$ is the product of different monic polynomials of degree 1

	$\iff$ $A$ is diagonalizable in $\overline{F}^{n\times n}$.
\end{proof}

\begin{proposition}
	Let $F$ be a perfect field, $a\in \overline{F}$.
	Then $a\notin F\iff$ exists an automorphism $\sigma$ s.t. $\sigma\big|_F = \id_F$,
	i.e. $\sigma\in \mathrm{Gal}(\overline{F}/F)$ but $\sigma(a)\ne a$.
\end{proposition}
\begin{remark}
    This proof is beyond the scope of this class, but the idea is similar
	to the conjugate operation on $ \mathbb{C}/\mathbb{R}$.
\end{remark}

Now we prove the Jordan decomposition:
\begin{proof}[Proof]
    Let $A = S + N$ is the Jordan decomposition on $\overline{F}^{n\times n}$.
	Then by applying $\sigma$ on this equation,
	\[
	A = \sigma(S) + \sigma(N)
	\]
	holds for all $\sigma\in \mathrm{Gal}(\overline{F}/F)$.
	Since $\sigma(S)$ is also diagonalizable, $\sigma(N)$ is nilpotent,
	as $\sigma$ is an automorphism.
	So by the uniqueness of Jordan decomposition, $\sigma(S)=S, \sigma(N)=N$.

	This implies $S, N \in F^{n \times n}$.
\end{proof}

\subsection{Bonus section}
\label{sub:Bonus section}
Starting from Galois groups mentioned above,
let
\[
\Aut(E / F) := \left\{\sigma \in \Aut(E)\mid \sigma|_F = \id_F\right\}
\]
be the automorphism group of field extension $E / F$.
\begin{example}
    Let $F = \mathbb{Q}$, $E = \mathbb{Q}(\sqrt{2})$, then
	$\sigma: a+b\sqrt{2} \mapsto a-b\sqrt{2}$ is in $\Aut(E / F)$.

	If $E = \mathbb{Q}(\sqrt[3]{2})$, if $\sigma\in \Aut(E / F)$, then
	$\sigma(\sqrt[3]{2})$ is a root of $x^3 - 2$ $ \implies \sigma = \id$.
	Thus $E / F$ is not a \textit{Galois extension}.
\end{example}

When $E / F$ is a Galois extension, we write $\Gal(E / F) = \Aut(E / F)$.

In the history, this concept is used to solve polynomial equations.

Let $f\in \mathbb{Q}[x]$, let $x_1,\dots,x_n$ be all roots of $f$.
Consider $E = \mathbb{Q}(x_1,\dots,x_n)$, and define $\Gal(f) = \Gal(E / \mathbb{Q})$.
Back in the times of Galois, the concept of field haven't been developed yet,
so what he did is to consider the bijections between the roots of $f$.

Galois discovered that $f$ has radical solutions if and only if
the group $\Gal(f)$ has a property, and he named it ``solvable''.
Since all the subgroups of $S_4$ are solvable, thus if $\deg f\le 4$,
$f$ always has radical solutions, but $A_5<S_5$ is not solvable,
so polynomials of degree greater than 4 may not have radical solutions.

One of the ultimate goal of modern algebra is to comprehend the
group $\Gal(\overline{\mathbb{Q}}/\mathbb{Q})$.

A tool developed for this goal is \textit{group representation}.
A representation of a group $G$ is a homomorphism $\varphi: G\to \GL(V)$.
Since $\GL(V)$ is something people knows very well, so when
the elements of an abstract group $G$ is viewed as linear maps,
it's easier to discover more properties of $G$.

When $G = \Gal(\overline{\mathbb{Q}}/\mathbb{Q})$, the representation is
called a \textit{Galois representation}. Even one dimensional
Galois representations are very nontrivial.
