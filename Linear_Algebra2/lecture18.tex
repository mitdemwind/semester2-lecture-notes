%! TeX root = ./main.tex
\begin{proposition}
	$\ker(T^*) = \Img(T)^\perp$, $\Img(T^*) = \ker(T)^\perp$.
	$(cT+U)^* = \overline{c}T^* + U^*$, $(TU)^* = U^*T^*$, $T^{**} = T$.

	This means the map $T\mapsto T^*$ is a conjugate anti-automorphism of $L(V)$,
	and it's an involution.
\end{proposition}

If $T^* = T$, then we say $T$ is \vocab{self-adjoint},
and if $T^* = -T$, we say $T$ is \vocab{anti self-adjoint}.

Let $F = \mathbb{C}$, $T$ is self-adjoint iff $iT$ is anti self-adjoint.
Like a function can be written as a sum of odd and even functions,
$\forall T\in L(V)$, there exists unique self-adjoint $T_1, T_2$
s.t. $T = T_1 + iT_2$. In fact, $T_1 = \frac{T+T^*}{2}, T_2 = \frac{T-T^*}{2i}$.

Let $\mathcal{B}$ be an orthonormal basis,
obviously $T$ self-adjoint $\iff [T]_{\mathcal{B}}$ Hermite.

\begin{example}
    Let $W \subset V$, $p_W$ be the orthogonal projection.
	then $p_W$ is self-adjoint as we can choose an orthonormal basis $\mathcal{B}$,
	such that $[p_W]_{\mathcal{B}} = \diag\{I_k, 0\}$, where $k=\dim W$.
\end{example}

Let $V, W$ be inner product spaces, we'll study the linear maps $T: V\to W$
which preserves the inner products, i.e.
\[
	\left<\alpha,\beta \right>_V = \left<T\alpha, T\beta \right>_W.
\]
If $T$ is an isomorphism, then we say $T$ is the isomorphism between
inner product spaces.
\begin{proposition}
	$T$ preserves inner product $\iff T$ is an isometry, i.e. preserves length.

	In particular, isometry is always injective implies that
	inner product presering maps are always injective.
\end{proposition}
\begin{proof}[Proof]
    Trivial by polarization identity.
\end{proof}
\begin{proposition}
	Let $V,W$ be finite dimensional inner product spaces, $\dim V = \dim W$,
	$T \in \Hom(V, W)$, the followings are equivalent:
	\begin{enumerate}[\indent(1)]
		\item $T$ preserves inner product;
		\item  $T$ is an isomorphism between inner product spaces;
		\item $T$ maps all the orthonormal bases in $V$ to
			orthonormal bases in $W$ ;
		\item $T$ maps \textit{one} orthonormal basis in $V$ to a
			orthonormal basis in $W$.
	\end{enumerate}
\end{proposition}
\begin{proof}[Proof]
    It's clear that $(1)\implies (2)\implies (3)\implies (4)$,
	since $T$ injective $ \implies T$ is an isomorphism of vector space.

	As for $(4)\implies (1)$, just expand everything using this orthonormal basis.
\end{proof}
\begin{corollary}
    Inner product spaces with same dimensions are always isomorphic
	as inner product spaces.
\end{corollary}

Recall the positive definite matrices we defined earlier,
we can also define \textit{positive definite maps}:
Let $T$ be a \textit{self-adjoint map}, if
\[
	\forall \alpha\in V \backslash \{0\},\quad \left<T\alpha, \alpha \right> > 0,
\]
then we say $T$ is positive definite.

The reason why we require $T$ self-adjoint is that,
\[
\left<T\alpha, \alpha \right> = \left<\alpha, T\alpha \right>
= \overline{\left<T\alpha, \alpha \right>}\implies \left<T\alpha, \alpha \right>\in
\mathbb{R}.
\]
so we can talk about ``positive'' safely.

\subsection{Orthogonal maps and Unitary maps}
\label{sub:Orthogonal maps and Unitary maps}

\begin{definition}[Orthogonal maps]
	Let $V$ be a real inner product space, the automorphisms of $V$
	(as inner product space) are
	called \vocab{orthogonal maps}, denoted the set by $\mathrm{O}(V)$.

	When $V$ is a complex inner product space,
	we use \vocab {unitary maps} and $\mathrm{U}(V)$ instead.
\end{definition}

\begin{proposition}
	Let $V$ be an inner product space,
	\[
		T\in \mathrm{O}(V)\iff T^* = T^{-1}.
	\]
\end{proposition}
\begin{proof}[Proof]
    \[
    T\in \mathrm{O}(V) \iff \left<\alpha, \beta \right> =
	\left<T\alpha, T\beta \right> = \left<\alpha, T^*T\beta \right>,
	\quad \forall \alpha, \beta \in V.
    \]
	This also holds for $\mathrm{U}(V)$.
\end{proof}
\begin{proposition}
	Let $A\in \mathbb{R}^{n\times n}$, TFAE:
	\begin{itemize}
		\item $A^tA = I_n$ ;
		\item The column (row) vectors of $A$ form
			an orthonormal basis of $\mathbb{R}^n$.
	\end{itemize}
\end{proposition}
\begin{proof}[Proof]
    Since $A$ maps the standard basis to the column vectors of $A$,
	so the conclusion follows immediately (use $A^t$ to get the row vectors).
\end{proof}

Let $\mathrm{O}(n) = \{A\in \mathbb{R}^{n\times n}\mid A^tA = I_n\}$,
and $\mathrm{U}(n) = \{A\in \mathbb{C}^{n\times n} \mid A^*A = I_n\}$.
We can see that $A^tA = I_n \implies \det(A) = \pm 1$,
and $A^*A = I_n \implies |\det(A)| = 1$.
