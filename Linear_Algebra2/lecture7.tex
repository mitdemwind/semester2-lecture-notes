%! TeX root = ./main.tex
\begin{proposition}
	Let $F$ be an algebraically closed field.
	Suppose the elements of $\mathcal{F}\subset L(V)$ are pairwise commutative,
	then $\mathcal{F}$ is simultaneously triangulable.
\end{proposition}
\begin{remark}
    The inverse of this proposition is not true:
	Just let $\mathcal{F}$ be the set consisting of all the
	upper triangular matrices.
\end{remark}
\begin{lemma}
	There's a common eigenvector of $\mathcal{F}$.
\end{lemma}
\begin{proof}[Proof of lemma]
    WLOG $\mathcal{F}$ is finite.
	(In fact, $\Span \mathcal{F} \subset L(V)$ is a finite dimensional vector space,
	so we can take a basis $\mathcal{F}_0$.)

	Now by induction, if $T_1,\dots, T_{k-1}$ have common eigenvector $\alpha$,
	let $T_i\alpha = c_i \alpha$.
	Then
	\[
	W = \bigcap_{i=1}^{k-1} \ker(T_i - c_i \id_V) \ne \{0\}
	\]
	is a $T_k$-invariant space.

	So any eigenvector $\alpha'$ of $T_k\big|_W$ is the common eigenvector.
\end{proof}
\begin{proof}[Proof of the proposition]
    It suffices to prove that there exists an $\mathcal{F}$-invariant flag.
	By the lemma, the proof is nearly identical as the proof
	of previous proposition.
\end{proof}

\subsection{Decomposition of linear maps}
\label{sub:Decomposition of linear maps}

In this section we mainly study how a linear map is decomposed into
irreducible maps and the structure of irreducible maps.

Recall that every vector space $V$ is an $F[x]$-module given a linear operator $T$.
If a subspace $W \subset V$ is a $T$-invariant space,
then $W$ is a submodule of $V$.

Hence it leads to decompose $V$ into direct sums of submodules.

\begin{definition}
	Let $V, W$ be isomorphic vector spaces.
	$T\in L(V)$, $T'\in L(W)$.
	If there exists an isomorphism $\Phi: V\to W$ s.t.
	$\Phi \circ T = T'\circ \Phi$,
	we say $T$ and $T'$ are \vocab{equivalent}.
\end{definition}

\begin{definition}[Primary maps]
	Let $T\in L(V)$ be a linear map. We say $T$ is \vocab{primary}
	if $p_T$ is a power of prime polynomials.
\end{definition}

\begin{theorem}[Primary decomposition]
    Let $T\in L(V)$,  $p_T = \prod_{i=1}^k p_i^{r_i}$,
	where $p_i$ are different monic prime polynomials of degree 1.

	We have
	\[
	V = \bigoplus_{i=1}^k W_i, \quad W_i = \ker\left(p_i^{r_i}(T)\right),
	\]
	with $W_i \ne \{0\}$ and $T\big|_{W_i}$ primary.
\end{theorem}
\begin{proof}[Proof]
    Let $f_i = \prod_{j\ne i}p_j^{r_j}$, $f_i$ and $p_i$ are coprime.

	Note that $f_i(T)\ne 0$ and $f_i(T)p_i^{r_i}(T) = p_T(T) = 0$,
	thus $p_i^{r_i}(T)$ is not inversible, which implies $W_i \ne \{0\}$.

	$W_i$ independent :
	If there exists $\alpha_j\in W_j$ s.t. $\sum_{j=1}^k \alpha_j = 0$,
	applying $f_i$ we get $f_i(\alpha_i) = 0$.
	But $p_i^{r_i}(\alpha_i) = 0 \implies \alpha_i = 0, \forall i$.

	To prove $V=\sum_{i=1}^k W_i$, observe that
	\[
	\gcd(f_1,\dots,f_k) = 1\implies \exists g_1,\dots,g_k
	\quad s.t.\quad 1 = \sum_{i=1}^k g_if_i
	\implies \alpha = \sum_{i=1}^k g_i(f_i\alpha),\quad \forall \alpha\in V.
	\]
	Since $f_i\alpha \in W_i$, $W_i$ is $T$-invariant $\implies g_if_i\alpha\in W_i$.

	Lastly, we'll prove that the minimal polynomial $q_i$ of $T\big|_{W_i}$ is $p_i^{r_i}$.

	Clearly $p_i^{r_i}(T\big|_{W_i}) = 0$, so $q_i\mid p_i^{r_i}$.

	On the other hand, $q_1q_2\dots q_k$ is an annihilating polynomial of $T$,
	hence
	\[
		\prod_{i=1}^k p_i^{r_i}\mid \prod_{i=1}^k q_i\implies q_i = p_i^{r_i}, \quad \forall i.
	\]
\end{proof}
