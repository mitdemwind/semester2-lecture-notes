%! TeX root = ./main.tex
First we prove a lemma:
\begin{lemma}
	Let $\alpha\in V$ with $p_\alpha = p_T$, $\forall L \in V / R\alpha$,
	exists $\beta\in L$ s.t. $p_\beta = p_L$.

	Here $f\cdot L := f(T_{V / R\alpha}) L$, so $fL=0\iff f(T)\beta\in R\alpha$,
	$\forall \beta\in L$.
\end{lemma}
\begin{proof}[Proof]
    For all $\beta\in L$, we must have $p_\beta L=0$,
	since $L = \beta + R\alpha, T(R\alpha) = R\alpha$.

	If $p_L\beta \ne 0$, since $p_L \beta\in R\alpha$,
	thus $p_L \beta = f\alpha$ for some $f\in R$.

	Because $p_L \mid p_\beta \mid p_\alpha = p_T$,
	\[
	\left( \frac{p_\alpha}{p_L} \right)f\alpha = p_\alpha \beta = 0.
	\]
	We have $\frac{p_\alpha}{p_L}f$ is an annihilator of $\alpha$,
	hence it's a multiple of $p_\alpha$, i.e.  $p_L \mid f$.

	Let $f = p_L h$, $\beta_0 = \beta - h\alpha$,
	we have $p_L\beta_0 = f\alpha - p_L h \alpha = 0$
	$\implies p_{\beta_0} = p_L$.
\end{proof}

Returning to our original theorem,
we'll prove by induction on $n$.

Take $\alpha_1\in V$ s.t. $p_{\alpha_1} = p_T$.
Consider $V / R\alpha_1$, its dimension is strictly lesser than $n$.
By induction hypo, $\exists L_2, L_3, \dots, L_r\in V / R\alpha_1$,
such that
\[
V / R\alpha_1 = \bigoplus _{i=1}^r RL_i, \quad p_{L_r}\mid \dots \mid p_{L_2}.
\]
Take $\alpha_i \in L_i$ s.t. $p_{\alpha_i} = p_{L_i}$,
we must have $p_{\alpha_r} \mid \dots \mid p_{\alpha_1} = p_T$.

If there exists $g_i\alpha_i \in R\alpha_i$ s.t. $\sum_{i=1}^r g_i\alpha_i = 0$,
then
\[
	\sum_{i=2}^r g_iL_i = 0 \implies g_i L_i = 0 \implies g_i\alpha_i = 0.
\]

For any $\gamma \in V$, since $\gamma \in \gamma+R\alpha_1$,
by induction hypo, $\gamma + R\alpha_1 = \sum_{i=2}^{r} h_iL_i$.

This means $\gamma - \sum_{i=2}^{r} h_i \alpha_i \in R\alpha_1$,
this completes the existence part of the theorem.
\vspace{1ex}

As for the uniqueness part,
note that $p_T = \lcm(p_1,\dots,p_r) = p_1$ and $f_T = p_1\cdots p_r$,
suppose $q_1,\dots,q_s$ are also invariant factors of $T$,
we must have $p_1=q_1=p_T$ and $\prod p_i = \prod q_i$.

Assume for contradiction that $\exists 2\le t\le \min\{r,s\}$ s.t.
$p_t\ne q_t$, but $p_i=q_i$ for all $i<t$.

Multiplying $p_t$ on both sides of $\bigoplus_{i=1}^r R\alpha_i
= \bigoplus_{i=1}^s R\beta_i$ we get:
\[
\bigoplus _{i=1}^{t-1} Rp_t\alpha_i = p_tV
= \bigoplus _{i=1}^{t-1}Rp_t\beta_i \oplus \bigoplus_{i=t}^s Rp_t\beta_i.
\]
Now observe that
\begin{itemize}
	\item For monic polynomial $f,g$, if $p_\alpha = fg$,
		then  $p_{f\alpha} = g$ as $h(f\alpha) = 0\iff (fh)\alpha = 0$.
\end{itemize}

Hence
\[
	\dim Rp_t\alpha_i = \deg p_{p_t\alpha_i} = \deg \frac{p_i}{p_t}
	= \deg \frac{q_i}{p_t} = \deg Rp_t\beta_i.
\]
This implies $\bigoplus_{i=t}^s Rp_t\beta_i = \{0\}$,
in particular $p_t\beta_t = 0\implies p_t\mid q_t$.
Similarly $q_t\mid p_t \implies p_t = q_t$, contradiction!

\begin{theorem}
    Let $G$ be a finite abelian group, then $\exists g_1,\dots,g_r\in G\backslash\{0\}$,
	such that $G = \bigoplus_{i=1}^r \mathbb{Z}g_i$ and
	$|\mathbb{Z}g_r| \mid \dots \mid |\mathbb{Z}g_1|$.
\end{theorem}
\begin{remark}
    The proof is identical to the proof above.
\end{remark}

\subsection{Rational canonical forms}
\label{sub:Rational canonical forms}

Let $d_i = \deg p_i = \dim R\alpha_i$,
$\mathcal{B}_i = \{\alpha_i,\dots,T^{d_i-1}\alpha_i\}$ is a basis of $R\alpha_i$.
Then $[T_{R\alpha_i}]_{\mathcal{B}_i}$ is the companian matrix $C_{p_i}$,
hence $T$ can be represented as a blocked diagonal matrix with
each block is $C_{p_i}$ for invariant factors $p_i$.
This is called the \vocab{rational canonical form} of $T$.

\begin{definition}
	We say $A\in F^{n\times n}$ is \vocab{rational} if exists monic $p_1,\dots,p_r\in F[x]$,
	such that $p_r\mid \dots \mid p_1$ and $A = \diag(C_{p_1},\dots,C_{p_r})$.
\end{definition}

\begin{theorem}
    Let $T\in L(V)$, then $T$ has a unique rational canonical form.
\end{theorem}
\begin{proof}[Proof]
	If $[T]_{\mathcal{B}'}=\diag(C_{q_1},\dots,C_{q_r})$ is another
	rational canonical form, let $\mathcal{B}' = (\mathcal{B}_1',\dots,\mathcal{B}_r')$.

	It's easy to observe that $\Span \mathcal{B}_i' = R\beta_i$, where
	$\beta_i$ is the first element in $\mathcal{B}_i$,
	so $V = \bigoplus_{i=1}^r R\beta_i$ is a cyclic decomposition of $V$,
	by the previous theorem we deduce the canonical form is unique.
\end{proof}
