%! TeX root = ./main.tex
\begin{definition}[Quadratic forms]
	Let $q: V\to F$ be a function, we say $q$ is a \vocab{quadratic form}
	if there exists $f\in M^2(V)$ s.t.
	\[
	q(\alpha) = f(\alpha, \alpha), \quad \forall \alpha\in V.
	\]
\end{definition}

When $V = F^n$, quadratic forms are just a homogenous quadratic polynomial
with $n$ variables, i.e.
\[
q(X) = X^t A X, \quad A\in F^{n\times n}, X \in F^{n}.
\]
Let $Q(V)$ denote all the quadratic forms on $V$, it's an $F$-vector space.
By definition there's a surjective linear map $M^2(V)\to Q(V)$ by
$\Phi(f)(\alpha) = f(\alpha, \alpha)$.

\begin{proposition}
	Let $\Char F \ne 2$,
	\begin{itemize}
		\item The map $\Phi: S^2(V) \to Q(V)$ is an isomorphism.
		\item Let $q\in Q(V)$, if $f\in S^2(V)$ and $\Phi(f) = q$, then
			\[
			f(\alpha, \beta) = \frac{1}{4}(q(\alpha+\beta) - q(\alpha-\beta)).
			\]
	\end{itemize}
\end{proposition}
\begin{proof}[Proof]
    The first one can be proved by $\ker(\Phi) = \Lambda^2(V)$ and
	$M^2(V) = S^2(V) \oplus \Lambda^2(V)$.

	The second one is trivial by direct computation.
\end{proof}

From this we can define the matrix of a quadratic form $q$ to be
the matrix of the symmetrical bilinear form $\Phi^{-1}(q)$,
thus $[q]_{\mathcal{B}}$ is always symmetrical.

\begin{theorem}
    Let $f\in M^2(V)$,
	\begin{itemize}
		\item  If $\Char F\ne 2$, then $f\in S^2(V) \iff \exists \mathcal{B}$,
			s.t. $[f]_ {\mathcal{B}}$ diagonal;
		\item $f\in\Lambda^2(V) \iff \exists \mathcal{B}$ s.t. $[f]_{\mathcal{B}}$
			is block diagonal with each block being $0$ or $\begin{pmatrix}
				0&1\\-1&0
			\end{pmatrix}$.
	\end{itemize}
\end{theorem}

To prove this theorem, it's sufficient to prove:
\begin{lemma}
	Let $f\in S^2(V)\cup A^2(V)$, $W \subset V$ is a subspace, let
	\[
	W^\perp = \{\beta\in V \mid f(\alpha, \beta) = 0, \forall \alpha\in W\}.
	\]
	If $f\big|_{W}$ is non-degenerate, then $V = W\oplus W^\perp$.
	In this case, let $\mathcal{B}_1, \mathcal{B}_2$ be basis of $W$, $W^\perp$,
	and  $\mathcal{B} = (\mathcal{B}_1, \mathcal{B}_2)$, we have
	\[
		[f]_{\mathcal{B}} = \diag([f|_W]_{\mathcal{B}_1},
		[f|_{W^\perp}]_{\mathcal{B}_2}).
	\]
\end{lemma}
\begin{proof}[Proof]
    Since $f\big|_W$ non-degenerate, $W\cap W^\perp = 0$.
	Note that
	\[
	W^\perp = \bigcap_{\alpha\in W}\ker(L_f(\alpha)) = L_f(W)^0.
	\]
	Thus $\dim W^\perp = n - \dim L_f(W) \ge n - \dim W$.
	This implies that  $V = W\oplus W^\perp$.

	For the second part, since  $f(\alpha, \beta) = 0\implies f(\beta, \alpha) = 0$,
	thus the matrix must obey the conclusion.
\end{proof}

Now by induction it's trivial when $n = 1$,
\begin{itemize}
	\item When $f\in S^2(V)$,
		WLOG $f\ne 0$, $\exists \alpha$ s.t. $f(\alpha, \alpha) \ne 0$.
		Let $W = \Span\{\alpha\}$, by lemma and induction hypo we're done.
	\item When $f\in A^2(V)$, there exists $\alpha, \beta$
		s.t. $f(\alpha, \beta) =1$. Let $W = \Span\{\alpha, \beta\}$,
		similarly by lemma and induction hypo, we're done.
\end{itemize}

\begin{corollary}
	For any $q\in Q(V)$,
	there exists a basis of $V$ s.t. $[q]_{\mathcal{B}}$ diagonal.
\end{corollary}

The non-degenerate alternating bilinear forms are called \vocab{symplectic forms}.
\begin{corollary}
    If there exists symplectic form $f$ on $V$, then $\dim V = 2m$ and
	\[
		[f]_{\mathcal{B}} = \begin{pmatrix}
			0&I_m \\ -I_m&0
		\end{pmatrix}
	\]
	for some basis $\mathcal{B}$.
\end{corollary}

\begin{theorem}
    Let $F$ be an algebraically closed field, and $\Char F \ne 2$.
	Let $f\in S^2(V)$, there exists a basis $\mathcal{B}$, s.t. $[f]_{\mathcal{B}}$ 
	diagonal and the diagonal entries can only be $0$ or $1$.
\end{theorem}
\begin{proof}[Proof]
    Use the previous result and multiply some scalars (the root of $x^2 = c$).
\end{proof}

When $F = \mathbb{R}$, using similar technique we can prove the
diagonal entries can only be $0, 1$ or $-1$.
