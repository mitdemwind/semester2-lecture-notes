%! TeX root = ./main.tex
\subsection{Lie algebras}
\label{sub:Lie algebras}

There's a class I missed, so the notes may not be complete.

\begin{definition}[Lie algebra]
	Let $L$ be a vector space over a field $F$. Suppose an operation
	(called \vocab{Lie bracket})
	\[
		L\times L \to L, \quad (x, y) \mapsto [x, y]
	\]
	is given and satisfies:
	\begin{itemize}
		\item (Bilinearity)
			\[
			\left\{\begin{aligned}
					&[ax+by,z] = a[x,z]+b[y,z],\\
					&[x,ay+bz] = a[x,y]+b[x,z],
			\end{aligned}
			\right. \quad \forall x,y,z\in L, a,b\in F;
			\]
		\item (Alternativity)
			\[
				[x, x] = 0, \quad \forall x\in L;
			\]
		\item (Jacobi identity)
			\[
				[[x, y], z] + [[y, z], x] + [[z, x], y] = 0,\quad \forall x,y,z\in L.
			\]
	\end{itemize}
	Then $L$ is called a \vocab{Lie algebra} over $F$.
\end{definition}
The Lie algebra can be viewed as a vectorization of Lie groups,
where Lie bracket is the commutator in Lie groups.

\begin{example}
    On any $F$-vector space $L$, one can define a trivial Lie bracket by
	\[
		[x, y] = 0,\quad \forall x,y\in L
	\]
	Then $L$ becomes a Lie algebra, called an \vocab{abelian Lie algebra}.
\end{example}

We can also define homomorphisms by $\phi([x, y]) = [\phi(x), \phi(y)]$.

\begin{definition}[Representation]
	Let $L$ be a Lie algebra over $F$. A \vocab{representation} of $L$ is
	a homomorphism $\phi: L\to \mathfrak{gl}(V)$, where $V$ is some
	finite-dimensional $F$-vector space.
\end{definition}

\begin{example}[Adjoint representation]
    Let $L$ be a Lie algebra over $F$. Define a linear map
	$\mathrm{ad}: L\to \mathfrak{gl}(L)$ by
	\[
		\mathrm{ad}(x)(y) = [x, y],\quad \forall x, y\in L.
	\]
	We claim that it is a representation, called the \vocab{adjoint representation}
	of $L$. In fact, it follows from the Jacobi identity that for any $x,y,z\in L$,
	\begin{align*}
		\mathrm{ad}([x, y])(z) &= [[x,y],z]\\
		& = [x, [y, z]] - [y, [x, z]]\\
		& = \mathrm{ad}(x)([y, z]) - \mathrm{ad}(y)([x, z])\\
		& = [\mathrm{ad}(x), \mathrm{ad}(y)](z).
	\end{align*}
\end{example}

\begin{definition}[Subalgebra, ideal, quotient algebra]
	Let $L$ be a Lie algebra over $F$.
	\begin{itemize}
		\item If $S, T \subset L$ are subspaces, write
			\[
				[S, T] := \Span\{[x, y]: x\in S, y\in T\}.
			\]
		\item A subspace $K \subset L$ is called a \vocab{subalgebra} if
			$[K, K] \subset K$, denoted $K < L$.
		\item A subspace $I \subset L$ is an \vocab{ideal} if $[I, L] \subset I$,
			denoted $I \lhd L$.
		\item Let $I\lhd L$. On the quotient space $L / I$,
			we introduce the Lie bracket
			\[
				[x + I, y + I] := [x, y] + I, \quad \forall x, y\in L.
			\]
			Then $L/ I$ becomes a Lie algebra, called the \vocab{quotient algebra}
			of $L$ by $I$.
	\end{itemize}
\end{definition}

\begin{example}
    Let $\phi: L\to L'$ be a homomorphism. Then
	\[
	\ker \phi \lhd L, \quad \img(\phi) \lhd L', \quad \img(\phi) \cong L / \ker \phi.
	\]
	The \vocab{center} of $L$ is defined as
	\[
		Z(L) := \{x\in L: [x,y] = 0, \forall y\in L\}.
	\]
	We have $Z(L) \lhd L$ and $Z(L) = \ker \mathrm{ad}$.
\end{example}

\begin{definition}[Direct sum]
	Let $L_1, \dots L_r$ be Lie algebras over $F$. On the (external) vector space
	Direct sum $L_1\oplus\dots\oplus L_r$ we introduce the Lie bracket
	\[
		[(x_1,\dots,x_r), (y_1,\dots, y_r)] = ([x_1,y_1], \dots, [x_r, y_r])
	\]
	This makes $L_1\oplus\dots\oplus L_r$ a Lie algebra, called the
	\vocab{(external) Lie algebra direct sum} of $L_1, \dots, L_r$.
\end{definition}

\begin{definition}[Linear Lie algebra]
	Subalgebras of $\mathfrak{gl}_n(F)$ and $\mathfrak{gl}(V)$ are called
	\vocab{linear Lie algebras}.
\end{definition}

We have the following deep result:
\begin{theorem}[Ado-Iwasawa]
    All finite-dimensional Lie algebras over $F$ are isomorphic to linear Lie
	algebras.
\end{theorem}

Let us introduce some important linear Lie algebras.
\begin{example}[Special linear Lie algebra]
    Let
	\[
	\mathfrak{sl}_n(F) = \{x\in \mathfrak{gl}_n(F): \tr(x) = 0\},
	\mathfrak{sl}(V) = \{x\in \mathfrak{sl}(V): \tr(V) = 0\},
	\]
	where $V$ is a vector space over $F$.
	We have $\mathfrak{sl}(V) \lhd \mathfrak{gl}(V)$.
\end{example}

\begin{example}[The Lie algebra $L(V, f)$]
    Let $V$ be a finite-dimensional $F$-vector space, and $f: V\times V\to F$ be
	a bilinear form. For $x\in \mathfrak{gl}(V)$, we say that $f$ is
	\vocab{invariant under $x$ (in the infinitesimal snese)} if
	\[
	f(xv, w) + f(v, xw) = 0, \quad \forall v, w\in V.
	\]
	This comes from the derivative of Lie groups:
	Let $L\in \GL(V)$, $g(0) = \id_V$. By taking derivatives at $t=0$ on
	\[
	f(g(t)v, g(t)w) = f(v, w),
	\]
	we get $f(g'(0)v, w) + f(v, g'(0)w) = 0$.

	Let $L(V, f) \subset \mathfrak{gl}(V)$ be the subspace of
	all $x\in \mathfrak{gl}(V)$ that leave $f$ invariant,
	we claim that $L(V, f) < \mathfrak{gl}(V)$.
\end{example}

\begin{example}
    Let's consider 2 special cases of $L(V, f)$ :
	\begin{itemize}
		\item Let $V = F^n$, and $f$ be the symmetrical form given by
			\[
			f(v, w) = v^tw, \quad \forall v, w\in F^n.
			\]
			Then $\mathfrak{o}_n(F):= L(F^n, f)$ is called the
			\vocab{orthogonal Lie algebra}.
			Under the identification $\mathfrak{gl}(F^n) \cong \mathfrak{gl}_n(F)$,
			we have $\mathfrak{o}_n(F) = \{x\in \mathfrak{gl}_n(F): x^t+x=0\}$.
		\item Let $V = F^{2n}$, and $f$ be the symplectic form given by
			\[
			f(v, w) = v^t \begin{pmatrix}
				0 & I_n\\ -I_n & 0
			\end{pmatrix}
			w, \quad \forall v, w\in V.
			\]
			Then $\mathfrak{sp}_{2n}(F) := L(F^{2n}, f)$ is called the
			\vocab{symplectic Lie algebra}.
	\end{itemize}
\end{example}

Suppose $I\lhd L$, and we understand $I$ and $L / I$, then we understand $L$ (in
the rough sense). This motivates the following:
\begin{definition}[Simple Lie algebra, semisimple Lie algebra]
	Let $L$ be a finite-dimensional Lie algebra over $F$.
	\begin{itemize}
		\item $L$ is \vocab{simple} if it's nonabelian and has no nontrivial ideals.
		\item  $L$ is \vocab{semisimple} if it's nonzero and has no nonzero abelian
			ideal.
	\end{itemize}
\end{definition}
Clearly, a simple Lie algebra is semisimple.
One of our main purposes is to explain the proof of the following classification
theorem:
\begin{theorem}
    Let $L$ be a finite-dimensional Lie algebra over $\mathbb{C}$.
	\begin{enumerate}[(1)]
		\item $L$ is semisimple iff it's isomorphic to the direct sum of
			finitely many simple Lie algebras.
		\item $L$ is simple iff it's isomorphic to one of the following Lie algebras:
			\begin{itemize}
				\item $\mathfrak{sl}_n(\mathbb{C}), n\ge 2$ ;
				\item $\mathfrak{o}_n(\mathbb{C}), n\ge 7$ ;
				\item $\mathfrak{sp}_{2n}(\mathcal{C}), n\ge 2$ ;
				\item one of the 5 exceptional complex simple Lie algebras,
					denoted by $\mathfrak{e}_6, \mathfrak{e}_7, \mathfrak{e}_8,
					\mathfrak{f}_4, \mathfrak{g}_2$ respectively.
			\end{itemize}
	\end{enumerate}
\end{theorem}
\begin{remark}
    It can be shown that
	\[
	\mathfrak{o}_2(\mathbb{C}) \cong \mathbb{C}, \quad
	\mathfrak{o}_3(\mathbb{C})\cong \mathfrak{sl}_2(\mathbb{C})
	= \mathfrak{sp}_2(\mathbb{C}),
	\]
	\[
	\mathfrak{o}_4(\mathbb{C}) \cong \mathfrak{sl}_2(\mathbb{C})
	\oplus \mathfrak{sl}_2(\mathbb{C}), \quad
	\mathfrak{o}_5(\mathbb{C})\cong \mathfrak{sp}_4(\mathbb{C}),
	\quad \mathfrak{o}_6(\mathbb{C}) \cong \mathfrak{sl}_4(\mathbb{C}).
	\]
\end{remark}

\subsection{Abelian, nilpotent and solvable Lie algebras}
\label{sub:ad-semisimple, nilpotent Lie algebras}

From now on, let us make the convention that
$L$ always denotes a finite-dimensional complex Lie algebra,
and $V$ always denoted a complex vector space.

Recall that for $x\in \mathfrak{gl}(V)$,
$x$ is said to be semisimple if it's diagonalizable;
and nilpotent if $x^r = 0$ for some $r\ge 1$.

\begin{definition}[ad-semisimple and ad-nilpotent]
	$x$ is \vocab{ad-semisimple} if $\mathrm{ad}(x)\in \mathfrak{gl}(V)$
	is semisimple.
	Similarly define ad-nilpotent.
\end{definition}

\begin{proposition}
	Let $L<\mathfrak{gl}(V), x\in L$.
	If $x$ is semisimple, then it's ad-semisimple. If $x$ is nilpotent,
	then it's ad-nilpotent.
\end{proposition}
\begin{remark}
    If $L$ is semisimple, then the converse of the proposition holds.
\end{remark}

\begin{theorem}
    A Lie algebra $L$ is abelian iff it consists only of ad-semisimple elements.
\end{theorem}

For a Lie algebra $L$, we define two sequences of ideals
\[
L = L^0 \supset L^1 \supset \cdots, \quad L = L^{(0)} \supset L^{(1)} \supset \cdots
\]
by
\[
	L^k = [L, L^{k-1}],\quad L^{(k)} = [L^{(k-1)}, L^{(k-1)}].
\]
\begin{definition}
	$L$ is said to be \vocab{nilpotent} if $L^k = 0$ for some $k$.
	$L$ is said to be \vocab{solvable} if $L^{(k)} = 0$ for some $k $.
\end{definition}

It's easy to see $L^k \supset L^{(k)}$, thus nilpotent Lie algebras
must be solvable.

\begin{proposition}
	Let $L$ be a finite-dimensional Lie algebra, TFAE:
	\begin{itemize}
		\item $L$ is semisimple;
		\item $L$ has no nonzero nilpotent subalgebras;
		\item $L$ has no nonzero solvable subalgebras.
	\end{itemize}
\end{proposition}

\begin{theorem}[Engel]
    Let $L<\mathfrak{gl}(V)$ be a linear Lie algebra consisting of nilpotent
	transformations, then the following statement holds:
	\begin{itemize}
		\item There exists $v\in V$ s.t. $Lv = 0$.
		\item There exists a basis of $V$ s.t. elements in $L$ are all
			upper triangular.
	\end{itemize}
\end{theorem}
\begin{remark}
    This implies that $L$ is nilpotent.
\end{remark}
