%! TeX root = ./main.tex
Since this theorem requires the field to be
algebraically closed, if $T$ is in a smaller field,
we wonder whether $D$ and $N$ is in that field.

Let $A\in \mathbb{R}^{n\times n}$, and $A = D + N$ be
its Jordan decomposition. We'll prove that
$D,N\in \mathbb{R}^{n\times n}$.
By taking conjugates,
\[
A = D + N \implies A = \overline{D} + \overline{N}.
\]
It's clear that $\overline{D} + \overline{N}$ is also
a Jordan decomposition of $A$, so we must have $D = \overline{D}$,
which means $D\in \mathbb{R}^{n\times n}$.

In fact when $\mathbb{R}$ is replaced by any perfect field $F$,
this property still holds.
To prove this we need to introduce the semisimple maps.

\subsection{Semisimple transformations}
\label{sub:Semisimple transformations}
As we've already seen, the ``diagonalizable'' property depends
on the base fields, thus next we'll generalize the concepts
of ``diagonalizable''.

\begin{definition}
	Let $T\in L(V)$,
	\begin{itemize}
		\item We say $T$ is \vocab{simple}(or irreducible) if $V$ has
			no nontrivial $T$-invariant subspaces.
		\item We say $T$ is \vocab{semisimple}(or totally reducible)
			if each $T$-invariant subspace
			$W \subset V$ there exists $T$-invariant subspace $Z$,
			s.t. $V = W\oplus Z$.
	\end{itemize}
\end{definition}
Obviously simple maps are always semisimple.

\begin{proposition}
	Let $T$ be a simple linear operator, then $\forall \alpha\in V\backslash\{0\}$,
	$\alpha$ is a cyclic vector of $T$.
\end{proposition}

\begin{lemma}
	Let $T\in L(V)$.
	\begin{itemize}
		\item If $T$ is semisimple,
			$V' \subset V$ is $T$-invariant, then $T_{V'}$ is semisimple.
		\item If $V = \bigoplus_{i=1}^k V_i$ s.t. $T_{V_i}$ semisimple,
			then $T$ is semisimple as well.
	\end{itemize}
\end{lemma}
\begin{proof}[Proof]
    Suppose $W \subset V'$ is a $T$-invariant subspace.
	Since $T$ is semisimple, $\exists Z \subset V$ s.t. $V = W \oplus Z$,
	and $Z$ is $T$-invariant.

	Let $Z' = Z \cap V'$, we claim that $V' = Z' \oplus W$.

	Clearly $W\cap Z' = \{0\}$ and $W + Z' \subset V'$.
	For all $v\in V'$, $\exists w\in W, z\in Z$ s.t. $v = w + z$,
	since $v, w\in V'$, $z = v - w \in V'$ as well, which means $z\in Z'$.

	For the second part, (We can assmue $k = 2$, but here we won't use it).

	Let $W \subset V$ be a $T$-invariant subspace.
	Since $T_{V_i}$ is semisimple, $\exists Z_i \subset V_i$ s.t.
	\[
		V_i = \left(\left(W + \sum_{j=1}^{i-1} V_j\right)\cap V_i\right)
		\oplus Z_i.
	\]
	Let $Z = \bigoplus_{i=1}^k Z_i$, we claim that $Z \oplus W = V$.
	If $w\in W\cap Z$, then $w = z_1+ \dots + z_k$,
	\[
	z_k = w - z_1 - \dots - z_{k-1}\in Z_k\cap (W + V_1 + \dots + V_{k-1})
	= \{0\}.
	\]
	Thus $z_k = 0$, similarly $z_{k-1} = \dots = z_1 = 0 = w$.

	Note that $W + \sum_{i=1}^{j} V_i \subset W \oplus \sum_{i=1}^{j} Z_i$
	for all $j = 1, \dots, k$, so $V = W \oplus Z$.
\end{proof}

\begin{corollary}
	Let $T\in L(V)$, $T$ is semisimple $\iff$ there exists
	a $T$-invariant decomposition $V = \bigoplus_{i=1}^k V_i$
	s.t. each $T_{V_i}$ is simple.
\end{corollary}

\begin{theorem}
	Let $T\in L(V)$.
	\begin{itemize}
		\item $T$ simple $\iff$ $f_T$ is a prime polynomial;
		\item $T$ semisimple $\iff$ $p_T$ has no multiple factors.
	\end{itemize}
\end{theorem}
\begin{proof}[Proof]
    $T$ simple $ \implies T$ cyclic $ \implies f_T = p_T$,
	so we only need to prove $p_T$ is a prime.

	Otherwise $p_T = gh$,
	\[
	0 = p_T(T) = g(T) h(T),
	\]
	So either $g(T)$ or $h(T)$ is not inversible.
	Thus $\ker(g(T)) \ne \{0\}\implies \ker(g(T)) = V\implies g(T) = 0$,
	contradiction!

	If $T$ is not simple, $\exists W \subset V$, $W$ is $T$-invariant nontrivial
	subspace, so $f_T = f_{T_W} \cdot f_{T_{V / W}}$ is not a prime.

	\vspace{1em}
	$T$ semisimple $ \implies \exists V_i$, $V = \bigoplus_{i=1}^k V_i$,
	such that $T_{V_i}$ is simple $ \implies p_{T_{V_i}}$ is prime.
	\[
	p_T = \lcm(p_{T_{V_1}}, \dots, p_{T_{V_k}})
	\]
	has no multiple factors.

	Conversely if $p_T$ has no multiple factors, consider the primary cyclic
	decomposition of $T$ :
	\[
	V = \bigoplus_i W_i,\quad f_{T_{W_i}} \text{ primary.}
	\]
	Since $p$ has no multiple factors,
	$f_{T_{W_i}} = p_{T_{W_i}}$ is prime polynomial.

	Hence $T_{W_i}$ simple $ \implies T$ semisimple.
\end{proof}

\begin{corollary}
    When $F$ is an algebraically closed field:
	\begin{itemize}
		\item $T$ simple $\iff \dim V = 1$.
		\item $T$ semisimple $\iff$ $T$ is diagonalizable.
	\end{itemize}
\end{corollary}

This corollary means that ``semisimple'' is indeed the
equivalent description of ``diagonalizable'' in the algebraic closure.

Note that whether $p_T$ has multiple factors or not does not
change under \textit{perfect} field extensions.
So ``semisimple'' is a more general property (it stays the same
under more transformations).

Recall that:
\begin{definition}[Perfect fields]
	If for all prime polynomials $p\in F[x]$,
	$p$ has no multiple roots in $\overline{F}$,
	we say $F$ is a \vocab{perfect field}.

	Finite fields, fields with charcter 0 and algebraically closed fields
	are always perfect fields.
\end{definition}

We can check that when $F$ is perfect, $f\in F[x]$ has no multiple factors iff
$f$ has no multiple factors in $\overline{F}[x]$.

Now we can generalize the Jordan decomposition:
